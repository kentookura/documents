The absolute Galois group $\Gal(\Q) = \G_\Q$ of the rational numbers is one of the most important but also one of the most intractable objects in number theory. It is conjectured that every finite group arises as a subgroup of $\G_\Q$. In order to understand this large profinite group, we could study the representations of $\G_\Q$, which are homomorphisms $\G_\Q \to \GL_n(V)$, where $V$ is a finite dimensional $\Q$-vector space. It is not at all clear how to obtain such representations. The goal of this text is to give an overview of the geometry behind \'etale cohomology and to explain how Galois representations of $G_\Q$ arise from \'etale cohomology of varieties over $\Q$
