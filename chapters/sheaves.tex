Much of modern geometry can be formulated in the language of sheaves. Sheaves are data that are given on a topological space that may be glued. One may more generally define sheaves on a category equipped with a so-called Grothendieck topology. A sheaf is a geometric gadget that relates local data (say, on a topological space) to global data. Sheaf theory gives us tools to systematically analyse local-global phenomena which are fundamental in geometry and topology.

\begin{construction}\label{def:opens}
  Let $X$ be a topological space. The open sets of $X$ are partially ordered by inclusion: \[V \le W \quad \text{iff} \quad V \subseteq W.\]
  We may thus consider the category of open sets of $X$, denoted by $\Op(X)$.
\end{construction}
A key observation is the following: In the category $\Op(X)$ pullbacks correspond to intersections and pushouts correspond to unions. By the definition of topological space, this means that $\Op(X)$ has finite pullbacks and arbitrary pushouts.

% TODO: replace res with id

\begin{definition}[Presheaves] 
  Let $X$ be a topological space. A \textit{presheaf of sets} $\mathcal{F}$  is a map which assings
  \begin{itemize}
    \item to each open subset $U$ of $X$ a set $\mathcal{F}(U)$. We call the elements of $\mathcal{F}(U)$ the \textit{sections of $\mathcal{F}$ over $U$}.
    \item to each inclusion of open sets $U \subseteq V$ a \textit{restriction map}
    \[i_*: \mathcal{F}(V) \longrightarrow\mathcal{F}(U).\]
  \end{itemize}
    The restriction maps need to have the following properties: 
    \begin{itemize}
      \item The restriction along the identity $V \subseteq V$ is the identity, so 
            \[id_* = id.\]
      \item For two inclusions $U \subseteq V \subseteq W$, the restricion maps adhere to the identity 
            \[res_{W\kern-0.1em , V}\ \circ\ res_{V\kern-0.1em , U} = res_{W\kern-0.1em , U}.\]
    \end{itemize}
  One may more slickly define a presheaf as a functor $\mathcal{F}: \Op(X)^{\op}\longrightarrow \Set$. To supress notation we will write $s|_V$ instead of $res_{V\kern-0.1em , U}(s).$ Here $s$ is a section over $U$.\\
\end{definition}
\begin{definition}[Sheaves]
  A presheaf $\mathcal{F}$ is a \textit{sheaf} if one can uniquely glue sections. Precisely, this means:
  \begin{enumerate}
    \item For an open set $U$, an open cover $\{U_i\}_{\indexset{i}{I}}$ of $U$ and $s, t \in \mathcal{F}(U)$, if $s|_{U_i} = t|_{U_i}$ for all $i \in I$, then $s = t$.
    \item For an open set $U$, an open cover $\{U_i\}_{\indexset{i}{I}}$ of $U$ and a family of sections
    \[\{s_i \in \mathcal{F}(U_i)\}_{\indexset{i}{I}},\]
     if $s_i|_{U_i \cap U_j} = s_j|_{U_i \cap U_j}$ for all $i,j \in I$, then there exists a section $s \in \mathcal{F}(U)$ such that $s|_{U_i} = s_i$ for all $i \in I$.
  \end{enumerate}
  In words this says that if we have a a cover $\{U_i\}$ of $U$ and a family of sections $\{s_i\}$ which agree on all intersections, we may glue these sections to a section over $U$. Moreover, this section is unique.
\end{definition}

Let $X$ be a topological space and $B$ a base for its topology. It is sometimes convenient to give a sheaf on a base instead of the topology.

% TODO: Examples: constant sheaves, sheaves of rings
The definition for presheaves and sheaves are given almost entirely in categorical language. The notion of presheaf on a category $C$ is readily defined as a functor 
\[\mathcal{F}: C^{\op}\longrightarrow \Set.\]
The missing piece is the notion of a covering in arbitrary categories.. To motivate the definition of a Grothendieck topology, we review the proerties of coverings in the language introduced in Definition \ref{def:opens}.
 \begin{enumerate}
  \item The identity $U \to U$ is an open cover of $U$.
  \item Coverings are stable under composition. Precisely this means: Given a covering \\$\{U_i \to U\}_{i \in I}$ and for each $U_i$ a covering $\{V_{ij} \to U_i\}_{j \in J}$, the composition
        \[\{V_{ij} \to U\}\]
        should be a covering of $U$.
   \item  Coverings are stable under pullback: If we have an inclusion $V \hookrightarrow U$ and a covering $\{U_i \to U\}$, the open sets $\{U_i \times_U V\}$ cover $V$ with the induced projection maps:
   % https://q.uiver.app/?q=WzAsNCxbMCwwLCJcXHtVX2kgXFx0aW1lc19VIFZcXH0iXSxbMCwxLCJWIl0sWzEsMCwiXFx7VV9pXFx9Il0sWzEsMSwiVSJdLFswLDFdLFswLDJdLFsyLDNdLFsxLDMsIiIsMCx7InN0eWxlIjp7InRhaWwiOnsibmFtZSI6Imhvb2siLCJzaWRlIjoidG9wIn19fV1d
      \[\begin{tikzcd}
      	{\{U_i \times_U V\}} & {\{U_i\}} \\
      	V & U.
      	\arrow[from=1-1, to=2-1]
      	\arrow[from=1-1, to=1-2]
      	\arrow[from=1-2, to=2-2]
      	\arrow[hook, from=2-1, to=2-2]
      \end{tikzcd}\]
      Here the fiber product may often be interpreted as the intersection.
\end{enumerate}
\begin{definition}[bases for Grothendieck topologies]
  
\end{definition}

\begin{example}
  We obtain a sheaf on $\R^n$ by associating to any open set $U\subseteq \R^n$ the set \[C(U) = \{f: U \to \R\}\]. The restriction maps are the usual ones. The other properties are also clear.
\end{example}
\begin{example}
  Let $X$ be a topological space. For any abelian group $A$ there is the so called \textit{constant sheaf} $X_A$. To any open set $U \subseteq X$ associate the set \[X_A(U) = \{f: U \to A | f \text{ is locally constant.} \}\]
\end{example}

\section{Maps of sheaves}
Since sheaves are presheaves, we define a map of sheaves $\varphi: \mathcal{F} \to \mathcal{G}$ as a natural transformation $\mathcal{F} \to \mathcal{G}$ between functors. Explicitly this natural transformation consists of a group homomorphism $\varphi_U: \mathcal{F}(U) \to \mathcal{G}(U)$ for each $U \in X$ that is compatible with the restrictions in the sense that the following diagram commutes:

% https://q.uiver.app/?q=WzAsNCxbMCwwLCJcXG1hdGhjYWx7Rn0oVikiXSxbMCwxLCJcXG1hdGhjYWx7Rn0oVSkiXSxbMSwwLCJcXG1hdGhjYWx7R30oVikiXSxbMSwxLCJcXG1hdGhjYWx7R30oVSkiXSxbMCwxLCJyZXNfe1YsVX0iLDJdLFswLDIsIlxcdmFycGhpX1YiXSxbMiwzLCJyZXNfe1YsVX0iXSxbMSwzLCJcXHZhcnBoaV9VIl1d
\[\begin{tikzcd}
	{\mathcal{F}(V)} & {\mathcal{G}(V)} \\
	{\mathcal{F}(U)} & {\mathcal{G}(U)}
	\arrow["{res_{V,U}}"', from=1-1, to=2-1]
	\arrow["{\varphi_V}", from=1-1, to=1-2]
	\arrow["{res_{V,U}}", from=1-2, to=2-2]
	\arrow["{\varphi_U}", from=2-1, to=2-2]
\end{tikzcd}\]

Let $\varphi : \mathcal{F} \to \mathcal{G}$ be a morphism sheaves of abelian groups on $X$. The sheaf kernel of the map $\varphi$ is the presheaf $U \mapsto ker(\varphi)$. The sheaf condition on $\mathcal{G}$ ensures that this is in fact a sheaf.



There are a number of properties of sheaves which are important for applications. These properties govern the behavior of the cohomology.

\begin{definition}
  A sheaf is called
  \begin{enumerate}
    \item flabby if the restriction maps are surjective,
    \item soft if any section over a closed subset can be extended to a global section,
    \item acyclic if the higher cohomology of the sheaf vanishes. 
  \end{enumerate} 
\end{definition}

\section{The \'Etale site of a Scheme}
%\section{Sheaf Semantics}
%It is possible to interpret intuitionistic first-order logic in an arbitrary topos. In the case of sheaves, this is done using the following rules