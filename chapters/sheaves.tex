%Much of modern geometry can be formulated in the language of sheaves. Sheaves are data that are given on a topological space that may be glued. One may more generally define sheaves on a category equipped with a so-called Grothendieck topology. A sheaf is a geometric gadget that relates local data (say, on a topological space) to global data. Sheaf theory gives us tools to systematically analyse local-global phenomena which are fundamental in geometry and topology.

\begin{construction}\label{def:opens}
  Let $X$ be a topological space. The open sets of $X$ are partially ordered by inclusion: \[V \le W \quad \text{when} \quad V \subseteq W.\]
  We may thus, as with any poset, consider the category of open sets of $X$, sometimes denoted by $\Op(X)$. By abuse of notation, we will often write $X$ instead of $\Op(X)$.
\end{construction}
\begin{remark}
 In the category $\Op(X)$ pullbacks correspond to intersections and pushouts correspond to unions. By the definition of topological space, this means that $\Op(X)$ has finite pullbacks and arbitrary pushouts. 
\end{remark}

\begin{definition}
	Let $\mathcal{C}$ be a category. A Grothendieck topology on $C$ is given by a set $\text{Cov}(\mathcal{C})$ of families of morphisms ${U_i \to U}$ for each object $U$ of $C$ with the following properties:
	\begin{enumerate}
		\item If $V \to U$ is an isomorphism, then $\{V \to U\} \in \mathcal{C}$
		\item If $\{U_i \to U\} \in \text{Cov}(\mathcal(C))$, and for each $U_i$ we have $\{V_{ij} \to Ui\}$, the composition $\{V_{ij} \to U\}$ is in $\text{Cov}(\mathcal{C})$.
		\item If $\{U_i \to U\} \in \text{Cov}(\mathcal(C))$, and $V \to U$ is a morphism, then $U_i \times_U V$ exists and 
		$\{U_i \times_U V \to V\}$ is in $\text{Cov}(\mathcal{C})$.
	\end{enumerate}	
	An element $\{U_i \to U\}$ of $\text{Cov}(\mathcal{C})$ is called a cover of $U$. A category $C$ together with a Grothendieck topology is called a \textit{site}.
\end{definition}
\begin{remark}
	A topological space $X$ may be equipped with the structure of a site by declaring the coverings of open sets $U \subseteq X$ to be those families of open sets $\{U_i\}$ such that $\bigcup U_i = U$.
\end{remark}


\begin{definition}[Presheaves] 
  Let $X$ be a category. A \textit{presheaf of sets} $\mathcal{F}$ on $X$ is a functor $X^{op} \to \Set$. Given an morphism $U \subseteq V$, the corresponding map $\mathcal{F}(V) \to \mathcal{F}(U)$ is called the \textit{restriction map}. The elements of $\mathcal{F}(U)$ are called the sections of $\mathcal{F}$ over $U$. On sections one often denotes the map by $s \mapsto s|_U$.
\end{definition}

\begin{definition}[Sheaves]
  Let $(C,J)$ be a site.  A presheaf $\mathcal{F}$ on $C$ is a \textit{sheaf}, if for any covering $\{U_i\}$ of $U$, the following diagram is an equalizer:
  \begin{equation} \label{equalizer:1}
    \mathcal{F}(U) \to \prod_{i} \mathcal{F}(U_i) \xbigtoto{} \prod_{i,j} \mathcal{F}(U_i \cap U_j)
  \end{equation}
\end{definition}

We need to explain the above diagram. First, note that if two subsets $U_1$ and $U_2$ form a cover of $X$, there are the inclusions

% https://q.uiver.app/?q=WzAsNCxbMSwwLCJVX3sxMn0iXSxbMCwxLCJVXzEiXSxbMiwxLCJVXzIiXSxbMSwyLCJYIl0sWzAsMV0sWzAsMl0sWzEsM10sWzIsM11d
\[\begin{tikzcd}
	& {U_{12}} \\
	{U_1} && {U_2} \\
	& X,
	\arrow[from=1-2, to=2-1]
	\arrow[from=1-2, to=2-3]
	\arrow[from=2-1, to=3-2]
	\arrow[from=2-3, to=3-2]
\end{tikzcd}\]
where $U_{12}$ is shorthand for $U_1 \times_X U_2 = U_1 \cap U_2$.

By the contravariance of $\mathcal{F}$, there is a corresponding diagram

% https://q.uiver.app/?q=WzAsNCxbMCwxLCJcXG1hdGhjYWx7Rn0oVV9pKSJdLFsxLDAsIlxcbWF0aGNhbHtGfShVX3tpan0pIl0sWzIsMSwiXFxtYXRoY2Fse0Z9KFVfaikiXSxbMSwyLCJcXG1hdGhjYWx7Rn0oWCkiXSxbMCwxXSxbMiwxXSxbMywwXSxbMywyXV0=
\[\begin{tikzcd}
	& {\mathcal{F}(U_{12})} \\
	{\mathcal{F}(U_1)} && {\mathcal{F}(U_2)} \\
	& {\mathcal{F}(X)}.
	\arrow[from=2-1, to=1-2]
	\arrow[from=2-3, to=1-2]
	\arrow[from=3-2, to=2-1]
	\arrow[from=3-2, to=2-3]
\end{tikzcd}\]

When taking the product in the middle row, we obtain the diagram 
% https://q.uiver.app/?q=WzAsMyxbMiwwLCJcXG1hdGhjYWx7Rn0oVV97MTJ9KSJdLFswLDAsIlxcbWF0aGNhbHtGfShYKSJdLFsxLDAsIlxcbWF0aGNhbHtGfShVXzEpXFx0aW1lc1xcbWF0aGNhbHtGfShVXzIpIl0sWzEsMl0sWzIsMCwiIiwwLHsib2Zmc2V0IjoxfV0sWzIsMCwiIiwwLHsib2Zmc2V0IjotMX1dXQ==
\[\begin{tikzcd}
	{\mathcal{F}(X)} & {\mathcal{F}(U_1)\times\mathcal{F}(U_2)} & {\mathcal{F}(U_{12})}.
	\arrow[from=1-1, to=1-2]
	\arrow[shift right=1, from=1-2, to=1-3]
	\arrow[shift left=1, from=1-2, to=1-3]
\end{tikzcd}\]

Given two section $s_1 \in \mathcal{F}(U_1)$ and $s_2 \in \mathcal{F}(U_2)$, the two parallel arrows map $(s_1, s_2)$ to $s_1|_{U_{12}}$ and $s_2|_{U_{12}}$. To say that $\mathcal{F}(X)$ is an equalizer in the above diagram amounts to saying that the sections of $\mathcal{F}(X)$ are recoverable from those sections on the sets $U_1$ and $U_2$ which agree on the overlap $U_{12}$. The diagram\eqref{equalizer:1} is the straightforward generalisation of the preceding discussion for a cover consisting of more than 2 open sets.




The purpose of the next definition is to capture the local behavior of a sheaf around a point.
\begin{definition}
  Let $X$ be a topological space and  $x \in X$ be a point. The stalk of $\mathcal{F}$ at $x$ is defined to be \[ \mathcal{F}_x \coloneqq \colim_{U \ni x}\mathcal{F}(U).\]
\end{definition}
\begin{proposition}
	Let $\mathcal{F}$ be a sheaf and $s,t$ be two sections of $\mathcal{F}(U)$. Then $s = t$ if and only if $s_x = t_x$ for all
\end{proposition}

\begin{example}
  Consider a manifold $M$, together with the sheaf which sends an open set $U$ to the set of real valued functions $f: U \to \R$. The stalks $\mathcal{F}_x$ of this sheaf are local rings. Explicitly, the stalk $\mathcal{F}_x$ is  $\{ (U,s) | x \in U,  s \in \mathcal{F}(U) \}/\sim$, where $(U,s)$ and $(V, t)$ are equivalent if there is an open set $W \subseteq U \cap V$ with $s|_W = t|_w$.

\end{example}

\begin{example}
  The stalk of the structure sheaf of $\Spec(R)$ at a point $p$ is the localization $R_p$. Its maximal ideal is generated by $p$.
\end{example}


\section{Maps of sheaves}
Since sheaves are presheaves, we define a map of sheaves $\varphi: \mathcal{F} \to \mathcal{G}$ as a natural transformation $\mathcal{F} \to \mathcal{G}$ between functors. Explicitly this natural transformation consists of a group homomorphism $\varphi_U: \mathcal{F}(U) \to \mathcal{G}(U)$ for each $U \in X$ that is compatible with the restrictions in the sense that the following diagram commutes:

% https://q.uiver.app/?q=WzAsNCxbMCwwLCJcXG1hdGhjYWx7Rn0oVikiXSxbMCwxLCJcXG1hdGhjYWx7Rn0oVSkiXSxbMSwwLCJcXG1hdGhjYWx7R30oVikiXSxbMSwxLCJcXG1hdGhjYWx7R30oVSkiXSxbMCwxLCJyZXNfe1YsVX0iLDJdLFswLDIsIlxcdmFycGhpX1YiXSxbMiwzLCJyZXNfe1YsVX0iXSxbMSwzLCJcXHZhcnBoaV9VIl1d
\[\begin{tikzcd}
	{\mathcal{F}(V)} & {\mathcal{G}(V)} \\
	{\mathcal{F}(U)} & {\mathcal{G}(U)}
	\arrow["{\cdot|_U}"', from=1-1, to=2-1]
	\arrow["{\varphi_V}", from=1-1, to=1-2]
	\arrow["{\cdot|_U}", from=1-2, to=2-2]
	\arrow["{\varphi_U}", from=2-1, to=2-2]
\end{tikzcd}\]

We obtain the category $\mathsf{Sh}(C,J)$ of sheaves on $C$ with respect to the topology $J$. This is in fact a full subcategory of presheaves $\Set^{C^{op}}$. 

Let $\varphi : \mathcal{F} \to \mathcal{G}$ be a morphism sheaves of abelian groups on $X$. The sheaf kernel of the map $\varphi$ is the presheaf $U \mapsto \ker(\varphi)$. The sheaf condition on $\mathcal{G}$ ensures that this presheaf is in fact a sheaf. 
The map $U \to \text{im}\varphi(U)$ does not in general define a sheaf, but there is an "optimal way" to turn a presheaf into a sheaf. This is the associated sheaf functor.

\begin{theorem}
  Given a presheaf $\mathcal{F}$ there is a unique morphism of presheaves $a : \mathcal{F} \to \mathcal{F}^{+}$ with the property that for any sheaf $\mathcal{G}$ and 
\end{theorem}

The terminal object in the category of sheaves is the constant sheaf associated to the one points set $*$.

\begin{definition}
  A sequence of sheaves $\to \mathcal{F}_{i-1} \to \mathcal{F}_{i} \to \mathcal{F}_{i+1} \to$
	is exact if for each $i$, $\ker \varphi_i = im \varpi_{i-1}$.
\end{definition}

%% TODO: replace res with id
%
%  One may more slickly define a presheaf as a functor $\mathcal{F}: \Op(X)^{\op}\longrightarrow \Set$. To supress notation we will write $s|_V$ instead of $res_{V\kern-0.1em , U}(s).$ Here $s$ is a section over $U$.\\
%\end{definition}
%\begin{definition}[Sheaves]
%  A presheaf $\mathcal{F}$ is a \textit{sheaf} if one can uniquely glue sections. Precisely, this means:
%  \begin{enumerate}
%    \item For an open set $U$, an open cover $\{U_i\}_{\indexset{i}{I}}$ of $U$ and $s, t \in \mathcal{F}(U)$, if $s|_{U_i} = t|_{U_i}$ for all $i \in I$, then $s = t$.
%    \item For an open set $U$, an open cover $\{U_i\}_{\indexset{i}{I}}$ of $U$ and a family of sections
%    \[\{s_i \in \mathcal{F}(U_i)\}_{\indexset{i}{I}},\]
%     if $s_i|_{U_i \cap U_j} = s_j|_{U_i \cap U_j}$ for all $i,j \in I$, then there exists a section $s \in \mathcal{F}(U)$ such that $s|_{U_i} = s_i$ for all $i \in I$.
%  \end{enumerate}
%  In words this says that if we have a a cover $\{U_i\}$ of $U$ and a family of sections $\{s_i\}$ which agree on all intersections, we may glue these sections to a section over $U$. Moreover, this section is unique.
%\end{definition}
%
%% TODO: Examples: constant sheaves, sheaves of rings
%The definition for presheaves and sheaves are given almost entirely in categorical language. The notion of presheaf on a category $C$ is readily defined as a functor 
%\[\mathcal{F}: C^{\op}\longrightarrow \Set.\]
%The missing piece is the notion of a covering in arbitrary categories.. To motivate the definition of a Grothendieck topology, we review the proerties of coverings in the language introduced in Definition \ref{def:opens}.
% \begin{enumerate}
%  \item The identity $U \to U$ is an open cover of $U$.
%  \item Coverings are stable under composition. Precisely this means: Given a covering \\$\{U_i \to U\}_{i \in I}$ and for each $U_i$ a covering $\{V_{ij} \to U_i\}_{j \in J}$, the composition
%        \[\{V_{ij} \to U\}\]
%        should be a covering of $U$.
%   \item  Coverings are stable under pullback: If we have an inclusion $V \hookrightarrow U$ and a covering $\{U_i \to U\}$, the open sets $\{U_i \times_U V\}$ cover $V$ with the induced projection maps:
%   % https://q.uiver.app/?q=WzAsNCxbMCwwLCJcXHtVX2kgXFx0aW1lc19VIFZcXH0iXSxbMCwxLCJWIl0sWzEsMCwiXFx7VV9pXFx9Il0sWzEsMSwiVSJdLFswLDFdLFswLDJdLFsyLDNdLFsxLDMsIiIsMCx7InN0eWxlIjp7InRhaWwiOnsibmFtZSI6Imhvb2siLCJzaWRlIjoidG9wIn19fV1d
%      \[\begin{tikzcd}
%      	{\{U_i \times_U V\}} & {\{U_i\}} \\
%      	V & U.
%      	\arrow[from=1-1, to=2-1]
%      	\arrow[from=1-1, to=1-2]
%      	\arrow[from=1-2, to=2-2]
%      	\arrow[hook, from=2-1, to=2-2]
%      \end{tikzcd}\]
%      Here the fiber product may often be interpreted as the intersection.
%\end{enumerate}
%\begin{definition}[bases for Grothendieck topologies]
%  
%\end{definition}
%
%\begin{example}
%  We obtain a sheaf on $\R^n$ by associating to any open set $U\subseteq \R^n$ the set \[C(U) = \{f: U \to \R\}\]. The restriction maps are the usual ones. The other properties are also clear.
%\end{example}
%\begin{example}
%  Let $X$ be a topological space. For any abelian group $A$ there is the so called \textit{constant sheaf} $X_A$. To any open set $U \subseteq X$ associate the set \[X_A(U) = \{f: U \to A | f \text{ is locally constant.} \}\]
%\end{example}
%
%\section{Maps of sheaves}
%Since sheaves are presheaves, we define a map of sheaves $\varphi: \mathcal{F} \to \mathcal{G}$ as a natural transformation $\mathcal{F} \to \mathcal{G}$ between functors. Explicitly this natural transformation consists of a group homomorphism $\varphi_U: \mathcal{F}(U) \to \mathcal{G}(U)$ for each $U \in X$ that is compatible with the restrictions in the sense that the following diagram commutes:
%
%% https://q.uiver.app/?q=WzAsNCxbMCwwLCJcXG1hdGhjYWx7Rn0oVikiXSxbMCwxLCJcXG1hdGhjYWx7Rn0oVSkiXSxbMSwwLCJcXG1hdGhjYWx7R30oVikiXSxbMSwxLCJcXG1hdGhjYWx7R30oVSkiXSxbMCwxLCJyZXNfe1YsVX0iLDJdLFswLDIsIlxcdmFycGhpX1YiXSxbMiwzLCJyZXNfe1YsVX0iXSxbMSwzLCJcXHZhcnBoaV9VIl1d
%\[\begin{tikzcd}
%	{\mathcal{F}(V)} & {\mathcal{G}(V)} \\
%	{\mathcal{F}(U)} & {\mathcal{G}(U)}
%	\arrow["{res_{V,U}}"', from=1-1, to=2-1]
%	\arrow["{\varphi_V}", from=1-1, to=1-2]
%	\arrow["{res_{V,U}}", from=1-2, to=2-2]
%	\arrow["{\varphi_U}", from=2-1, to=2-2]
%\end{tikzcd}\]
%
%Let $\varphi : \mathcal{F} \to \mathcal{G}$ be a morphism sheaves of abelian groups on $X$. The sheaf kernel of the map $\varphi$ is the presheaf $U \mapsto ker(\varphi)$. The sheaf condition on $\mathcal{G}$ ensures that this is in fact a sheaf.
%
%
%
%There are a number of properties of sheaves which are important for applications. These properties govern the behavior of the cohomology.
%
%\begin{definition}
%  A sheaf is called
%  \begin{enumerate}
%    \item flabby if the restriction maps are surjective,
%    \item soft if any section over a closed subset can be extended to a global section,
%    \item acyclic if the higher cohomology of the sheaf vanishes. 
%  \end{enumerate} 
%\end{definition}
%
%\section{The \'Etale site of a Scheme}
%%\section{Sheaf Semantics}
%%It is possible to interpret intuitionistic first-order logic in an arbitrary topos. In the case of sheaves, this is done using the following rules