We have seen strong analogies in the behavior of \'etale maps and open sets. \'Etale maps allow us to trivialize certain geometric constructions as if we were working in a finer topology than the Zariski topology. Furthermore, the theory of surjective\'etale maps should be interpreted as the theory of covering spaces for schemes. We will now further develop some topological analogies, equipping the category $\Et/X$ with a Grothendieck topology.

\section{Topological Spaces and Locales}

In this section we introduce two notions related to topology. The first notion, which is central to topos theory is the notion of site. The definition of sites will allow us to speak of sheaves on categories equipped with a so-called Grothendieck topology. Taking the category of sheaves on a site yields a rich category to do geometry in. We will in particular be interested in the \'etale topology on schemes. As we have seen, \'etale morphisms of schemes are the correct formulation of local homeomorphisms for schemes as we are able to trivialize geometric information (in the sense of covering spaces and fiber bundles) over \'etale morphisms $S \to X$ which is not always possible over the open sets of $X$.

\subsection{Locales}
Locales are important in topos theory because there is an adjunction

% https://q.uiver.app/?q=WzAsMixbMCwwLCJcXG1hdGhzZntUb3BvaX0iXSxbMSwwLCJcXG1hdGhzZntMb2NhbGVzfSJdLFswLDEsInN1Yl97KC0pfVxcbWF0aGJmezF9IiwwLHsib2Zmc2V0IjotMX1dLFsxLDAsIlxcbWF0aHNme1NofSIsMCx7Im9mZnNldCI6LTF9XV0=
\[\begin{tikzcd}
		{\Topos} & {\Loc}
		\arrow["{\mathsf{sub}_{(-)} \mathbf{1}} ", shift left=1, from=1-1, to=1-2]
		\arrow["{\sh}", shift left=1, from=1-2, to=1-1]
	\end{tikzcd}
\]

which exhibits $\Loc$ as a reflective subcategory of $\Topos$, and every topos $\mathcal{E}$ has a so-called \textit{localic reflection}. Here the functor $\mathsf{sub}_{(-)}\mathbf{1}$ takes a topos $\mathcal{E}$ and returns $\mathsf{sub}_{\mathcal{E}}(\mathbf{1})$, the subobjects of the final object $\mathbf{1}$ of $\mathcal{E}$, which is the localic reflection of $\mathcal{E}$. The functor $\sh$ simply takes a locale $L$ to its category of sheaves. Here $\sh$ is the right adjoint. The following definition is based on the algebraic behavior of open sets of a topological space: Viewing a space $X$ as a poset, the intersection and union satisfy a distributivity law.

\begin{definition}
	A frame $\mathcal{O}$ is a partially ordered set $(U, \le)$ with all coproducts $\bigvee U_i$ and all finite products $U_i \land U_j$ which satisfies the distributive law
	\[
		U \land (\bigvee U_i) = \bigvee (U \land U_i).
	\]
	A \textit{frame homomorphism} is a homomorphism of posets preserving finite products and arbitrary coproducts. Any topological space $X$ gives rise to a frame by considering the poset of opens $\mathcal{O}(X)$. A continuous map $f: X \to Y$ between spaces gives rise to a frame homomorphism $f^*: \mathcal{O}(Y) \to \mathcal{O}(X)$, so there is a contravariant functor from topological spaces to frames. We define the category of locales to be the opposite category of the category of frames. A (continuous) map of locales $f: X \to Y$ is then nothing but a map between frames from $Y$ to $X$.
\end{definition}

\begin{construction}\label{def:opens}
	Let $X$ be a topological space. The open sets of $X$ are partially ordered by inclusion: $V \le W \quad \text{when} \quad V \subseteq W$. We may thus, as with any poset, consider the category of open sets of $X$, sometimes denoted by $\mathcal{O}(X)$. By abuse of notation, we will often write $X$ instead of $\mathcal{O}(X)$.
\end{construction}

\begin{remark}
	In the category $\mathcal{O}(X)$ pullbacks correspond to intersections and pushouts correspond to unions. By the definition of topological space, this means that $\mathcal{O}p(X)$ has finite pullbacks and arbitrary pushouts.
\end{remark}


\begin{example}
	The frame structure on the lattice of open sets for the Zariski topology is given by
	\begin{align*}
		\Big( \bigvee D_{f_i}\Big) \wedge D_g & = D(\Sigma f_i) \wedge D_g     \\
		                                      & = D(\Sigma f_i \, g)           \\
		                                      & = \bigvee (D_{f_i} \wedge D_g)
	\end{align*}
\end{example}

\begin{remark}
	A topos equivalent to the category of sheaves on a locale is called a \textit{localic topos}. Note that for a topos $\mathcal{E}$, there may be many sites which give rise to a sheaf topos equivalent to $\mathcal{E}$. In our case, the sheaf topos $\sh_{\et}(X)$ for a scheme $X$ is a priori not localic, as the presence of automorphisms implies that the underlying category for $\Et/X$ is not a poset. By \cite{SIGL}, \textbf{IX}. 5. Theorem 1, A topos $\mathcal{E}$ is localic if and only if there exists a site for $\mathcal{E}$ with a poset as underlying category.
\end{remark}

% https://q.uiver.app/?q=WzAsMyxbMCwxLCJcXGJ1bGxldCJdLFsxLDEsIlxcYnVsbGV0Il0sWzEsMCwiXFxidWxsZXQiXSxbMCwxLCIiLDAseyJvZmZzZXQiOjF9XSxbMSwwLCIiLDAseyJvZmZzZXQiOjF9XSxbMiwwLCIiLDAseyJvZmZzZXQiOi0xfV0sWzIsMSwiIiwwLHsib2Zmc2V0IjotMX1dLFswLDIsIiIsMCx7Im9mZnNldCI6LTF9XSxbMSwyLCIiLDAseyJvZmZzZXQiOi0xfV1d
\[\begin{tikzcd}
		& \Topos \\
		\Top & \Loc
		\arrow[shift right=1, from=2-1, to=2-2]
		\arrow[shift right=1, from=2-2, to=2-1]
		\arrow[shift left=1, from=1-2, to=2-1]
		\arrow[shift left=1, from=1-2, to=2-2]
		\arrow[shift left=1, from=2-1, to=1-2]
		\arrow[shift left=1, from=2-2, to=1-2]
	\end{tikzcd}\]

There is also an adjunction between topological spaces and locales, which restricts to an quivalence between so-called sober topological spaces and spatial locales.

\subsection{Subobjects and Truth Values}


\begin{definition}
	Let $\mathcal{C}$ be a category. A Grothendieck topology on $C$ is given by a set $\text{Cov}(\mathcal{C})$ of families of morphisms ${U_i \to U}$ for each object $U$ of $C$ with the following properties:
	\begin{enumerate}
		\item If $V \to U$ is an isomorphism, then $\{V \to U\} \in \mathcal{C}$
		\item If $\{U_i \to U\} \in \text{Cov}(\mathcal{C})$, and for each $U_i$ we have $\{V_{ij} \to Ui\}$, the composition $\{V_{ij} \to U\}$ is in $\text{Cov}(\mathcal{C})$.
		\item If $\{U_i \to U\} \in \text{Cov}(\mathcal{C})$, and $V \to U$ is a morphism, then $U_i \times_U V$ exists and
		      $\{U_i \times_U V \to V\}$ is in $\text{Cov}(\mathcal{C})$.
	\end{enumerate}
	An element $\{U_i \to U\}$ of $\text{Cov}(\mathcal{C})$ is called a cover of $U$. A category $C$ together with a Grothendieck topology is called a \textit{site}.
\end{definition}

\begin{example}
	A topological space $X$ may be equipped with the structure of a site by declaring the coverings of open sets $U \subseteq X$ to be those families of open sets $\{U_i\}$ contained in $U$ such that $\displaystyle\bigcup U_i = U$. Let $\mathcal{O}(X)$ be the corresponding locale. A family of morphisms $\{U_i \to U\}$ is a cover if $U = \bigvee U_i$
\end{example}

\begin{example}
	The category of sets may be equipped with a topology where coverings consist of families of set maps $\{f_i \colon X_i \to X\}$ such that $\displaystyle\bigcup_i f_i(X_i) = X$.
\end{example}

\begin{example}
	The main site we will consider is the \'etale site of a scheme $X$. The underlying category has as objects \'etale morphisms $f \colon S \to X$ and as morphisms commuting triangles
	\[
		% https://q.uiver.app/?q=WzAsMyxbMCwwLCJTIl0sWzIsMCwiUyciXSxbMSwxLCJYIl0sWzAsMSwiXFx2YXJwaGkiXSxbMSwyLCJmJyJdLFswLDIsImYiLDJdXQ==
		\begin{tikzcd}
			S && {S'} \\
			& X
			\arrow["\varphi", from=1-1, to=1-3]
			\arrow["{f'}", from=1-3, to=2-2]
			\arrow["f"', from=1-1, to=2-2]
		\end{tikzcd}
	\]
	such that $f = f' \circ \varphi$.
	From (\textellipsis) it follows that the set of families jointly surjective \'etale morphisms form a site. Note that a key difference to the example from classical topology is that there may be more than one morphism between ``opens''! The \'etale topology topology is finer than the Zariski topology in the sense that an open immersion $U \to X$ is always \'etale, so each open set $U \in X$ may be thought of as an \'etale open set. There are a couple of variations on this notion:

	\begin{enumerate}
		\item The previous site is sometimes called the \textit{small \'etale site of $X$}, denoted by $\Et/X$. The \textit{big \'etale site} consists of all schemes over $X$, where covering families are again given by \'etale covers.
		\item We can also consider the site of finite \'etale morphisms, whose objects are finite \'etale schemes over $X$.
	\end{enumerate}
\end{example}

We want to transfer topological notions and intuitions to work for sites. Some definitions are obvious: For a point $x \in X$, w


\begin{definition}[Presheaves]
	Let $C$ be a category. A \textit{presheaf of sets} on $C$ is a functor $F \colon C^{\op} \to \Set$. Given an morphism $f \colon U \to V$, the corresponding map $F(V) \to F(U)$ is called the \textit{restriction map}. The elements of $F(U)$ are called the sections of $F$ over $U$. If the map $f \colon  U \to V$ we are restricting along is clear from context, we will denote the action of the induced map on $s \in F(V)$ by $s \mapsto s|_U$. We may also consider presheaves valued in an arbitrary category $D$. For instance, a presheaf of groups on $C$ is a presheaf $F \colon C^{op} \to \Ab$ valued in the category $\Ab$ of abelian groups. The restriction maps are group homomorphisms.
\end{definition}

\begin{example}
	A group $G$ determines a category with one object and only isomorphisms. A presheaf of sets on this category is simply a set $S$ endowed with a right $G$-action.
\end{example}

\begin{example}
	Let $C$ be any locally small category and $X$ an object of $C$. Then the association $T \to \Hom(T,X)$ is a presheaf of sets since any morphism $\varphi \colon T \to S$ induces a map between $\Hom$-sets
	\begin{align*}
		(- \circ \varphi) \colon \Hom(T,S) & \to \Hom(T,X)        \\
		f                                  & \to f \circ \varphi.
	\end{align*}
	We call the presheaf $\Hom(-,X)$ the \textit{presheaf represented by $X$}.
\end{example}

\begin{definition}[Sheaves]
	Let $(C,J)$ be a site. A presheaf $F$ on $C$ is a \textit{sheaf}, if for any covering $\{U_i\}$ of $U$, the following diagram is an equalizer:
	\begin{equation} \label{diagram:equalizer}
		F(U) \to \prod_{i} F(U_i) \xbigtoto{} \prod_{i,j} F(U_i \times_U U_j)
	\end{equation}
	If the map $F(U) \to \displaystyle\prod_i F(U_i)$ is injective, then we say that $F$ is a \textit{separated presheaf}. If the reference to the site $J$ is clear from context, we will also say that $F$ is a sheaf on $C$.
\end{definition}

We need to explain the above diagram. In the case that $X$ is covered by two objects $U_1$ and $U_2$, there are the morphisms $U_{12} \to U_1 \to X$ and $U_{12} \to U_2 \to X$, fitting into  where $U_{12}$ is shorthand for $U_1 \times_X U_2$ or $U_1 \cap U_2$. By the contravariance of $F$, there obtain two diagrams~\ref{cover} and~\ref{restriction}

\begin{figure}[h!]
	\tikzset{column sep=small, ampersand replacement=\&}
	\begin{floatrow}
		\centering
		\ffigbox{
			% https://q.uiver.app/?q=WzAsNCxbMSwwLCJVX3sxMn0iXSxbMCwxLCJVXzEiXSxbMiwxLCJVXzIiXSxbMSwyLCJYIl0sWzAsMV0sWzAsMl0sWzEsM10sWzIsM11d
			\begin{tikzcd}
				\& {U_{12}} \\
				{U_1} \& \& {U_2} \\
				\& X
				\arrow[from=1-2, to=2-1]
				\arrow[from=1-2, to=2-3]
				\arrow[from=2-1, to=3-2]
				\arrow[from=2-3, to=3-2]
			\end{tikzcd}
		}{\caption{A covering of $X$}\label{cover}}

		% https://q.uiver.app/?q=WzAsNCxbMCwxLCJcXG1hdGhjYWx7Rn0oVV9pKSJdLFsxLDAsIlxcbWF0aGNhbHtGfShVX3tpan0pIl0sWzIsMSwiXFxtYXRoY2Fse0Z9KFVfaikiXSxbMSwyLCJcXG1hdGhjYWx7Rn0oWCkiXSxbMCwxXSxbMiwxXSxbMywwXSxbMywyXV0=
		\ffigbox{
			\begin{tikzcd}
				\& {F (U_{12})} \\
				{F (U_1)} \& \& {F (U_2)} \\
				\& {F (X)}.
				\arrow[from=2-1, to=1-2]
				\arrow[from=2-3, to=1-2]
				\arrow[from=3-2, to=2-1]
				\arrow[from=3-2, to=2-3]
			\end{tikzcd}} {\caption{Restrictions}\label{restriction}}
	\end{floatrow}
	\tikzset{column sep=small, ampersand replacement=&}
\end{figure}

When taking the product in the middle row of~\ref{restriction}, we obtain the diagram
% https://q.uiver.app/?q=WzAsMyxbMiwwLCJcXG1hdGhjYWx7Rn0oVV97MTJ9KSJdLFswLDAsIlxcbWF0aGNhbHtGfShYKSJdLFsxLDAsIlxcbWF0aGNhbHtGfShVXzEpXFx0aW1lc1xcbWF0aGNhbHtGfShVXzIpIl0sWzEsMl0sWzIsMCwiIiwwLHsib2Zmc2V0IjoxfV0sWzIsMCwiIiwwLHsib2Zmc2V0IjotMX1dXQ==
\[
	\begin{tikzcd}
		{F(X)} & {F(U_1)\times F(U_2)} & {F(U_{12})}.
		\arrow[from=1-1, to=1-2]
		\arrow[shift right=1, from=1-2, to=1-3]
		\arrow[shift left=1, from=1-2, to=1-3]
	\end{tikzcd}
\]

Given two section $s_1 \in F(U_1)$ and $s_2 \in F(U_2)$, the two parallel arrows map $(s_1, s_2)$ to $s_1|_{U_{12}}$ and $s_2|_{U_{12}}$. To say that $F(X)$ is an equalizer in the above diagram amounts to saying that the sections of $F(X)$ are recoverable from those sections on the sets $U_1$ and $U_2$ which agree on the overlap $U_{12}$. The diagram\eqref{diagram:equalizer} is the straightforward generalisation of the preceding discussion for a cover consisting of more than 2 open sets.

We define a map of sheaves $\varphi \colon F \to \Sh{G}$ as a natural transformation $F \to \Sh{G}$ between functors. Explicitly this natural transformation consists of a morphism $\varphi_U \colon F(U) \to \Sh{G}(U)$ for each $U \in X$ such that the following diagram commutes:

% https://q.uiver.app/?q=WzAsNCxbMCwwLCJcXG1hdGhjYWx7Rn0oVikiXSxbMCwxLCJcXG1hdGhjYWx7Rn0oVSkiXSxbMSwwLCJcXG1hdGhjYWx7R30oVikiXSxbMSwxLCJcXG1hdGhjYWx7R30oVSkiXSxbMCwxLCJyZXNfe1YsVX0iLDJdLFswLDIsIlxcdmFycGhpX1YiXSxbMiwzLCJyZXNfe1YsVX0iXSxbMSwzLCJcXHZhcnBoaV9VIl1d
\[
	\begin{tikzcd}
		{F(V)} & {\Sh{G}(V)} \\
		{F(U)} & {\Sh{G}(U)}
		\arrow["{\cdot|_U}"', from=1-1, to=2-1]
		\arrow["{\varphi_V}", from=1-1, to=1-2]
		\arrow["{\cdot|_U}", from=1-2, to=2-2]
		\arrow["{\varphi_U}", from=2-1, to=2-2]
	\end{tikzcd}
\]
If we consider sheaves of groups, this means that the restriction are group homomorphisms.

We obtain the category $\sh(C,J)$ of sheaves on $C$ with respect to the topology $J$. This is in fact a full subcategory of presheaves $\Set^{C^{op}}$. We call categories equivalent to categories of sheaves on a site Grothendieck toposes. The category of sheaves $\sh(X, \et_X)$ on the \'etale toplogy of a scheme $X$ is called the \textit{\'etale topos of $X$}. We have seen that a number of geometric constructions from algebraic topology fail in the Zariski topology. The \'etale topology captures these phenomena, in essence because we may speak of ``local triviality'' in the \'etale setting. In this sense the \'etale topos enables us to reason about schemes more geometrically. So if we want to understand the geometry a scheme $X$, we should also study the \'etale topos of $X$.

\section{Representables and the Yoneda Lemma}
In this section we will now prove the important fact that all representable presheaves are sheaves in the \'etale topology. This gives us a rich supply of sheaves on $\Et/X$ to study.

\begin{lemma}\label{sheaf:disjoint}
	Let $F$ be a sheaf on the Zariski topology on a scheme $X = \coprod U_i$ which is the disjoint union of subschemes $U_i$. Then $F(X) = \prod F(U_i)$.
\end{lemma}

\begin{proof}
	The pullback $U_i \times_X U_j$ is empty for $i \neq j$. Evaluating a sheaf on the empty set yields the terminal object $\{*\}$, so the sheaf condition becomes
	\[
		F(X) \to \prod_i F(U_i) \xbigtoto{} \prod_{i,j} \{*\} \simeq \{*\},
	\]
	which yields the required isomorphism.
\end{proof}

\begin{theorem}\label{thm:sheaf_condition}
	A presheaf $F$ on $\Et/X$ is a sheaf if and only if $F$ satisfies the sheaf condition for Zariski open coverings and for \'etale coverings consisting of a single map $V \to U$, where $V$ and $U$ are affine.
\end{theorem}

The idea of the proof is to rewrite coverings $\{ U_i \to U\}$ as a single morphism $\coprod U_i \to U$ and to use the fact that schemes are locally affine.

\begin{proof}
	Let $F$ be a presheaf satisfying the condition of the theorem. Because of the previous lemma and because of the equality
	\[
		\Bigl(\coprod U_i \Bigr) \times_U \Bigl( \coprod U_j \Bigr) = \coprod_{i,j} U_i \times_U U_j,
	\] the sheaf condition for the covering $\{U_i \to U\}$ is equivalent to the sheaf condition for the covering $\coprod_i U_i \to U$. If the indexing set $I$ is affine each $U_i$ is affine, $\coprod U_i$ is again affine and the sheaf condition holds.

	Now let $\{U_i \to U\}$ be an arbitrary cover and let $f: \mathcal{U} \to U$ the corresponding morphism from the coproduct $\mathcal{U} = \coprod U_i$. Let $\{V_i \to U\}$ be a covering of $U$ by affine open sets. Then $f^{-1}(V_i)$ is a union of open affines, say
	\[
		f^{-1}(V_i) = \bigcup_j V_{ij}
	\]
	Each $f(V_{ij})$ is open in $V_i$ and $U_i$ is quasi-compact. Therefore there is a finite set $K_i$ of indices such that $\{V_{ik} \to V_i\}_{k \in K_i}$ is a covering. We obtain a finite affine cover $\{V_{ik} \to U_i\}_{k \in K_i}$ for each $V_i$. Consider the diagram:

	\[
		% https://q.uiver.app/?q=WzAsOCxbMCwwLCJGKFUpIl0sWzEsMCwiRihVJykiXSxbMiwwLCJGKFUnIFxcdGltZXNfVSBVJykiXSxbMSwxLCJcXHByb2RfaSBcXHByb2RfaiBGKFZfe2lqfSkiXSxbMCwxLCJcXHByb2QgRihVX2kpIl0sWzAsMiwiXFxwcm9kX3tpLGp9IEYoVV9pICBcXGNhcCBVX2opIl0sWzEsMiwiXFxwcm9kX3tpLGp9XFxwcm9kX3trLGx9IEYoVl97aWt9IFxcY2FwIFZfe2psfSkiXSxbMiwxLCJcXHByb2RfaSBcXHByb2Rfe2osbH0gRihWX3tpan0gXFx0aW1lc19VIFZfe2lsfSkiXSxbMCwxXSxbMSwyLCIiLDAseyJvZmZzZXQiOjF9XSxbMSwyLCIiLDAseyJvZmZzZXQiOi0xfV0sWzEsM10sWzAsNF0sWzQsM10sWzQsNSwiIiwyLHsib2Zmc2V0IjoxfV0sWzQsNSwiIiwyLHsib2Zmc2V0IjotMX1dLFs1LDZdLFszLDYsIiIsMix7Im9mZnNldCI6LTF9XSxbMyw2LCIiLDIseyJvZmZzZXQiOjF9XSxbMyw3LCIiLDIseyJvZmZzZXQiOjF9XSxbMyw3LCIiLDIseyJvZmZzZXQiOi0xfV0sWzIsN11d
		\begin{tikzcd}
			{F(U)} & {F(\mathcal{U})} & {F(\mathcal{U} \times_U \mathcal{U})} \\
			{\displaystyle \prod F(V_i)} & {\displaystyle \prod_i \displaystyle \prod_k F(V_{ik})} & {\displaystyle \prod_i \displaystyle \prod_{k,l} F(V_{ik} \times_U V_{il})} \\
			{\displaystyle \prod_{i,j} F(V_i  \cap V_j)} & {\displaystyle \prod_{i,j} \displaystyle \prod_{k,l} F(V_{ik} \cap V_{jl})}
			\arrow[from=1-1, to=1-2]
			\arrow[shift right=1, from=1-2, to=1-3]
			\arrow[shift left=1, from=1-2, to=1-3]
			\arrow[from=1-2, to=2-2]
			\arrow[from=1-1, to=2-1]
			\arrow[from=2-1, to=2-2]
			\arrow[shift right=1, from=2-1, to=3-1]
			\arrow[shift left=1, from=2-1, to=3-1]
			\arrow[from=3-1, to=3-2]
			\arrow[shift left=1, from=2-2, to=3-2]
			\arrow[shift right=1, from=2-2, to=3-2]
			\arrow[shift right=1, from=2-2, to=2-3]
			\arrow[shift left=1, from=2-2, to=2-3]
			\arrow[from=1-3, to=2-3]
		\end{tikzcd}
	\]

	We need to verify that the top row is exact. The columns are because the $V_i$ are open subsets of $X$ and $F$ is a sheaf for the Zariski topology by assumption. The middle row is a product of exact sequences and hence exact. It follows that the map $F(U) \to F(\mathcal{U})$ is injective and that $F$ is a separated presheaf. Thus the bottom arrow is injective. It follows by a diagram chase that the top row is exact, so $F$ is a sheaf.
\end{proof}

\begin{theorem}
	Every presheaf $\widehat{Z}$ represented by an $X$-scheme $Z$ given by $U \to \Hom_X(U,Z)$ is a sheaf. We will often omit the notation and simply write $Z(U)$ instead of $\widehat{Z}(U)$.
\end{theorem}

\begin{proof}
	It is obviously a sheaf for the Zariski topology. By the previous theorem, it is sufficient to show exactness of the sequence
	\[
		\widehat{Z}(\Spec(A)) \to \widehat{Z}(\Spec(B)) \xbigtoto{} \widehat{Z}(\Spec(B \otimes_A B)).
	\]
	But if $\Spec(A) \to \Spec(B)$ is surjective and \'etale, then $B \to A$ is faithfully flat and of finite type. This implies that $B \to A$ is a strict epimorphism, which means that
	\[
		\Hom(A,Z) \to \Hom(B,Z) \xbigtoto[p_2^*]{p_1^*}\Hom(B \otimes_A B, Z)
	\]
	is exact for all $Z$ (for a proof of this fact, see \cite{milneLEC}, Theorem 2.17.) and the statement follows.
\end{proof}


\section{Sheaves with algebraic structure}
Let $F$ be a sheaf of groups on $C$. For each object $X$ of $C$ there are morphisms
\[
	m \colon F(X) \times F(X) \to F(X),\ i \colon F(X) \to F(X),\ e \colon 1 \to F(X)
\]
induced by the group operation on $F(X)$. These morphisms assemble into morphisms of sheaves
\[
	m \colon F \times F \to F,\ i \colon F \to F,\ e \colon 1 \to F
\]
This is the description ``internal'' to the category $\sh(C,J)$. The external description demands that $F$ carries the structure of a group locally, meaning that $F(U)$ is a group for each $U$. From the internal point of view, a homomorphism $\varphi\colon F \to G$ of sheaves of groups is just a homomorphism of \textit{group objects} in $\sh(C,J)$.

\begin{definition}
	Let $\mathcal{C}$ be a category with finite products and a terminal object $1$. A \textit{group object} in $\mathcal{C}$ consists of an object $A$ of $\mathcal{C}$ together with morphisms
	\[e\colon * \to A,\ m \colon A \times A \to A,\ \iota\colon A \to A \]
	such that the following diagrams commute:
	\[
		% https://q.uiver.app/?q=WzAsNCxbMCwwLCJBIl0sWzAsMSwiQSBcXHRpbWVzIEEiXSxbMSwwLCJBIFxcdGltZXMgQSJdLFsxLDEsIkEiXSxbMCwxLCJcXGlvdGEgXFx0aW1lcyBpZCIsMl0sWzAsMiwiaWQgXFx0aW1lcyBcXGlvdGEiXSxbMiwzLCJtIl0sWzEsMywibSIsMl0sWzAsMywiXFxpZCIsMV1d
		\begin{tikzcd}
			A & {A \times A} \\
			{A \times A} & A
			\arrow["{\iota \times id}"', from=1-1, to=2-1]
			\arrow["{id \times \iota}", from=1-1, to=1-2]
			\arrow["m", from=1-2, to=2-2]
			\arrow["m"', from=2-1, to=2-2]
			\arrow["id", from=1-1, to=2-2]
		\end{tikzcd}
	\]
	\[
		% https://q.uiver.app/?q=WzAsNCxbMCwwLCJBIFxcdGltZXMgQSBcXHRpbWVzIEEiXSxbMSwwLCJBIl0sWzEsMSwiQSJdLFswLDEsIkEiXSxbMCwxLCJpZCBcXHRpbWVzIG0iLDJdLFsxLDIsIm0iLDJdLFswLDMsIm0gXFx0aW1lcyBpZCJdLFszLDIsIm0iXV0=
		\begin{tikzcd}
			{A \times A \times A} & A \\
			A & A
			\arrow["{id \times m}"', from=1-1, to=1-2]
			\arrow["m"', from=1-2, to=2-2]
			\arrow["{m \times id}", from=1-1, to=2-1]
			\arrow["m", from=2-1, to=2-2]
		\end{tikzcd}
	\]
	% https://q.uiver.app/?q=WzAsNCxbMCwwLCJBIl0sWzEsMCwiQSBcXHRpbWVzIEEiXSxbMSwxLCJBIl0sWzAsMSwiQSBcXHRpbWVzIEEiXSxbMCwxLCIoZSxpZCkiXSxbMSwyLCJtIl0sWzAsMywiKGlkLGUpIiwyXSxbMywyLCJtIiwyXV0=
	\[
		\begin{tikzcd}
			A & {A \times A} \\
			{A \times A} & A
			\arrow["{(e,id)}", from=1-1, to=1-2]
			\arrow["m", from=1-2, to=2-2]
			\arrow["{(id,e)}"', from=1-1, to=2-1]
			\arrow["m"', from=2-1, to=2-2]
		\end{tikzcd}
	\]
\end{definition}

There are obvious analogues for other notions of algebraic structure. Particularly important examples for algebraic geometry are given by $\mathcal{O}_X$-modules. Omitting some details, a sheaf of groups $M$ on $X$ is an $\mathcal{O}_X$-module if each $M(U)$ carries the structure of an $\mathcal{O}_X(U)$-module that is compatible with restriction, so the following diagram commutes:

\[
	% https://q.uiver.app/?q=WzAsNCxbMCwwLCJcXG1hdGhjYWx7T31fWChVKSBcXHRpbWVzIE0oVSkiXSxbMCwxLCJcXG1hdGhjYWx7T31fWChWKSBcXHRpbWVzIE0oVikiXSxbMSwwLCJNKFUpIl0sWzEsMSwiTShWKSJdLFswLDFdLFswLDJdLFsyLDNdLFsxLDNdXQ==
	\begin{tikzcd}
		{\mathcal{O}_X(U) \times M(U)} & {M(U)} \\
		{\mathcal{O}_X(V) \times M(V)} & {M(V)}.
		\arrow[from=1-1, to=2-1]
		\arrow[from=1-1, to=1-2]
		\arrow[from=1-2, to=2-2]
		\arrow[from=2-1, to=2-2]
	\end{tikzcd}
\]


\begin{definition}
	The \textit{category of abelian sheaves on $\Et/X$} is defined to be the category of abelian group objects in $\sh_{\Et}(X)$
\end{definition}


\begin{corollary}
	A group object internal to $\Et/X$ is a scheme $S$ over $X$ with maps $\circ \colon S \times_X S \to S$, $e \colon X \to S$ and $\iota \colon S \to S$ subject to the conditions of the previous definition.
\end{corollary}

\subsection{Abelian Sheaves and Group Schemes}
A natural question arises: What is the relationship between the group schemes over $X$ and sheaves of groups on $\Et/X$? To answer this question we recall the basics of representing objects and the yoneda lemma.

For a locally small category $C$, we write $\widehat{c}$ for the functor from $C$ to $\Set$ given by $A \to \Hom(A,c)$. We say that $\widehat{c}$ is the presheaf on $C$ represented by $c$. There is a functor from $C$ to the category of presheaves on $C$ called the Yoneda embedding\footnote{Named after japanese mathematician Nobuo Yoneda, the symbol $\yo$ is a japanese ``yo''}:
\begin{align*}
	\yo \colon C & \to [C^{\op}, \Set] \\
	c            & \to \widehat{c}
\end{align*}
We will also write $\widehat{C}$ instead of $[C^{\op}, \Set]$ for the functor category. Later in this section we will prove that every representable presheaf is a sheaf in the \'etale topology.
In algebraic geometry the approach to studying schemes via the Yoneda lemma bears a special name. It is the ``functor of points approach''. Let $\widehat{Z}$ be the set-valued functor represented by the scheme $Z$. We call the functor $\widehat{Z}\colon \text{Sch} \to \Set,\ X \to \Hom(X,Z)$ the functor of points of $Z$. Set $\widehat{Z}(Y)$ the $Y$-valued points of $Z$. If the scheme $Z = \Spec(A)$ is affine, we also call them the $A$-valued points of $Z$.

\begin{corollary}
	The yoneda embedding $\yo\colon \Et/X \to \widehat{\Et/X}$ factors through the inclusion $\sh(\Et/X) \to \Set$.
\end{corollary}

Recall the statement of the Yoneda lemma:
\begin{theorem}[The Yoneda lemma]
	Let $C$ b a locally small category and $[C^{\op}, \Set]$ its presheaf category. For every presheaf $F\colon C^{\op} \to \Set$ there is a canonical isomorphism
	\[
		\Hom_{\widehat{C}}(\widehat{c}, F) \simeq F(c)
	\] between natural transformations from $\widehat{c}$ to $F$ and the elements of $F(c)$
\end{theorem}

\begin{definition}
	A functor $F\colon C^{\op} \to \Set$ is representable if there is a natural isomorphism $F \cong \widehat{c}$ for some object $c$ of $C$.
\end{definition}

\begin{example}
	The functor $(-)^\times\colon \Rng \to \Ab$ that sends a ring $R$ to its set of units $R^\times$ is represented by the ring $\Z[x,x^{-1}]$ of Laurent polynomials. This means that a ring homomorphism $\Z[x,x^{-1}] \to R$ may be defined by sending $x$ to any unit of $R$, and this morphism is completeley determined by this choice, so we have
	\[
		R^\times \cong \Hom(\Z[x,x^{-1}], R).
	\]
	The scheme $\Gm \coloneqq \Spec(\Z[x, x^{-1}])$ is a group scheme over $\Z$. By contravarince, the multiplication $\Gm \times \Gm \to \Gm$ is given by the algebra map
	\[
		\begin{array}{c @{{}\to{}} c @{{}{}} c}
			\Z[x,x^{-1}] & \centermathcell{\Z[x,x^{-1}]\otimes_\Z \Z[x,x^{-1}]} \\
			x            & \centermathcell{\, x\otimes x}.
		\end{array}
	\]
	Note that $\Spec(\Z[x,x^{-1}])$ is \'etale over $\Spec(\Z)$ because $\Z[x,x^{-1}] = \Z[x][y]/(xy-1)$ and the Jacobian determinant of $(xy-1)$ is $x$, which is invertible in $\Z[x,x^{-1}]$. In order to consider $\Gm$ as a sheaf on the \'etale site of an arbitrary scheme $X$, we consider the base change $\mathbf{G}_{m,S}$ of $\Gm$ along $X \to \Spec(\Z)$:
	\[
		%https://q.uiver.app/?q=WzAsNCxbMCwwLCJcXG1hdGhiZntHfV97bSxTfSJdLFsxLDAsIlxcR20iXSxbMSwxLCJcXFNwZWMoXFxaKSJdLFswLDEsIlMiXSxbMCwxXSxbMSwyXSxbMywyXSxbMCwzXV0=
		\begin{tikzcd}
			{\mathbf{G}_{m,S}} & \Gm \\
			S & {\Spec(\Z)}.
			\arrow[from=1-1, to=1-2]
			\arrow[from=1-2, to=2-2]
			\arrow[from=2-1, to=2-2]
			\arrow[from=1-1, to=2-1]
		\end{tikzcd}.
	\]
	By the universal property of pullbacks and the fact that $\Spec(\Z)$ is the final object in schemes we have
	\[
		\mathbf{G}_{m,X} = \Hom_X(U, \mathbf{G}_{m,X}) = \Hom_{\Spec(\Z)}(U, \Spec(\Z[x,x^{-1}])) = \Gm(U),
	\]
	By abuse of notation we will write $\Gm$ for $\mathbf{G}_{m,X}$. It follows that evaluating the sheaf represented by $\Gm$ on a scheme $X$ yields
	\[
		\Gm(X) = \Hom(X, \Spec(\Z[x,x^{-1}])) \cong \Hom(\Z[x,x^{-1}], \Gamma(Y, \Sh{O}_Y)) \cong \Sh{O}_Y^\times.
	\]
	so $\Gm$ defines a sheaf of groups on $\Et/X$.

	%Let $S = \Spec(A)$ be an affine scheme. The base change of $\Gm$ along $S \to \Z$ is $\Spec(A \otimes \Z[x,x^{-1}]) = \Spec(A[x,x^{-1}])$
	%If $X = \Spec(A)$ is affine, then $\mathbf{G}_{m,X} = \Spec(A[x,x^{-1}])$.
\end{example}

\begin{example}
	Similarly consider the ring $\Z[x]$. Defining a ring homomorphism from $\Z[x]$ to $R$ is the same as choosing an image for $x$ in $R$, so $\Z[x]$ is the representing object for the functor which sends a ring $R$ to its underlying additive group.
	The scheme $\Ga = \Spec(\Z[x])$ is an \'etale group scheme over $\Z$, with multiplication $\Ga \times \Ga \to \Ga$ induced by the maps
	\[
		\begin{array}{c @{{}\to{}} c @{{}{}} c}
			\Z[x] & \centermathcell{\Z[x]\otimes_\Z \Z[x]}     \\
			x     & \centermathcell{x\otimes 1 + 1 \otimes x}.
		\end{array}
	\]
	The sheaf $\mathbf{G}_{a,X}$ sends each \'etale scheme $U$ over $X$ to the underlying additve group of $\mathcal{O}_X(U)$ as the following calculation shows:
	\[
		\Ga(u) = \Hom(X, \Spec(\Z[x])) \cong \Hom(\Z[x], \Gamma(Y, \Sh{O}_Y)) \cong \Sh{O}_Y
	\]
\end{example}

\begin{example}
	The functor sending a ring $R$ to the set of its roots of unity
	\[
		\mu_n(R) = \{f \in R \mid f^n = 1\}
	\]
	is represented by the ring $\Z[x]/(x^n-1)$, so we have $\Hom(\Z[x]/(x^n-1),R) \cong \mu_n(R)$. Using similar arguments as before we obtain the sheaf of roots of unity $\mu_n$ on the \'etale site $\Et/X$ by base changing $\mu_n = \Spec(\Z[x]/(x^n-1))$ along $X \to \Spec(\Z)$. The sheaf is the kernel of$\mu_n$ is a subsheaf of $\Gm$. Our first important computation in cohomology will be the cohomology of the sheaf $\mu_n$.

\end{example}

\begin{example}
	Let $R=\Z[x_z, \dots, x_n]/(f_1, \dots , f_m)$ and $X = \Spec(R)$. What are the $\Z$-valued points of $X$? Specifying a morphism from $X$ to $\Spec(\Z)$ is equivalent to specifying a morphism $\varphi\colon R \to \Z$, which is equivalent to specifying images $a_i$ of the generators $x_i$ of $R$ such that $f_j(a_1, \dots, a_n) = 0$ for all $j$. In other words, the $\Z$-valued points are the integral solutions to the equation $f_j = 0$
\end{example}

\section{Local Rings and Stalks}
%ROUGH
Local rings are rings with a unique maximal ideal. A ring is local if and only if a sum of any two non-units is again a non-unit. To motivate the name, consider the following example:

Let $C[-1,1]$ be the ring of continuous real valued functions on the interval $[-1,1]$ and consider the equivalence relation $\sim$ which identifies two function $f$ and $g$ if there is an open neighborhood $U$ of $0 \in [-1,1]$ such that $f_U = g_U$. The equivalence classes defined by this relation are called the germs of $C[-1,1]$ at $0$ and in fact form a ring, since we may add and multiply two germs. This ring is called the \textit{stalk} of $C[-1,1]$ at $0$. It is clear that a germ is invertible if and only if $f(0) \neq 0$ and it follows that the sum of two non-units is again a non-unit, so $C[-1,1]$ is local.

More generally we can define the stalk for presheaves on an arbitrary topological space: If $F$ is a presheaf on a space $X$ and $x \in X$ is a point, we define the stalk $F_x$ of $F$ at $x$ as the colimit $\colim_{U \ni x}F(U)$ over all neighborhoods of $x$. If $X = \Spec(A)$ is an affine scheme,
%ROUGH%

We can also form stalks of sheaves on the \'etale toplogy. Let $X$ be a scheme and $x$ a point of $X$. An \'etale neighborhood of $x$ is defined to be a pair $(U, y)$ together with an \'etale morphism $\varphi\colon U \to X$ such that $\varphi(y) = x$. A morphism of \'etale neighborhoods $f\colon (V,z) \to (U,y)$ is a morphism of $X$-schemes $f\colon V \to U$ such that $f(v) = u$. The stalks of sheaves are defined as a colimit over all \'etale neighborhoods.


We will show that for each $x \in X$, the functor $F \to F_x$ is exact. This provides us with a way to check exactness locally: An exact sequence of abelian sheaves
\[
	0 \to F \to G \to H \to 0
\]
is exact if and only if the seqence
\[
	0 \to F_x \to G_x \to H_x \to 0
\]
is exact for each $x \in X$.
Recall that a category $\mathcal{C}$ is \textit{cofiltered} if the following holds:
\begin{enumerate}
	\item For any pair of objects $a_1$ and $a_2$, there is an object $b$ with maps $b \to a_1$ and $b \to a_2$.
	\item Every pair of morphisms $f,g\colon a \to b$, there is an equalizer $h\colon c \to a$. This means that $f \circ h = g \circ h$.
\end{enumerate}

\begin{example}
	The subcategory of $\Op(X)$ consisting of all open sets containing $x$ is cofiltered because $\Op(X)$ has all finite pullbacks. The second condition does not need to be verified, since there is at most one morphism between any two open sets.
\end{example}


\begin{example}
	Let $X$ be a scheme and consider the structure sheaf $\mathcal{O}_X$ in the Zariski topology. The stalks of this sheaf, written $\mathcal{O}_{X,x}$ for $x \in X$ are local rings: By definition each point $x \in X$ is contained in an affine neighorhood $\Spec(A) \subset X$. We define a map from the stalk $\Sh{O}_p$ to the ring $A_p$, where $p$ is the localisation of $A$ at the the multiplicative subset $A\setminus (p)$ by sending a section of .
	In the \'Etale topology, the limits $\varprojlim_{U \ni x} \mathcal{O}(U)$ turn out to be the strict henselizations of the local rings $\mathcal{O}_{X,x}$.
\end{example}

%\section{Some categories and functors related to sheaves}
%
%Rough draft \par
%Sheaves are closely related to bundles. Because bundles are not locally trivial, we define vector bundles on schemes via sheaves. We prove that there is the sheaf-bundle adjunction and hopefully apply it.
%
%\subsection{Bundles and sheaves}
%
%A map $f: Y \to X$ of topological spaces is also called a bundle over $X$. Every bundle gives rise to the sheaf $Gamma_f$, defined by
%\[
%	\Gamma_f(U)	= \{ s : U \to f^{-1}(U) \mid f \circ s = id \}.
%\]
%This sheaf is called the \textit{sheaf of sections of $f$}. Each map $f \to g$ of bundles over $X$ induces a map of sheaves.
%\begin{remark}
%	It is not the case that the \'espace \'etale of an \'etale sheaf is again a scheme. I have heard this is a reason to consider algebraic spaces
%\end{remark}
%
%\begin{proposition}
%	Let $f: X \to Y$ be a morphism of schemes. The presheaf $\Gamma_f$ is a sheaf in the \'etale topology.
%\end{proposition}
%\begin{proof}
%
%\end{proof}
%
%\begin{definition}[Fiber bundles]
%	Let $X$ be a topological space. A \textit{fiber bundle over $X$ with fiber $F$} is a space $p: B \to X$ over $X$ such that for each point $x \in X$ there is an open neighborhood $U$ of $x$ such that $p^{-1}(U) \cong U \times F$. A cover $\{U_i\}$ is called \textit{a trivialization of $B$} if $p^{-1}(U_i) \cong U_i \times F$ for each $i$. We will write $U_{ij}$ instead of $U_i \cap U_j$ and similarly for $U_i \cap U_j \cap U_k$.
%\end{definition}
%\begin{remark}
%	Suppose we are given a cover $\{U_i\}$ of $X$. Under which conditions can we construct a fiber bundle $p: B \to X$ with fiber $F$ such that $\{U_i\}$ is a trivialization of $B$? We know that we want $B_i \coloneqq p^{-1}(U_i)\cong U_i \times F $, so $B$ should be glued together from the $B_i$. In order to do this we need to specify isomorphisms $\phi_{ij} : U_i \to U_j$ subject to the \textit{cocycle condition}:
%	\[(...)\]
%	.
%\end{remark}
%
%\begin{construction}
%	Each fiber bundle $B$ over $X$ with projection $p$ gives rise to a sheaf. For each open set $U$ of $X$ we can consider the set of sections
%	\[
%		\{s \in \mathcal{M}(U) \mid p \circ s = \text{id}\}.
%	\]
%	This forms a sheaf on $X$. We have seen (...) that fibre bundles are not locally trivial in the Zariski topology. For this reason the analogs of fibre bundles in algebraic geometry are defined using sheaves with a local triviality condition.
%\end{construction}
%\begin{example}
%	Let $p : M \to S^1$ be the M\"obius strip with fiber $[-1,1]$ from the introduction. Let $\mathcal{M}$ be the sheaf of sections of this bundle. There is a subsheaf $\mathcal{N}$ which consists of those sections which vanish nowhere:
%	\[
%		\mathcal{N}(U) = \{s : U \to U \times [-1,1] \mid \pi_1(s) \neq 0 \text{ for all } s \in U\}.
%	\]
%	This subsheaf has no global sections. This is because of the non-orientability of the M\"obius strip. We may choose a trivialisation $\{U_1, U_2, U_3\}$ of $M$ such that each $U_i$ is connected. Then the sign of sections $s_i: U_i \to U_i \times [-1,1]$ must be constant.
%\end{example}


\section{Sheafification}
In this section we will characterise Grothendieck toposes as left exact reflective subcategories $\mathcal{E} \to \widehat{C}$ of presheaf categories. More precisely, given a small category $C$ there is a bijection between
\begin{itemize}
	\item Equivalence classes of left exact reflective subcategories $\mathcal{E} \to \widehat{C}$
	\item Grothendieck topologies $J$ on $C$ such that $\mathcal{E} \simeq \sh(C,J)$.
\end{itemize}

We first show that any Grothendieck topos $\sh(C,J)$ on $C$ is indeed a left exact reflective subcategory of $\widehat{C}$. Recall that a full subcategory $i \colon D \to C$ reflective if the inclusion functor $i$ admits a left adjoint $a \colon C \to D$. We already know that $\sh(C,J)$ is a full subcategory of $\widehat{C}$.

\begin{proposition}\label{thm:associated_sheaf}
	The forgetful functor $\mathsf{Sh}(C,J) \to \Set^{C^{op}}$ admits a limit-preserving left adjoint
	\[a : \Set^{C^{op}} \to \mathsf{Sh}(C,J).\]
\end{proposition}

\begin{corollary}
	The cateogry of sheaves on $X$ is a full reflective subcategory of the category of presheaves on $X$.
\end{corollary}

The functor $a$ is the \textit{associated sheaf functor}, also called \textit{sheafification}. As the name suggests, it provides universal way to turn a presheaf into a sheaf.
Note that there are two ways a presheaf can fail to be a sheaf:
\begin{itemize}
	\item
	      Local sections may fail to patch to a global section. An example of a presheaf whose local sections fail to patch is the presheaf of bounded continuous functions on $\R$. If we cover $\R$ by bounded intervals $\{U_i\}$, the identity function is bounded on each $U_i$ but is obviously not globally bounded.
	\item
	      Sections that agree locally may not agree globally. We construct an example. $S$ be the a discrete topological space with two points $0$ and $1$. We define a presheaf $\Sh{F}$ by setting $\Sh{F}(\varnothing) = \{*\}$ and $\Sh{F}(U) = \R^U$. We define the restriction maps to send a section $s \in \Sh{F}(U)$ to the constant function $s|_V \equiv 0$ when $V \subseteq U$ is nonempty. Now let $s,t \in \Sh{F}(S)$ be defined by $s \equiv 1$ and $t \equiv -1$. Then $\{\{0\}, \{1\}\}$ is an open cover of $S$ and $s|_{0} = t|_{0}$ and $s|_{1} = t|_{1}$, but $s \neq t$.
	      This cannot happen if the map
	      \[
		      i: \Sh{F}(U) \to \mathsf{eq}(\prod_{i} \Sh{F}(U_i) \xbigtoto{} \prod_{i,j} \Sh{F}(U_i \cap U_j))
	      \]
	      is injective. A presheaf for which $i$ is injective is called separated.
\end{itemize}

Thus, sheafification proceeds in two steps. The first step removes sections that agree locally but not globally, making a presheaf into a separated presheaf. The second step adds the sections of matching local sections.

\begin{proof}[Proof of Theorem~\ref{thm:associated_sheaf}]
	See~\cite{SIGL}.
\end{proof}

\begin{theorem}
	A category is a Grothendieck topos if and only if it is a localization of $\widehat{C}$ for some small category $C$.
\end{theorem}
\begin{proof}
	See~\cite{Borceux3}, Proposition 3.5.4 and Corollary 3.5.5.
\end{proof}

\section{The global sections functor}
Let $X$ be a topological space and $F$ a sheaf on $X$. The elements of $F(X)$ are called \textit{global sections of $F$}. It is convenient to view $F$ as a variable and to write $\Gamma(X,-)$ as the functor which sends a sheaf $F$ to the set of its global section $\Gamma(X, F)$. Note that a sheaf may not have global sections. For example, the sheaf of sections of the double cover of a circle has no global sections, but locally  there are 3. Sheaf cohomology is all about applying the techniques from homological algebra to study the global sections of sheaves. Note that $X$ is the terminal object in both $\Op(X)$ and in $\Et/X$, so a global section for an \'etale sheaf is defined in the same way.

Consider the terminal object $1$ in the category $\Set^{C^{op}}$. It is the presheaf defined by $1(X) = \{*\}$ for all $X \in C$. A morphism of presheaves $\gamma: 1 \to P$ picks an element $\gamma_U$ for each object $U$ of $C$ such that the following diagram commutes for each $f: U \to U'$:
\[
	% https://q.uiver.app/?q=WzAsMyxbMSwwLCJcXHsqXFx9Il0sWzAsMSwiXFxTaHtGfShVJykiXSxbMiwxLCJcXFNoe0Z9KFUpIl0sWzAsMV0sWzEsMiwiZl8qIl0sWzAsMl1d
	\begin{tikzcd}
		& {\{*\}} \\
		{\Sh{F}(U')} && {\Sh{F}(U)}
		\arrow[from=1-2, to=2-1]
		\arrow["{f_*}", from=2-1, to=2-3]
		\arrow[from=1-2, to=2-3]
	\end{tikzcd}
\]
We obtain a functor $\Gamma: \Set^{C^{op}} \to \Set, \Gamma(P) = \Hom(1, P)$. Conversely, we can assign to each set $S$ the constant presheaf $\Delta C$ by setting $\Delta C(U) = S$ and letting all restrictions be identities. There are natural isomorphisms
\[
	\Hom_{\widehat{C}}(\Delta C, P) \cong \Hom(S, \Gamma P),
\]
so the global sections functor is left adjoint to the constant presheaf functor.
Since the inclusion of sheaves into presheaves has a left adjoint it preserves limits. In particular, terminal object $1$ is also a sheaf. Since adjoint functors compose, it follows that the functor $\Gamma : \mathsf{Sh}(C,J) \to \Set$ is right adjoint to the composition $a \circ \Delta: \Set \to \mathsf{Sh}(C,J)$.


For example, when $X$ is a topological space, $\Hom(1,F)(X)$ consists of all global sections of the sheaf $\Sh{F}$. Here $\Hom(1,F)$ is in the category $\mathsf{Sh}(C,J)$

\begin{theorem}
	There is an adjoint pair of functors
	% https://q.uiver.app/?q=WzAsMixbMCwwLCJcXG1hdGhzZntTaH0oQyxKKSJdLFsxLDAsIlxcU2V0Il0sWzEsMCwiXFxEZWx0YSIsMCx7Im9mZnNldCI6LTF9XSxbMCwxLCJcXEdhbW1hIiwwLHsib2Zmc2V0IjotMX1dXQ==
	\[
		\begin{tikzcd}
			{\mathsf{Sh}(C,J)} & \Set
			\arrow["\Delta", shift left=1, from=1-2, to=1-1]
			\arrow["\Gamma", shift left=1, from=1-1, to=1-2]
		\end{tikzcd}
	\]
	The global sections functor $\Gamma: \mathsf{Sh}(C,J) \to \Set,\ \Gamma(F) = \Hom(1,F)$ is right adjoint to the constant sheaf functor $\Delta: \Set \to \Sh(C,J)$.
\end{theorem}

\subsection{Sheaf semantics}
We will now construct a semantics of intuitionistic propositional logic using a site $(C,J)$ and its topos $\sh(C,J)$. The basic idea is that we can use the structure of a site to speak of local truth. These semantics will provide us with a way to construct subobjects $\{x \mid \varphi(x) \} \rightarrowtail
	F$ by pulling back along the map $\text{true}: \mathbf{1} \rightarrowtail \Omega$:
% https://q.uiver.app/?q=WzAsNSxbMSwwLCJcXHt4ICBcXG1pZCBcXHZhcnBoaSh4KVxcfSJdLFsyLDAsIjEiXSxbMSwxLCJYIl0sWzIsMSwiXFxPbWVnYSJdLFswLDEsIlUiXSxbMCwxXSxbMCwyXSxbMiwzLCJcXHZhcnBoaSh4KSIsMl0sWzEsMywidHJ1ZSIsMCx7InN0eWxlIjp7InRhaWwiOnsibmFtZSI6Im1vbm8ifX19XSxbNCwwLCIiLDAseyJzdHlsZSI6eyJib2R5Ijp7Im5hbWUiOiJkYXNoZWQifX19XSxbNCwyLCJcXGFscGhhIl1d
\[\begin{tikzcd}
		& {\{x  \mid \varphi(x)\}} & 1 \\
		U & F & \Omega
		\arrow[from=1-2, to=1-3]
		\arrow[from=1-2, to=2-2]
		\arrow["{\varphi(x)}"', from=2-2, to=2-3]
		\arrow["\text{true}", tail, from=1-3, to=2-3]
		\arrow[dashed, from=2-1, to=1-2]
		\arrow["\alpha", from=2-1, to=2-2]
	\end{tikzcd}\]
The map $\{ x \mid \varphi(x) \} \to F$ has the property such that any other \'etale open $U \to X$ with $U \vDash \varphi$ factors through $\{x \mid \varphi(x) \}$ where $x$ is a variable of type $F$ and denote this by $x \colon F$. We will now formulate some conditions from scheme theory in this internal language. A good reference is \cite{BlechSchmidt}.
%of a sheaf $F$ such that for any \'etale open $U \to X$ with $U \vDash \varphi$

\begin{center}
	\def\arraystretch{1.5}%  1 is the default, change whatever you need

	\begin{tabular}{|c|p{8cm}|}
		\hline
		Logic                               & Geometry                                                                                                             \\
		\hline
		$U \vDash \top$                     & $U = U$.                                                                                                             \\
		$U \vDash \bot$                     & $U = \varnothing$.                                                                                                   \\
		$U \vDash s \colon F$               & $s(U) \in F(U)$.                                                                                                     \\
		$U \vDash s = t \colon F$           & $s(u) = t(u) \in F(U)$.                                                                                              \\
		$U \vDash \varphi \land \psi$       & $U \vDash \varphi$ and $U \vDash \psi$.                                                                              \\
		$U \vDash \bigwedge_{j} \varphi_j$  & $ U \vDash \varphi_j $ for all $j$.                                                                                  \\
		$U \vDash \varphi \vee \psi$        & There is some covering $\coprod_i U_i \to U$ such that for all $i$ we have $U_i \vDash \varphi$ or $U_i \vDash \psi$ \\
		$U \vDash \bigvee_{j} \varphi_j$    & There is some covering $\coprod_i U_i \to U$ such that for all $i$ we have $U_i \vDash \varphi_j$ for some $j \in J$ \\
		$U \vDash \varphi \Rightarrow \psi$ & For all  $V \to U$ we have $V \vDash \varphi$ implies $V \vDash \psi$                                                \\
		\hline
	\end{tabular}
\end{center}

In our context we view $U$ as an \'etale open of some scheme $X$ and $F$ is some sheaf on $\Et/X$. Notice that as in topology, we allow arbitrary unions but only finite intersections.
\begin{proposition}
	Let $\coprod U_i \to U$ be a cover of $U$ and $\varphi$ a formula over $U$
\end{proposition}



\begin{construction}
	Let $F$ be a presheaf on $\Et/\Spec(k)$. By abuse of notation we write $F(K)$ instad of $F(\Spec(K))$. We define a descrete $G$-module $M_F$ as follows: If $L/k$ is a finite separable extension, then $G = \Gal(k)$ acts on $F(L)$ by functoriality of $F$. Define $M_F = \colim F(L)$ where $L$ runs over all subfields $L$ of $k_s$ that are finite over $k$. Then $M_F$ is a discrete $G$-module. Conversely, for a discrete $G$-module $M$ we define a sheaf $F_M$ by setting $F_M(A) = \Hom_G(F(A),M)$ where $F(A) = \Hom_k(A,k_s)$. By Theorem \ref{thm:sheaf_condition}, presheaf $F$ on $\Et/k$ is a sheaf if and only if \ref{diagram:equalizer} is exact for a single affine cover $\Spec(A) \to \Spec(B)$, but affine \'etale covers in $\Et/k$ are precisely of the form $\displaystyle \coprod \Spec(L_i) \to \displaystyle \coprod \Spec(K_j)$ where each $L_i$ and $K_j$ is separable over $k$, so the sequence \[ F(\Spec(L)) \to \prod_{i} F(\Spec(L_i)) \xbigtoto{} \prod_{i,j} F(\Spec(L_i \otimes_L L_j)) \] needs to be exact.
\end{construction}

\begin{enumerate}
	\item $F(\prod A_i) = \bigoplus F(A_i)$ for every finite family $\{A_i\}$ of \'etale algebra
	\item $F(L) \xrightarrow{\sim} F(K)^{\Gal(K/k)}$ for finite Galois extensions $K/L/k$
\end{enumerate}

For $F$ a sheaf on $\Et/k$ define $M_F = \varinjlim F(K)$, where the colimit runs over all finite Galois extensions over $k$. Then $M_F$ is a discrete $\Gal(k)$-module. Converseley, if $M$ is a discrete $G$-module, define $F_m(A) = \Hom_{\Gal(k)}(F(A),M)$, where $F(A) = \Hom(A, k_s)$. Then $F_M$ is a sheaf on $\Spec(k)$. This defines an equivalence of categories between the \'etale topos $\sh(\Et/\Spec(k))$ of $k$ and the category of discrete $G$-modules.
