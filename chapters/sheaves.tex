Much of modern geometry may be formulated in the language of sheaves. Sheaves are data that are given on a topological space that may be glued. One may more generally define sheaves on a category equipped with a so-called Grothendieck topology.

\begin{construction} 
  Let $X$ be a topological space with a topology $\tau \subseteq \mathcal{P}(X)$. The open sets of $\tau$ are partially ordered by inclusion: \[V \le W \quad \text{iff} \quad V \subseteq W.\]
  We may thus consider the category of open sets of $X$, denoted by $\mathcal{O}(X)$.

\end{construction}


\begin{definition}[Preheaves and Sheaves]
  Let $X$ be a topological space. A presheaf $\mathcal{F}$ (of sets) is a rule which assings
  \begin{enumerate}
    \item to each open subset $U$ of $X$ a set $\mathcal{F}(U)$.
    \item to each inclusion of open sets $U \subseteq V$ a \textbf{restriction map}
    \[res_{V\kern-0.1em , U}: \mathcal{F}(V) \longrightarrow\mathcal{F}(U).\]
  \end{enumerate}
    The restriction map needs to have the following properties: 
    \begin{enumerate}
      \item The restriction along the identity $V \subseteq V$ is the identity, so 
            \[res_{V\kern-0.1em , V} = id.\]
      \item For a sequence of inclusions $U \subseteq V \subseteq W$, the restricion maps need to adhere to the identity 
            \[res_{W\kern-0.1em , V}\ \circ\ res_{V\kern-0.1em , U} = res_{W\kern-0.1em , U}.\]
    \end{enumerate}
\end{definition}

\begin{definition}[bases for Grothendieck topologies]
  
\end{definition}

\begin{definition}[the atomic site]

\end{definition}
The atomic site is subcaonical if and only if every morphism in the category is a regular epimorphism. This means that every morphism is the coequalizer of some parallel pair of morphisms.

Taken from \cite{barr/diaconescu:1980}. \\
Fix a perfect field $k$. We denote by $k_{Ext}$ the category of separable field extensions of $k$. If $L_1$ and $L_2$ are two extensions of $k$, then $L_1 \otimes_k L_2$ is a product of a finite number of separable extensions of $k$. Thus $k_{Ext}$ is dual to an atomic site. A sheaf on this site is given by a morphism of sites $f\colon k_{Ext}^{\op} \longrightarrow \Set$. This is just a presheaf on $k_{Ext}$, and is thus a colimit of representables $f(K) = \colim(K, K_i)$. ...\\

$k_{\Ext}^{\op}$ vs. $k_{Ext}^{op}$