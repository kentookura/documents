%The notion of cohesion is an attempt to axiomatize what it means for a category $\mathcal{E}$ to be a category of spaces consisting of basic shapes of a category $\mathcal{S}$.
In what way are topological spaces different from mere sets? One may imagine that the points contained in an open set $U$ of a space $X$ cohere together, like molucules of water in a droplet. The notion of cohesion is an attempt to formalise this idea at the level of categories, and topological spaces are the principal example of cohesion. We will outline the cohesion of topological spaces over sets before giving the abstract defintion of cohesion.\\

\section{The cohesion of topological spaces}
Let $\Top$ and $\Set$ denote the category of locally connected topological spaces\footnote{The construction does not technically work for the category of \textit{all} topological spaces, but it is fine to ignore this technicality for the moment.} and sets respectively. There is the evident forgetful functor from $\Top$ to $\Set$, which we will denote by $\Gamma$. It assigns to each topological space $X$ the underlying set $\Gamma(X)$. There are also two functors $\Disc$ and $\CoDisc$, going from $\Set$ to $\Top$. The functor $\Disc$ assigns to a set $S$ the topological space $\Disc(S)$ with underlying set $S$ and the discrete topology, while $\CoDisc(S)$ is the space with the trivial topology. Lastly, there is also a functor $\Pi_0$ which takes a topological space to its set of connected components.
\begin{proposition}
  The functors described above form an adjoint quadruple \[\Pi_0 \dashv\ \Disc \dashv\ \Gamma \dashv\ \CoDisc.\]
\end{proposition}
\begin{proof}
  Let $S$ be a set and consider a set function $f: S \to \Gamma(X)$. Since any subset of $\Disc(S)$ is open, there exists a continuous function $\tilde{f}: \Disc(S) \to X$ such that the underlying set function is $f$. Conversely, any continuous function $\tilde{f} : \Disc(S) \to X$ gives rise to a set function $ f: S \to \Gamma(X)$. These assignments are clearly inverse to each other. (Naturality missing from proof). This shows that $\Disc \dashv\ \Gamma$.\\
  Next, for $g: X \to \CoDisc(S)$ a continous function, continuity just means that that $g^{-1}(S)$ is open since the only open subsets of $\CoDisc(S)$ are $\varnothing$ and $S$.
  But $g^{-1}(S)$ is just $X$. This implies that any set function from $\Gamma(X)$ to $S$ gives rise to a continuous function from $X$ to $\CoDisc(S)$. This shows that $\Gamma \dashv\ \CoDisc$.\\
  Finally, any continuous function into a discrete space must be locally constant, which means $\Hom(\Pi_0(X), S) \simeq \Hom(X, \Disc(S))$. This proves the last adjunction
\end{proof}

\section{Axiomatic Cohesion}
\section{Cohesion for gros Topoi}
\section{Cohesion for petit Topoi}
\section{WIP/Notes}

\subsection{Sufficient Cohesion over Galois topoi}
\begin{definition}[finitely presented $k$-algebra]
  Let $k$ be a field. A $k$-algebra is a \textit{finitely presented $k$-algebra} if it is a quotient of a polynomial ring $k[x_1, \dots, x_n]$
\end{definition}

\begin{definition}
  Let $k$ be a field. A $k$-algebra is \textit{connected} if its only idempotent elements are $0$ and $1$. An element $a \in k$ is idempotent if $a^2 = a$.
\end{definition}
Fix a field $k$.  We denote by $\mathsf{Ext}$ the category of separable extensions of $k$, and by $\mathsf{Con}$ the category of finitely presented and connected $k$-algebras.

\begin{lemma}
  The full inclusion $\mathsf{Ext} \longrightarrow \mathsf{Con}$ has a right adjoint.
\end{lemma}
\begin{proof}
  Let $A$ be a finitely presented and conneted $k$-algebra and choose a maximal ideal $M \subseteq A$.
\end{proof}
