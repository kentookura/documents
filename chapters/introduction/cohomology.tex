
\subsection{Interpretation of Cohomology}
Cohomology groups are not just of interest in their own right. In many cases, one can show that the cohomology classes $[\gamma] \in H^n(X, \Sh{A})$ correspond bijectively to some other kind of geometrical construction involving $X$.

For instance, when $p:Z \to B$ is a map of topological spaces and $\Sh{G}$  is a sheaf of subgroups of $Aut_B(Z)$, one can define a notion of “twist of $p$ with structure sheaf $\Sh{G}$”. One obtains the bijection
\begin{align}
            \left\{ \parbox[c]{1.1in}{\centering
                       Isomorphism classes of twists of $p$
                       with structure sheaf $\Sh{G}$}
            \right\}
            \stackrel{\sim}{\to}
            \parbox[c]{0.5in}{\centering
                       $H^1(B, \Sh{G})$}
\end{align}

There is also a bijection
\begin{align}
            \left\{ \parbox[c]{1.1in}{\centering
              Isomorphism classes of vector bundles of rank $n$ with basis $B$}
            \right\}
            \stackrel{\sim}{\to}
            \parbox[c]{0.5in}{\centering
            $H^1(B, GL_{n,B})$}
\end{align}
and in particular
\begin{align}
            \left\{ \parbox[c]{1.1in}{\centering
            Isomorphism classes of line bundles with basis $B$}
            \right\}
            \stackrel{\sim}{\to}
            \parbox[c]{0.5in}{\centering
                $H^1(B, \Ring(O)_B^\times)$}
\end{align}
