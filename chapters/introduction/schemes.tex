%A locally ringed space is a pair $(X, \mathcal{O}_X)$ where $X$ is a topological space and $\mathcal{O}_X$ is a sheaf of rings on $X$, such that all stalks $\mathcal{O}_{X,x}$ are local rings. There is a functor \[\mathsf{CommR} \to \mathsf{LRS}\]
%A scheme is in particular a locally ringed space, meaning it comes equipped with a sheaf of rings such that the stalks are local rings. This has some important consequences for the theory of schemes. If we wish to take seriously the \'etale topology as a replacement for the Zariski topology, these consequences should have analogues in the \'etale setting as well.
%
%For a sheaf of rings $\mathcal{F}$, the stalk of $\mathcal{F}$ at $x \in X$ is defined to be $\colim F(U)$, where $U$ runs over all neighborhoods of $x$. This group captures the local behavior of $\mathcal{F}$ around $x$. For this definition to make sense in the \'etale setting, we need to verify that the category of \'etale neighborhoods at $x$ is cofiltered.k
%
%\section{A brief Reminder on Schemes}
%The starting point of modern algebraic geometry is the definition of affine schemes. We briefly recall the definition and its most important consequences, after which we will construct their analogues in the \'etale setting.
%
%Let $A$ be a ring. The spectrum of $A$ is the set $\{\mathfrak{p} \subseteq A | \mathfrak{p} \text{ prime ideal}\}$. For any ideal $\mathfrak{a}$ of $A$ we write $V(\mathfrak{a})$ for the set of prime ideals containing $\mathfrak{a}$. By basic algebra one may deduce that the sets $V(\mathfrak{a})$ form the closed sets of a topology on $\Spec A$. 
%%Affine schemes are locally ringed spaces constructed from commutative rings. We briefly recall the construction. Let $A$ be a commutative ring. We put a topology on the set of prime ideals of $A$. 
%%$\{\mathfrak{p} \subseteq A | \mathfrak{p} \text{ prime }\}$
%
%
%Schemes are locally ringed spaces that are locally affine.
%Now that we have available the \'etale topology for a scheme