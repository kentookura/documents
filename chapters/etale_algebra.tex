\section{\'Etale Algebras}
We will first consider the case of \'etale algebras over a field $k$. Under the $\Spec$ functor, a finite \'etale algebra $k \to A$ gets mapped to to a disjoint union of points lying ``smoothly'' over $\Spec(k)$. This should be expected, as a space locally isomorphic to a point should just be a disjoint union of points.

\subsection{Reminder on field extensions and Galois theory}
\begin{construction}
	Let $k$ be a field and fix once and for all separable and algebraic closures $k_s \subset \overline{k}$ of $k$. Recall that an extension $L/k$ is separable if its minimal polynomial $p(x)$ has no multiple roots (in a splitting field of $p(x)$). It follows from basic algebra that a polynomial $p(x)$ is separable if and only if it is coprime to its derivative $p'(x)$. This means that $p'(x)$ gets mapped to a unit under the canonical map $k[x] \to k[x]/(p(x)) \simeq L$. An algebraic extension $L$ of $k$ is a \textit{Galois extension} if the elements of $L$ that are fixed under the automorphism group $\Aut_k(L)$ are precisely the elements of $k$. A separable extension $L/k$ is Galois if and only if the minimal polynomial of each element $a \in L$ splits into linear factors in $L$. If $L$ is Galois, we call $\Aut_k(L)$ the \textit{Galois group} of $L$ over $k$ and denote it by $\Gal(L/k)$. In particular $k_s$ is a Galois extension. We call its Galois group the \textit{absolute Galois group of $k$} and denote it by $\Gal(k)$.
\end{construction}


\begin{remark}
	Separability of an extension $L/k$ is in effect a smoothness condition, reminiscent of the conditions under which the inverse mapping theorem from analysis holds
\end{remark}

\begin{theorem}[Main theorem of Galois theory]\label{theorem:galois}
	Let $K$ be a Galois extension of $k$ and $L$ a subextension. Then $\Gal(K/L)$ is a closed subgroup of $\Gal(K/k)$. Moreover, there is an inclusion reversing bijection between subfields of $K$ and closed subgroups $H \subset G$, given by the maps
	\[
		L \to H \coloneqq \Gal(K/L) \quad \text{ and } \quad H \to L \coloneqq K^H,
	\]
	where $K^H$ denotes the subfield of $K$ fixed by $H$. A subextension $L/k$ is Galois over $k$ if and only if $\Gal(K/L)$ is normal in $\Gal(K/k)$. In this case there is an isomorphism of groups $\Gal(L/k) \simeq \Gal(K/k)/\Gal(K/L$
\end{theorem}

\begin{proof}
	See \cite{Szamuely}, Theorem 1.3.11.
\end{proof}

\begin{definition}[\'Etale Algebras and Schemes \'etale over a Point]
	A $k$-algebra $A$ is said to be \'etale over $k$ if it is isomorphic to a product of separable extensions of $k$. If $A$ is finite dimensional over $k$ we say that $A$ is finite \'etale over $k$. We define $\Spec(A) \to \Spec(k)$ to be \'etale (respectively finite \'etale) if $A$ is an \'etale (respectively finite \'etale) algebra over $k$.
\end{definition}

\begin{construction}\label{construction:separable_category}
	Consider the category $\Sep/k$ of \'etale algebras over $k$ and fix separable and algebraic closures $k^\sep \subset \overline{k}$. The initial object of this category is $k$ itself. Consider the representable functor
	\[
		\Hom(-, k^\sep) \colon \Sep/k \to \Gal(k)\text{-Set}.
	\]
	For a finite separable extension $L$ of $k$, the number of homomorphisms from $L$ into $k^\sep$ is $[L:k] = n$, so the cardinality of $\Hom(L, k^\sep)$ is $n$. This is indeed a $\Gal(k)$-Set, with the action given by $(g, \varphi) \to g \circ \varphi$ for $g \in \Gal(k)$ and $\varphi \in \Hom_k(L, k^\sep)$. If $k_s$ is an infinte extension, $\Gal(k)$ is a profinite group and hence carries a totally disconnected topology. More generally, consider an \'etale $k$-algebra $A \simeq \prod L_i$. Then the number of homomorphisms from $A$ to $k^\sep$ are $\prod n_i$
\end{construction}

\subsection{Discrete \texorpdfstring{$G$}{G}-modules}
The ultimate goal of this section is now to construct an equivalence of categories between the category of finite \'etale $k$-algebras and the category of finite sets with continuous let $\Gal(k)$-action. In general, sets with a continuous $G$-action for a fixed group $G$ will be relevant to us, so we make some preliminary definitions and construct some examples.

\begin{definition}[Profinite Groups]
	Let $(I, \le)$ be a poset such that for all $i, j \in I$ there exists some $k \in I$ such that $i \le k$ and $j \le k$. Let $\Gamma \colon I \to \text{Groups}$ be a diagram in groups, with homomorphisms between the groups denoted by $\psi_{ij} \colon G_j \to G_i$. If each $G_i \coloneqq \Gamma(i)$ is finite, the inverse limit of this system is said to be a \textit{profinite group}.
\end{definition}

\begin{remark}
	We always consider profinite groups as topological groups as follows: Let $G = \varprojlim G_i$ be a profinite group and let each $G_i$ carry the discrete topology. Recall that we can explicitly construct $G$ as a subgroup of $\prod_i G_i$. Now we can equip $G$ with the subspace topology. The natural projection maps $G \to G_i$ are continuous and their kernels form a basis of open neighbourhoods of $1$ in $G$.
\end{remark}

\begin{lemma}
	The inverse limit $\varprojlim G_i$ of a system of groups equipped with the discrete topology is a closed subgroup of $\prod G_i$.
\end{lemma}

\begin{proof}
	Take an element $g = (g_i) \in \big(\prod G_i \setminus \varprojlim G_i \big)$. We will show that $g$ is contained in an open neighborhood $U$ with $\varprojlim G_i \cap U  = \varnothing$. By assumption, there are some $i$ and $j$ such that $\psi_{ij}(g_j) \neq g_i$. Define $U$ to be the subset of $\prod G_i$ consisting of all elements with $i$-th component $g_i$ and $j$-th component $g_j$. By the fact that each $G_i$ is discrete and by the definition of the product topology, this set contains $g$ and has empty intersection with $\varprojlim G_i$.
\end{proof}

\begin{corollary}
	A profinite group is compact.
\end{corollary}

\begin{proof}
	Finite discrete groups are compact and by Tychonoff's theorem the product of compact groups is compact as well. Thus the statement follows from the fact that $\varprojlim G_i$ is closed in $\prod G_i$.
\end{proof}

\begin{corollary}
	The open subgroups of a profinite group $G$ are precisely the closed subgroups of finite index.
\end{corollary}

\begin{proof}
	Note that for each open subgroup $U$ the map $U \to gU$ is a homeomorphism for each $g \in G$. Thus each open subgroup $U$ is closed since its complement is the union of open cosets $gU$. By compactness of $G$, there are a finite number of thse cosets. Conversely, a closed subgroup of finite index must be open because it is the complement of a finite union of its cosets, which are closed.
\end{proof}

\begin{remark}
	Let $L/k$ be an infinite Galois extension. The Galois group of $L$ over $k$ is isomorphic to the inverse limit of all finite Galois subextensions of $L/k$, which indeed forms an inverse system.
\end{remark}

\begin{definition}[Discrete $G$-modules]
	For a topological group $G$, a discrete $G$-module is a set $X$ endowed with a left action
	\[
		m \colon G \times M \to M
	\]
	such that $a$ is continuous if $X$ is given the discrete topology. Denote by $BG$ the category of \textit{right $G$-sets} whose objects are sets $X$ equipped with a right $G$-action $X \times G \to X$ which is continuous when $X$ is given the discrete topology. Let $\Hom_G(X,Y)$ denote the set of morphisms in $BG$, which consists $G$-equivariant functions.
\end{definition}


\begin{lemma}\label{lemma:stabilizer}
	Let $G$ be a topological group and $X$ a discrete space with a $G$-action $m \colon G \times X \to X$. Then $m$ is continuous if and only if each stabilizer
	\[
		G_x = \{ g \in G \mid gx = x \}
	\]
	is open in $G$.
\end{lemma}

\begin{proof}
	Consider the preimage $U_x \coloneqq \{ (g,y) \in G \times X \colon gy = x\}$ of a point $x \in X$ under the map $m \colon G \times X \to X$. We can write $U_x$ as a disjoint union of sets $\{(g,y) \in G \times \{y\} \colon gy = x\}$. These sets are either empty or homeomorphic to $G_x$ under the map $g \to (gh, y)$, where $h$ is such that $hy = x$, which implies that $ghy = gx = x$, so we have that the openness of $G_x$ implies the continuity of $m$. Conversely, $G_x$ is the preimage of $x$ under the composition $G \xrightarrow{i_x} G \times X \xrightarrow{m} X$, where $i_x(g) = (g,x)$, so $G_x$ is open if $m$ is continuous.
\end{proof}

\subsection{Classification theorems for \'etale algebras}

\begin{theorem}
	The action of $\Gal(k)$ on $\Hom_k(L, k^\sep)$ is continuous and transitive.
\end{theorem}

\begin{proof}
	The stabilizer $U$ of an element $f$ of $\Hom_k(L, k^\sep)$ consists of those elements of $\Gal(k)$ which fix $f(L)$. By theorem \ref{theorem:galois}, $U$ is open in $\Gal(k)$, so the action of $\Gal(k)$ is continuous by lemma \ref{lemma:stabilizer}. Suppose the extension $L$ of $k$ is generated by a primitive element $\alpha$. Then each element $f \in \Hom_k(L, k^\sep)$ is given by a choice of a root of $f$ in $k^\sep$. The Galois group $\Gal(k)$ permutes threse roots transitively and the statement follows.
\end{proof}

Let $U_f$ be the open stabilizer of some element $f \in \Hom_k(L, k^\sep)$ and suppose $U_f$ is normal in $\Gal(k)$. The previous proof shows that the map $f \to U_f$ induces an isomorphism between $\Hom_k(L, k^\sep)$ and $\Gal(k)/U_f$ as $\Gal(k)$-sets, and by theorem \ref{theorem:galois}, this happens if and only if $L$ is Galois over $k$. We have shown:

\begin{theorem}
	Let $L$ be Galois over $k$. The left coset space $\Hom_k(L, k^\sep)$ is isomorphic to a quotient of $\Gal(k)$ by an open normal subgroup.
\end{theorem}

\begin{theorem}
	The functor mapping an \'etale algebra to a $\Gal(k)$-set restricts to an equivalence between the category of finite separable extensions of $k$ and the category of finite sets with a continuous and transitive left $\Gal(k)$-action, with Galois extensions $L$ corresponding to $\Gal(k)$-sets isomorphic to a finite quotient of $\Gal(k)$.
\end{theorem}

\begin{proof}
	We show that the functor $\Hom_k(-,k^\sep)$ is fully faithful and essentially surjective. For essential surjectivity, let $X$ be any continuous transitive left $\Gal(k)$-set and let $x \in X$. The stabilizer $U_s$ is an open subgroup of $\Gal(k)$ and hence fixes a finite separable extension $L$ of $k$. Let $i \colon L \to k^\sep$ be the natural inclusion and $g$ an element of $\Gal(k)$. The map $g \circ i \to gs$ is a map of $\Gal(k)$-sets from $\Hom_k(L,k^\sep) \to X$ and in fact an isomorphism: both become isomorphic to the left coset space $U_s\setminus \Gal(k)$.
	To see that the functor is fully faithful, we need to show a bijective correspondence between $k$-homomorphisms $L \to K$ and maps of $\Gal(k)$-sets $\Hom_k(K, k^\sep) \to \Hom_k(L, k^\sep)$. Because the action of $\Gal(k)$ is transitive on both sets, a map
	\[
		\varphi \colon \Hom_k(K, k^\sep) \to \Hom_k(L, k^\sep)
	\]
	between them is determened by the image of a fixed $f \in \Hom_k(K, k^\sep)$. Because $\varphi$ is $\Gal(k)$-equivariant, the elements of the stabilizer $U_f \subseteq \Gal(k)$ fix $\varphi(f)$ as well. We obtain an inclusion $U_f \subset U_{\varphi(f)}$ of open subgroups of $\Gal(k)$, hence an inclusion of subfields $\varphi(f)(L) \subset f(K)$. The map $f(K) \to K$ is an isomorphism, and composing with the inverse $f^{-1} \circ \varphi(f)$ yields the unique element of $\Hom_k(L,M)$ inducing $\varphi$. The last statement follows from the previous
\end{proof}

\begin{theorem}[Main Theorem of Galois Theory, Grothendieck's version]\label{theorem:galois_grothendieck}
	Let $k$ be a field. The functor mapping a finite \'etale $k$-algebra $A$ to the finite set $\Hom_k(A, k_s)$ induces an equivalence of categories between the cateogry of \'etale $k$-algebras and the category of sets with a continous left $\text{Gal}(k)$-action
\end{theorem}

\begin{proof}
	$A$ is isomorphic to a product of finite separable extension $\prod L_i$.Each element of the $\Gal(k)$-set $\coprod_i \Hom_k(L_i, k_s)$ has an open stabilizer by definition of the profinite topology on $\Gal(k)$. Conversely consider a $\Gal(k)$ set $T$ and write it as a disjoint union of orbits $\coprod T_i$. Pick an element $t_i \in T_i$ and let $G_i \subset \Gal(k)$ be its open stabilizer. Then the scheme
	\[
		\coprod_i \Spec(k_\sep^{G_i}) = \Spec\big(\prod(k^{G_i}_\sep)\big)
	\]
	is \'etale over $k$ because by Galois theory each $k_\sep^{G_i}$ is a finite separable extension of $k$. This gives an inverse to the functor.
\end{proof}

\begin{theorem}
	Let $A$ be a finite dimensional commutative $k$-algebra and denote by $\overline{A}$ the $\overline{k}$-algebra $A \otimes_k \overline{k}$. The following are equivalent:
	\begin{enumerate}
		\item $A$ is \'etale over $k$.\label{etale}
		\item $A \otimes_k \overline{k}$ is isomorphic to a finite product of copies of $\overline{k}$, hence \'etale over $\overline{k}$.\label{product}
		\item $A \otimes_k \overline{k}$ is reduced
		      %\item The discriminant of any basis of $A$ over $k$ is nonzero.\label{trace}
		      %The trace pairing $A \times A \to k, (x,y) \to \text{Tr}(xy)$ is nondegenerate.
	\end{enumerate}
\end{theorem}

\begin{proof}
	See \cite{Szamuely}, Proposition 1.5.6.
	%(\Implies{etale}{product}):
	%Suppose that $L$ is a finite separable extension of $k$. This means that $L = k[x]/(f)$, where $f$ is a polynomial of degree $n$ which splits into $n$ distinct factors, say $(x-\alpha_i)$ in $\overline{k}$. By the chinese remainder theorem we have
	%\[L \otimes_k \overline{k} \cong \overline{k}[x]/(f) = \overline{k}[x]/(x-\alpha_1)\cdots(x-\alpha_n) \cong \prod_{i=1}^n \overline{k}[x] / (x-\alpha_i) \cong \prod_{i=1}^n \overline{k},\] from which the first direction follows.\\
	%(\Implies{etale}{trace}):
	%If $A = \prod k_i$ where each $k_i$ is a separable field extension of $k$, then $\text{disc}(A) = \prod \text{disc}(k_i)$ which is nonzero by the fact that $\text{disc}(k_i) \neq 0$ if $k_i$ is a separable extension.
	%%TODO finish the proof
\end{proof}

\subsection{\'Etale Algebras over Rings}
We will now generalize the previous section to algebras over an arbitrary commutative ring $R$. Let $L/K$ be a finite field extension of degree $n$. Each element $a$ of $L$ induces a $K$-linear map
\[
	m_a: L \to L,\ b \to ab.
\]
Since $L \cong K^{\oplus n}$ as a $K$-module, we may choose a basis of $L$ over $K$ and represent the map $m_a$ as an $n \times n$ matrix with entries in $K$. We may take the trace and determinant of this matrix. But the trace is independent of the choice of the choice of basis, so the trace map
\begin{gather*}
	\Tr_{L/K}(a) = \Tr(\text{matrix of }m_a)
\end{gather*}

is well defined. Let $L/K$ be a finite field extension. We define the \textit{trace pairing for $L/K$} to be the symmetric $K$-bilinear form
\begin{gather*}
	Q_{L/K} : L \times L \to K\\
	(a,b) \to \Tr_{L/K}(ab)
\end{gather*}

It is well known that a field extension $L/K$ is separable if and only if $Q_{L/K}$ is nondegenerate. (See \cite{konradSep}). This leads naturally to the notion of an \'etale algebra $A$ over an arbitrary commutative ring $R$.

\begin{definition}
	Let $R$ be a ring, $A$ a free $R$-algebra of finite rank. As before, every $a \in A$ defines an $R$-linear map by multiplication and the trace map $A \to R$ is well defined. We say that $A$ is a \textit{separable} or \textit{\'etale} algebra if the $R$-bilinear mapping
	\[
		Q : A \times A \to R,\ (a,b) \to \Tr(ab)
	\]
	is nondegenerate.
\end{definition}
