The \'etale cohomology of a scheme was defined by Alexandre Grothendieck, Michael Artin and Jean-Luis Verdier in the late 50's. Andr\'e Weil had conjectured the existence of a cohomology theory for varieties over finite fields, from which his famous Weil conjectures could be deduced.

The topology that arises naturally in algebraic geometry is the Zariski topology. It is
\section{\'Etale Morphisms}

\subsubsection{Local homeomorphisms}
In topology, a continuous map $f: X \to Y$ is called a local homeomorphism if $f$ restricts to a homeomorphism around any point $x \in X$. More precisely, for any point $x \in X$ there is a neighborhood $U$ of $x$ such that $f(U)$ is open in $Y$ and $p|_U: U \to p(U)$ is a homeomorphism. In this situation one also says that $X$ is an \'etale space over $Y$.
In the case of manifolds, this concept is easy to picture, since locally a manifold looks like $\mathbb{R}^n$ for some $n \in \mathbb{N}$. In the case of schemes, however, this is trickier.

\subsubsection{\'Etale morphsims of nonsingular varieties}
For varieties defined over an algebraically closed field $k$, one may use a formalism in direct analogy to smooth manifolds to define \'etale morphsims, namely the formalism of tangent spaces. However, instead of defining the tangent space at a point $p \in X$ analytically, we define it algebraically.

\subsubsection{\'Etale morphisms of affine schemes}
\begin{definition}
  A morphism $f: Y \to X$ is called an \'etale morphism if for every $y \in Y$ there exist open affine neighborhoods $V = spec B$ of $y$ and $Y = spec A$ of $f(y)$ such that 
  \[B = A[x_1, \dots, x_n]/(P_1, \dots, P_n)\]
  and $det(\partial P_i /\partial T_j)$ is a unit in $B$.
\end{definition}
This is not the standard definition of \'etale morphisms. Usually, an \'etale morphism is defined to be a flat and unramified morphism. In differential geometry, these notions correspond to






It formally mimics the notion of local homeomorphism in calculus. 

\begin{proposition}
  The composite of two \'etale morphisms and base change of an \'etale morphims are \'etale.
\end{proposition}

Computing sheaf cohomology on a topological space using the formalism of derived functors is often unfeasable. Calculations are often done using Cech cohomology. The same is possible for \'etale cohomology. The definition of Cech cohomology for \'etale sheaves works with basically no modifications.
\section{\'Etale sheaves}
Examples of \'Etale sheaves:

The structure sheaf $\OO_{X_{\text{\'et}}}(U) = \Gamma(U, \OO_U)$
\par
Representables $\mathcal{F}: \mathsf{Et}/X \to \Set , \mathcal{F}(U) = \Hom_X(U,Z)$ for any $X$-scheme $Z$. In particular this means that the \'etale topology is subcanonical.
\par

%It is a theorem that any category of sheaves on a site (in particular a topological space) is a topos. As such it carries a lot of structure

\begin{definition}
  Let $X$ be a scheme and $x$ a point of $X$. An \'etale neighborhood of $x$ is defined to be a pair $(U, u)$ together with an \'etale morphism $\varphi: U \to X$ such that $\varphi(u) = x$. A morphism of \'etale neighborhoods $f: (V,v) \to (U,u)$ is a morphism of $X$-schemes $f: V \to U$ such that $f(v) = u$.
\end{definition}
We obtain the category of \'etale neigborhoods of $x$ for each $x \in X$.

\begin{construction}
  Let $S$ be a topological space and let $O(S)$ be its lattice of open sets. Consider the sublattice $O_x(S) = \{ U \in O(S) | x \in U \}$. For two open sets $U, V \in O_x(S)$, the intersection $U \cap V$ is contained in both $U$ and $V$.
  A generalization of this situation is that of a \textit{cofiltered category}. We say that a category is \textit{cofilterd} if
  % finish this
\end{construction}

\begin{definition}
  Let $\varphi: X \to Y$ be a morphism of schemes over a field $k$. We say $\varphi$ is an affine map if every open affine subscheme $U = \Spec(R) \subseteq Y$ has as preimage an affine open subscheme $f^{-1}(U) = \Spec(S)$ of $X$. If, in addition, the corresponding map $R \to S$ makes $S$ into a finitely generated $R$-module, we say that $\varphi$ is finite.
\end{definition}

\begin{proposition}
Let $k$ be a field. Then $X$ is finite \'etale over $\Spec(k)$ iff $X$ is isomorphic do a disjoint union $\amalg \Spec(K_i)$, where each $K_i$ is a finite separable extension of $k$
\end{proposition}
\begin{proof}
  Consider first the case that $X$ is connected. It follows that $X = \Spec(A)$ and that $A$ is a vector space of finite dimension over $k$. If $A$ is not a field then it has a
\end{proof}


\section{The \'Etale Fundamental Group}
\subsection{The Fundamental Group in Topology}
\subsection{Classical Galois Theory}
Recall the main theorem of classical Galois theory:
\begin{theorem}[Main Theorem of Galois theory for finite extensions]
  Let $L/k$ be a finite Galois extension with Galois group $G$. There is an inclusion reversing bijection between subextension $ L \supset M \supset k$ and subgroups $H \subset G$. It is given by the maps
  \begin{align*}
    M &\mapsto \text{Aut}(L/M) \\
    H &\mapsto L^H.
  \end{align*}
  %missing some explanation
\end{theorem}
This theorem fails for infinite Galois extensions, as the following example shows:
% missing example
