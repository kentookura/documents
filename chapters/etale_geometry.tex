\section{\'Etale morphisms of schemes}
\begin{definition}
	A morphism of schemes $f: X \to Y$ is called \textit{affine} if there is an open affine cover by subsets $U_i$ such that $f^{-1}(U_i)$ is affine for each $i$. A morphism of schemes $f: X \to Y$ is called \textit{finite} if there is an affine cover by subsets $U_i = \Spec(A)$ such that $f^{-1}(U_i) = \Spec(B_i)$ is a finitely generated $A_i$-module. If moreover each $B_i$ is a free separable $A_i$-algebra, it is said to be \'etale.
\end{definition}
It follows that $f: \Spec(B) \to \Spec(A)$ is an \'etale morphism if and only if $B$ is an \'etale $A$-algebra.

There is the following structure theorem:
\begin{theorem}
	Let $A$ be a ring. An $A$-algebra $B$ is \'etale if it is of finite presentation and if the following equivalent conditions are satisfied:
	\begin{enumerate}
		\item For every $A$-algebra $C$ and the square zero ideal $I$ in $C$, the canonical map $\Hom_{A-\text{alg}}(B,C) \to \Hom_{A\text{-alg}}(B,C/I)$ is a bijection.
		\item $B$ is a flat $A$-module and $\Omega_{B/A} = 0$, where $\Omega_{B/A}$ is the module of relative differentials.
		\item Let $B = A[x_1, \dots, x_n]/I$ be a presentation of $B$. Then, for each point $p \in \Spec(A[x_1, \dots, x_n)]$ containing $I$, there exists functions $f_1, \dots, f_n \in I$ such that $I_p$ is generated by the images of $f_1, \dots, f_n$ and $\det(\partial f_i/\partial x_j) \not\in p$.
	\end{enumerate}
\end{theorem}


\begin{definition}\label{def:flatness}
	A ring homomorphism $f: A \to B$ is flat if the functor $- \otimes_A B$ is exact. A morphism $f: Y \to X$ of schemes is flat if for all afffine subsets $U \subset Y$ and $V \subset X$ with $f(V) \subset U$, the map $\Gamma(U, \mathcal{O}_X) \to \Gamma(V, \mathcal{O}_Y)$ is flat.
\end{definition}

\textbf{Examples of flat maps}
\begin{itemize}
	\item Open immersions are flat
	\item The composite of flat morphisms is flat
	\item Any base extension a flat morphism is flat
\end{itemize}

\begin{theorem}
	Let $M$ be a finitely generated $A$-module. The following are equivalent:
	\begin{itemize}
		\item $M$ is flat.
		\item $M_m$ is a free $A_m$-module for all maximal ideals $m$ of $A$.
		\item $\widetilde{M}$ is a locally free sheaf on $\Spec(A)$.
		\item $M$ is projective as an $A$-module.
	\end{itemize}
\end{theorem}

We recall some basic facts about flat maps and
\begin{lemma}
	\begin{enumerate}
		\item The composition of affine morphisms is affine.
		\item The composition of finite locally free morphisms is finite locally free.
		\item The composition of \'etale morphisms is \'etale.
	\end{enumerate}
\end{lemma}

\begin{proof}
	The first statement follows immediately from the fact that a morphism $f: X \to Y$ is affine if and only if for \textit{every} affine open $V \subseteq X$, $f^{-1}(V)$ is affine, see\cite{Hartshorne}.
\end{proof}

\begin{lemma}
	If $f: Y \to X$ is \'etale, then the base change $U \times_X Y$ along any morphism $U \to X$ is \'etale.
\end{lemma}

\begin{proof}

\end{proof}

\begin{definition}
	A surjective finite \'etale morphism is called an \'etale cover. An \'etale cover $\varphi : Y \to X$ is called trivial if $Y$ is isomorphic do a finite disjoint union of copies of $X$ $Y \cong \coprod X$ and $\varphi$ restricts to the identity on each component.
\end{definition}

\begin{theorem}
	\label{locallyTrivial}
	Let $X$ be a connected scheme and $\varphi : Y \to X$ an affine surjective morphism. Then $\varphi$ is finite \'etale if and only if there is a finite, locally free and surjective morphism $f: S \to X$ such that $Y \times_X S$ is a trivial cover of $S$.
\end{theorem}
\begin{proof}
	$(\implies)$
	We first show that $\varphi$ is finite and locally free. Since $f$ is locally free, each point of $X$ has an affine open neighborhood $U = \Spec(R)$ such that $f$ restricts to a morphism $\Spec(A) \to \Spec(R)$ where $A$ is a finitely generated and free $R$-module. Since $\varphi$ is affine, it restricts to $\Spec(B) \to \Spec(R)$ over $U$ for some $R$-algebra $B$. The basechange $S \times_X Y$ restricts to $\Spec(A \otimes_R B)$ over $\Spec(R)$ and we have the following commutative diagrams:
	\begin{figure}
		\centering
		\begin{minipage}{0.4\textwidth}
			\begin{tikzcd}
				% https://q.uiver.app/?q=WzAsNCxbMCwxLCJTIl0sWzEsMCwiWSJdLFsxLDEsIlgiXSxbMCwwLCJTIFxcdGltZXNfWCBZIl0sWzEsMiwiXFx2YXJwaGkiXSxbMCwyLCJmIiwyXSxbMywxXSxbMywwXV0=
				{S \times_X Y} & Y \\
				S & X
				\arrow["\varphi", from=1-2, to=2-2]
				\arrow["f"', from=2-1, to=2-2]
				\arrow[from=1-1, to=1-2]
				\arrow[from=1-1, to=2-1]
			\end{tikzcd}
		\end{minipage}
		\begin{minipage}{0.4\textwidth}
			% https://q.uiver.app/?q=WzAsNCxbMCwwLCJcXFNwZWMoQSBcXG90aW1lc19SQikiXSxbMCwxLCJcXFNwZWMoQSkiXSxbMSwxLCJcXFNwZWMoUikiXSxbMSwwLCJcXFNwZWMoQikiXSxbMCwxXSxbMCwzXSxbMSwyXSxbMywyXV0=
			\begin{tikzcd}
				{\Spec(A \otimes_RB)} & {\Spec(B)} \\
				{\Spec(A)} & {\Spec(R)}
				\arrow[from=1-1, to=2-1]
				\arrow[from=1-1, to=1-2]
				\arrow[from=2-1, to=2-2]
				\arrow[from=1-2, to=2-2]
			\end{tikzcd}
		\end{minipage}
	\end{figure}

	By assumption $A \otimes_R B$ is a finitely generated and free $A$-module, so it is also fnitely generated and free as an $R$-module. It is also isomorphic to a finite direct sum of copies of $A$. This can only happen if $B$ is finitely generated and free over $R$. \par

	Now let $\overline{x} : \Spec(\overline{k}) \to S$ be a geometric point of $S$. By composition with $f$ we get a geometric point of $X$. Now the geometric fibers $Y_{(f \circ \overline{x})}$ and $(S \times_X Y)_{\overline{x}}$  are isomorphic by the universal property of pullbacks.
	\[
		% https://q.uiver.app/?q=WzAsOSxbMiwxLCJYIl0sWzIsMCwiWSJdLFsxLDAsIllfeyhmIFxcY2lyYyBcXG92ZXJsaW5le3h9KX0iXSxbMSwxLCJcXG92ZXJsaW5le2t9Il0sWzMsMCwiKFMgXFx0aW1lc19YIFkpX3tcXG92ZXJsaW5le3h9fSJdLFszLDEsIlxcb3ZlcmxpbmV7a30iXSxbNCwwLCJTIFxcdGltZXNfWCBZIl0sWzQsMSwiUyJdLFswLDJdLFsxLDBdLFsyLDFdLFszLDAsImYgXFxjaXJjIFxcb3ZlcmxpbmV7eH0iLDJdLFsyLDNdLFs0LDVdLFs0LDZdLFs2LDddLFs1LDcsIlxcb3ZlcmxpbmV7eH0iLDJdXQ==
		\begin{tikzcd}
			& {Y_{(f \circ \overline{x})}} & Y & {(S \times_X Y)_{\overline{x}}} & {S \times_X Y} \\
			& {\overline{k}} & X & {\overline{k}} & S \\
			{}
			\arrow[from=1-3, to=2-3]
			\arrow[from=1-2, to=1-3]
			\arrow["{f \circ \overline{x}}"', from=2-2, to=2-3]
			\arrow[from=1-2, to=2-2]
			\arrow[from=1-4, to=2-4]
			\arrow[from=1-4, to=1-5]
			\arrow[from=1-5, to=2-5]
			\arrow["{\overline{x}}"', from=2-4, to=2-5]
		\end{tikzcd}
	\]
	By assumption, $(S\times_X Y)_{\overline{x}} \cong \prod \Spec(\overline{k})$ and since $f$ is surjective, all of the fibers of $\varphi$ are also \'etale.
\end{proof}
Theorem \ref{locallyTrivial} shows that \'etale morphsism are locally trivial in the \'etale topology.

\begin{remark}
	If we have an \'etale cover $f: Y \to X$ and consider a geometric point $\overline{x} : \Spec(\Omega) \to X$ of $X$, then the fiber of $f$ over $\overline{x}$ arises from the pullback
	% https://q.uiver.app/?q=WzAsNCxbMCwwLCJTcGVjKFxcT21lZ2EpXFx0aW1lc19YIFkiXSxbMCwxLCJTcGVjKFxcT21lZ2EpIl0sWzEsMCwiWSJdLFsxLDEsIlgiXSxbMCwxXSxbMCwyXSxbMiwzXSxbMSwzXV0=
	\[\begin{tikzcd}
			{\Spec(\Omega)\times_X Y} & Y \\
			{\Spec(\Omega)} & X
			\arrow[from=1-1, to=2-1]
			\arrow[from=1-1, to=1-2]
			\arrow[from=1-2, to=2-2]
			\arrow[from=2-1, to=2-2]
		\end{tikzcd}\]
	Since $\Spec(\Omega)\times_X Y$ is \'etale over $\Spec(\Omega)$ and $\Omega$ is algebraically closed, it follows that $\Spec(\Omega)\times_X Y \cong \coprod \Spec(\Omega)$.
\end{remark}

\subsection{Galois theory for \'etale covers}
Now we develop some Galois theory for \'etale morphisms of schemes. The nontriviality of the group of automorphisms $\Aut_X(S)$ is a key difference between the Zariski topology and the \'etale topology: As the open sets $U \subset X$ of a scheme $X$ form a poset, there is at most one morphism $U \to X$ when viewing the topological structure as a category. This is no longer the case for the Zariski topology and this fact plays a major role in \'etale cohomology.

\begin{definition}
	A connected finite \'etale cover $X \to S$ is said to be \textit{Galois} if its automorphism group $\Aut(X/S)$ over $S$ acts transitively on geometric fibers.
\end{definition}

%WRONG NOTATION USED IN THE PROPOSITION
\begin{proposition}\label{prop:galois_cover_construction}
	For any connected finite \'etale cover $f \colon Y \to X$ there exists a morphism $\varphi \colon S \to U$ such that $f \circ \varphi \colon S \to X$ is a finite \'etale Galois cover such that every $X$-morphism from a Galois cover to $Y$ factors through $f \circ \varphi$
\end{proposition}

\begin{proof}
	Fix a geometric point $\overline{s} \colon \Spec(\overline{k}) \to S$. The geometric fiber $X_{\overline{s}} = \Spec(\prod \overline{k}) \coprod \Spec(\overline{k})$ induces $n$ geometric points, where $n$ is the cardinality of the fiber. Because all the points of $X_{\overline{s}}$ have the same image in $S$, we may choose an ordering $(\overline{x}_1, \dots, \overline{x}_n)$ to obtain a geometric point $\overline{x}$ of the $n$-fold fiber product $X^n \coloneqq X \times_S \cdots \times_S X$. Now let $P$ be the connected component of $X^n$ containing $\overline{x}$. We claim that $P$ is a Galois cover of $S$. To show this, we need to prove that $\Aut(P/S$ acts transitively on the geometric fiber $P_{\overline{s}}$.
\end{proof}

\begin{definition}[The fiber functor $F_x$]
	Let $x: \Spec(\Omega) \to X$ be a geometric point, where $\Omega$ is an algebraically field. The fiber functor at $x$ associates to each \'etale cover $f: Y \to X$ the underlying set of $\Spec(\Omega) \times_X Y$.
\end{definition}

\begin{remark}
	In topology, the fiber functor is representable if $X$ is a connected and locally simply connected space. The representing object $\tilde{X}$ is called the universal covering. In algebraic geometry this functor is usually not representable. It is however pro-representable:
\end{remark}

\begin{definition}[Pro-representability]
	Let $C$ be a category and $F: C \to \Set$ a functor. We say that $F$ is \textit{pro-representable} if there exists an inverse system $(A_\alpha,\varphi_{\alpha \beta})$ in $C$ such that
	\[
		\varinjlim \Hom(A_\alpha, X) \cong F(X)
	\]
	for each $X$ in $C$.
\end{definition}

\begin{lemma}
	Let $f: Y \to X$ be a connected finite \'etale cover. There is a morphism $\varphi: S \to Y$ such that $f \circ \varphi: S \to X$ is a finite \'etale cover. Moreover every morphism over $X$ from a Galois cover to $Y$ factors through $\varphi$.
\end{lemma}

\begin{theorem}\label{thm:pro_rep}
	The fiber functor $F_{\overline{x}}$ is prorepresentable.
\end{theorem}

\begin{proof}[Proof of theorem \ref{thm:pro_rep}]
	We need to construct an inverse system in $\Et/X$ such that the above isomorphism holds. Take as index set $I$ all finite \'etale Galois covers $A_\alpha$ of $X$. This is a poset under the order $A_\alpha \le A_\beta$ if there is a morphism $A_\beta \to A_\alpha$. We can applyi proposition \ref{prop:galois_cover_construction} to
\end{proof}

\begin{theorem}
	Let $X$ be a scheme and $\overline{x} : \Spec(\Omega) \to X$ a geometric point.
\end{theorem}

\begin{definition}[Galois covers]
	A connected finite \'etale cover $Y \to X$ is \textit{Galois} if its group of $X$-automorphism acts transitively on geometric fibers.
\end{definition}


\begin{definition}[Affine maps of schemes]
	Let $\varphi: X \to Y$ be a morphism of schemes over a field $k$. We say $\varphi$ is an affine map if every open affine subscheme $U = \Spec(R) \subseteq Y$ has as preimage an affine open subscheme $f^{-1}(U) = \Spec(S)$ of $X$. If, in addition, the corresponding map $R \to S$ makes $S$ into a finitely generated $R$-module, we say that $\varphi$ is finite.
\end{definition}

\begin{proposition}
	Let $k$ be a field. Then $X$ is finite \'etale over $\Spec(k)$ if and only f $X$ is isomorphic to a disjoint union $\amalg \Spec(K_i)$, where each $K_i$ is a finite separable extension of $k$
\end{proposition}
\begin{proof}
	Consider first the case that $X$ is connected. It follows that $X = \Spec(A)$ and that $A$ is a vector space of finite dimension over $k$. If $A$ is not a field then it has a
\end{proof}



\begin{definition}[Faithfully flat morphisms]
	An $R$-module $M$ is \textit{flat} if the functor $N \to M \otimes_R M$ is exact. In other words: tensoring with $M$ preserves exact sequences. If the functor is also faithful, we say that $M$ is \textit{faithfully flat} over $R$.
\end{definition}
