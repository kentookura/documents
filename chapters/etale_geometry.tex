\section{\'Etale morphisms of Schemes}
Now we come to the main concept of this chapter. There are two main intuitions behind etale morphisms: Theorem \ref{} shows that we can trivialize

\begin{reminder}
	A morphism of schemes $f: X \to Y$ is called \textit{affine} if there is an open affine cover by subsets $U_i$ such that $f^{-1}(U_i)$ is affine for each $i$. A morphism of schemes $f: X \to Y$ is called \textit{finite} if there is an affine cover by subsets $U_i = \Spec(A)$ such that $f^{-1}(U_i) = \Spec(B_i)$ is a finitely generated $A_i$-module. If moreover each $B_i$ is a free separable $A_i$-algebra, it is said to be \'etale.
\end{reminder}

It follows that $f: \Spec(B) \to \Spec(A)$ is an \'etale morphism if and only if $B$ is an \'etale $A$-algebra. There is the following structure theorem:

\begin{theorem}
	Let $A$ be a ring. An $A$-algebra $B$ is \'etale if it is of finite presentation and if the following equivalent conditions are satisfied:
	\begin{enumerate}
		\item For every $A$-algebra $C$ and the square zero ideal $I$ in $C$, the canonical map $\Hom_{A-\text{alg}}(B,C) \to \Hom_{A\text{-alg}}(B,C/I)$ is a bijection.
		\item $B$ is a flat $A$-module and $\Omega_{B/A} = 0$, where $\Omega_{B/A}$ is the module of relative differentials.
		\item Let $B = A[x_1, \dots, x_n]/I$ be a presentation of $B$. Then, for each point $p \in \Spec(A[x_1, \dots, x_n)]$ containing $I$, there exists functions $f_1, \dots, f_n \in I$ such that $I_p$ is generated by the images of $f_1, \dots, f_n$ and $\det(\partial f_i/\partial x_j) \not\in p$.
	\end{enumerate}
\end{theorem}

\begin{remark}
	Geometrically, condition 1. says that there is a unique way to lift tangent vectors. Condition 3. i
\end{remark}

\begin{theorem}
	Let $f \colon X \to Y$ be a morphism of schemes. The following are equivalent:
	\begin{enumerate}
		\item The morphism $f$ is \'etale.
		\item The sheaf of relative differentials $\Omega^1_{X/Y}$ is $0$.
		\item The diagonal morphism $\Delta \colon X \to X \times_Y X$ is an isomorphisms onto an open and closed subscheme of $X \times_Y X$.
	\end{enumerate}
\end{theorem}

\begin{proof}
	See \cite{milneEC}, Proposition I.3.5.
\end{proof}
\begin{lemma}
	If $f: Y \to X$ is \'etale, then the base change $U \times_X Y$ along any morphism $U \to X$ is \'etale.
\end{lemma}

\begin{proof}

\end{proof}

\begin{remark}
	\'Etale maps satisfy the 2 out of 3 property. This means that for morphisms of schemes $f \colon X \to Y$ and $g \colon Y \to Z$, if any two of $f, g$ and $f \circ g$ are \'etale, then so is the third.
\end{remark}

\begin{definition}
	A surjective finite \'etale morphism is called a finite \'etale cover. An \'etale cover $\varphi : Y \to X$ is called trivial if $Y$ is isomorphic do a finite disjoint union of copies of $X$ $Y \cong \coprod X$ and $\varphi$ restricts to the identity on each component. Denote by $\text{Cov}_{\et}(X)$ the category of finite \'etale covers of $X$ whose morphisms are morphisms over $X$.
\end{definition}

\begin{remark}
	If we have an \'etale cover $f: Y \to X$ and consider a geometric point $\overline{x} : \Spec(\overline{k}) \to X$ of $X$, then the fiber of $f$ over $\overline{x}$ arises from the pullback
	% https://q.uiver.app/?q=WzAsNCxbMCwwLCJTcGVjKFxcT21lZ2EpXFx0aW1lc19YIFkiXSxbMCwxLCJTcGVjKFxcT21lZ2EpIl0sWzEsMCwiWSJdLFsxLDEsIlgiXSxbMCwxXSxbMCwyXSxbMiwzXSxbMSwzXV0=
	\[\begin{tikzcd}
			{\Spec(\overline{k})\times_X Y} & Y \\
			{\Spec(\overline{k})} & X
			\arrow[from=1-1, to=2-1]
			\arrow[from=1-1, to=1-2]
			\arrow[from=1-2, to=2-2]
			\arrow[from=2-1, to=2-2]
		\end{tikzcd}\]
	Since $\Spec(\overline{k})\times_X Y$ is \'etale over $\Spec(\overline{k})$ and $\overline{k}$ is algebraically closed, it follows that $\Spec(\overline{k})\times_X Y \cong \coprod \Spec(\overline{k})$.
\end{remark}

%\begin{lemma}[Path-lifting property of covering spaces]
%	Let $p: Y \to X$ be a cover, $y \in Y$ and $x = p(y) \in X$.
%	\begin{enumerate}
%		\item If $f: [0,1] \to X$ is a path in $X$ with $f(0) = x$, then there is a unique path $\tilde{f}: [0,1] \to Y$ with $\tilde{f}(0) = y$ and $p \circ \tilde{f} = f$
%		\item Assume we have a second path $g \colon [0,1] \to X$ homotopic to $f$ with the same endpoints. Then the unique lift $\tilde{g}$ of $g$ to $Y$ is homotopic to $\tilde{f}$ with the same endpoints.
%	\end{enumerate}
%\end{lemma}
%
%\begin{proof}
%	See~\cite{Szamuely}, Lemma 2.3.2.
%\end{proof}

\section{Galois Theory for \'Etale covers and the \'Etale Fundamental Group}
Now we develop some Galois theory for surjective \'etale morphisms of schemes. The nontriviality of the group of automorphisms $\Aut_X(S)$ is a key difference between the Zariski topology and the \'etale topology: As the open sets $U \subset X$ of a scheme $X$ form a poset, there is at most one morphism $V \to U$ when viewing the topological structure as a category. This is no longer the case for the \'etale topology and this fact plays a major role in \'etale cohomology. Surjective \'etale morphisms also play the role of covering spaces in algebraic geometry and we will prove theorems which make the analogy clear. In topology, covering spaces of $X$ and the fundamental group $\pi_1(X,x)$ of $X$ are related by Theorem \ref{theorem:fiber_functor} . This connection will guide us in defining the \'etale fundamental group $\pi^\et_1(X)$ of a scheme $X$. As an example consider the absolute Galois group $\Gal(\Q) = \G_\Q$ of the rational numbers. It may be considered as the \'etale fundamental group of $\Spec(\Q)$. As one of the most important objects in number theory it is also one of the most intractable. It is conjectured that every finite group arises as a subgroup of $\G_\Q$. In order to understand this large profinite group, we could study the representations of $\G_\Q$, which are homomorphisms $\G_\Q \to \GL_n(V)$, where $V$ is a finite dimensional $\Q$-vector space. It is not at all clear how to obtain such representations. Galois representations of $G_\Q$ arise from \'etale cohomology of varieties over $\Q$.

\subsection{Fundamental Groups in Topology}
The fundamental group is an important invariant of topological spaces. The fundamental group of a pointed topological space $(X,x)$, denoted by $\pi_1(X,x)$ may be defined in two equivalent ways: Via homotopy classes of loops in $X$ based at $x$, or as the automorphism group of the fiber functor $F_x: \text{Cov}(X) \to \Set$. The definition using loops in $X$ hints at the intuition that the fundamental group measures the connectivity of a space. The construction however makes essential use of the unit interval. The unit interval $[0,1] \subset \R$ is not an algebraic set so the definition using the interval does not have a natural formulation using only the notions of algebraic geometry \footnote{The reader might object that we could use the circle $S^1 \subset \R^2$, which is an algebraic set, to define loops. However, a loop $\gamma \colon S^1 \to X$ would still have to be an algebraic map, of which there are not enough.}. The second approach yields a theory strongly reminiscent of Galois theory. Grothendieck pioneered this approach in algebraic geometry.

\subsubsection{The fundamental group via paths}
A loop with base point $x$ is a map $\gamma : [0,1] \to X$ such that $\gamma(0) = \gamma(1) = x$. The homotopy class of $\gamma $ is denoted by $[\gamma]$. Define a composition operation by concatenation: $[\gamma] \circ [\eta] = [\gamma \circ \eta]$, where $\gamma \circ \eta$ is defined to be the path
\[
	\gamma \circ \eta =
	\begin{cases}
		\ \eta(2x) \text{ for } x \in [0, \tfrac{1}{2}] \\
		\ \gamma(2x - 1) \text{ for } x \in [\tfrac{1}{2}, 1].
	\end{cases}
\]
It is clear that this construction yields a group structure on the set of homotopy classes of loops at $x$:
\[
	\{\ [\gamma] \mid \gamma : [0,1] \to X,\ \gamma(0) = \gamma(1) = x \}
\]

We would like to have a notion of fundamental group for schemes. As before, there are some obstacles:

\begin{proposition}\label{scheme_contractible}
	An irreducible scheme is contractible.
\end{proposition}

\begin{proof}
	Define a map $f: X \times I \to X$  by $f(x,0) = x$ and $f(x,t) = \eta$ for $t > 0$. This is a contraction of $X$ onto the point $\eta$, so the singular cohomology of $X$ is identically 0.
\end{proof}

As every loop in an irreducible scheme is contractible, this implies that $\pi_1(X)$ is $0$. In a sense this means that the unit interval is not a suitable ``test space'' to probe varieties and schemes.

There is no universal covering space. In topology, the fiber functor $F_x \colon \text{Cov}(X) \to \Set$ is representable by a topological space. This is no longer the case in algebraic geometry. For instance, the algebra homomorphism given by $x^n \to x^n, k[x^n] \to k[x]$ corresponds to a map $x \to x^n, \mathbb{A}^1_k \to \mathbb{A}^1_k$. Depicted is the map from $\mathbb{A}^1_\C$ to $\mathbb{A}^1_\C$ for the case $n=2$. This map \textit{should} be local homeomorphism, but this is false in the Zariski topology. There do not exist nonempty open subsets $V$ and $U$ such that $x \to x^n$ maps $V$ isomorphically onto $U$. The theory of \'etale coverings will give us a good notion of covering which we will use to define the analog of universal coverings

The \'etale topology will provide a solution for the problems we have with both cohomology and fundamental groups. The \'etale topology is not a topology in the classical sense, but a category equipped with a notion of covering (in the sense of point-set topology). The analog of open sets of a scheme $X$ will be so called \'etale morphisms $U \to X$. \'Etale morphisms are the natural analogs of local homeomorphisms in scheme theory, so they provide a finer topology than the Zarisiki topology, giving rise to a better sheaf theory. Furthermore, surjective \'etale morphisms share a lot of formal properties with covering spaces, allowing us to define the \'etale fundamental group $\pi_1^{\et}(X, \overline{x})$.

\begin{remark}[Covering spaces]
	Let $X$ be a topological space. A \textit{space over $X$ } is a topological space $Y$ together with a continuous map $Y \to X$. A morphism between two spaces $Y_1, Y_2$ over $X$ is a continuous map $f: Y_1 \to Y_2$ such that the diagram
	\[
		% https://q.uiver.app/?q=WzAsMyxbMCwwLCJZXzEiXSxbMiwwLCJZXzEiXSxbMSwxLCJYIl0sWzAsMSwiZiJdLFsxLDIsInBfMiJdLFswLDIsInBfMSIsMl1d
		\begin{tikzcd}
			{Y_1} && {Y_2} \\
			& X
			\arrow["f", from=1-1, to=1-3]
			\arrow["{p_2}", from=1-3, to=2-2]
			\arrow["{p_1}"', from=1-1, to=2-2]
		\end{tikzcd}
	\]
	commutes. We obtain the category $\Top/X$ of spaces over $X$. A space $f: Y \to X$ over $X$ is called a local homeomorphism if for any point $y \in Y$ there is a neighborhood $U$ of $x$ such that the preimage $f^{-1}(U)$ is homeomorphic to a disjoint union of the open sets $f^{-1}(U) \cong \coprod U_i$ such that each $U_i$ gets mapped to $U$ homeomorphically under $f|_{U_i}$. Surjective local homeomorphisms over a space $X$ are also called \textit{covering spaces of $X$} or simply \textit{coverings}. Let $f: Y \to X$ be a surjective local homeomorphism. A trivial covering is one of the form $p: \coprod X \to X$, where $p$ restricts to the identity on each component. Using this terminology one can also say that a covering is a \textit{locally trivial local homeomorphism}. The following theorem shows that \'etale morphisms are also locally trivial in a suitable sense.
\end{remark}

\begin{theorem}
	\label{locallyTrivial}
	Let $X$ be a connected scheme and $\varphi : Y \to X$ an affine surjective morphism. Then $\varphi$ is finite \'etale if and only if there is a finite, locally free and surjective morphism $f: S \to X$ such that $Y \times_X S$ is a trivial cover of $S$.
\end{theorem}

\begin{proof}
	\par $\impliedby:$
	We first show that $\varphi$ is finite and locally free. Since $f$ is locally free, each point of $X$ has an affine open neighborhood $U = \Spec(R)$ such that $f$ restricts to a morphism $\Spec(A) \to \Spec(R)$ where $A$ is a finitely generated and free $R$-module. Since $\varphi$ is affine, it restricts to $\Spec(B) \to \Spec(R)$ over $U$ for some $R$-algebra $B$. The basechange $S \times_X Y$ restricts to $\Spec(A \otimes_R B)$ over $\Spec(R)$ and we have the following commutative diagrams:
	\[
		\centering
		\begin{minipage}{0.4\textwidth}
			\begin{tikzcd}
				% https://q.uiver.app/?q=WzAsNCxbMCwxLCJTIl0sWzEsMCwiWSJdLFsxLDEsIlgiXSxbMCwwLCJTIFxcdGltZXNfWCBZIl0sWzEsMiwiXFx2YXJwaGkiXSxbMCwyLCJmIiwyXSxbMywxXSxbMywwXV0=
				{S \times_X Y} & Y \\
				S & X
				\arrow["\varphi", from=1-2, to=2-2]
				\arrow["f"', from=2-1, to=2-2]
				\arrow[from=1-1, to=1-2]
				\arrow[from=1-1, to=2-1]
			\end{tikzcd}
		\end{minipage}
		\begin{minipage}{0.4\textwidth}
			% https://q.uiver.app/?q=WzAsNCxbMCwwLCJcXFNwZWMoQSBcXG90aW1lc19SQikiXSxbMCwxLCJcXFNwZWMoQSkiXSxbMSwxLCJcXFNwZWMoUikiXSxbMSwwLCJcXFNwZWMoQikiXSxbMCwxXSxbMCwzXSxbMSwyXSxbMywyXV0=
			\begin{tikzcd}
				{\Spec(A \otimes_RB)} & {\Spec(B)} \\
				{\Spec(A)} & {\Spec(R)}
				\arrow[from=1-1, to=2-1]
				\arrow[from=1-1, to=1-2]
				\arrow[from=2-1, to=2-2]
				\arrow[from=1-2, to=2-2]
			\end{tikzcd}
		\end{minipage}
	\]

	By assumption $A \otimes_R B$ is a finitely generated and free $A$-module, so it is also fnitely generated and free as an $R$-module. It is also isomorphic to a finite direct sum of copies of $A$. This can only happen if $B$ is finitely generated and free over $R$. \par

	Now let $\overline{x} : \Spec(\overline{k}) \to S$ be a geometric point of $S$. By composition with $f$ we get a geometric point of $X$. Now the geometric fibers $Y_{(f \circ \overline{x})}$ and $(S \times_X Y)_{\overline{x}}$  are isomorphic by the universal property of pullbacks.
	\[
		% https://q.uiver.app/?q=WzAsOSxbMiwxLCJYIl0sWzIsMCwiWSJdLFsxLDAsIllfeyhmIFxcY2lyYyBcXG92ZXJsaW5le3h9KX0iXSxbMSwxLCJcXG92ZXJsaW5le2t9Il0sWzMsMCwiKFMgXFx0aW1lc19YIFkpX3tcXG92ZXJsaW5le3h9fSJdLFszLDEsIlxcb3ZlcmxpbmV7a30iXSxbNCwwLCJTIFxcdGltZXNfWCBZIl0sWzQsMSwiUyJdLFswLDJdLFsxLDBdLFsyLDFdLFszLDAsImYgXFxjaXJjIFxcb3ZlcmxpbmV7eH0iLDJdLFsyLDNdLFs0LDVdLFs0LDZdLFs2LDddLFs1LDcsIlxcb3ZlcmxpbmV7eH0iLDJdXQ==
		\begin{tikzcd}
			& {Y_{(f \circ \overline{x})}} & Y & {(S \times_X Y)_{\overline{x}}} & {S \times_X Y} \\
			& {\overline{k}} & X & {\overline{k}} & S \\
			{}
			\arrow[from=1-3, to=2-3]
			\arrow[from=1-2, to=1-3]
			\arrow["{f \circ \overline{x}}"', from=2-2, to=2-3]
			\arrow[from=1-2, to=2-2]
			\arrow[from=1-4, to=2-4]
			\arrow[from=1-4, to=1-5]
			\arrow[from=1-5, to=2-5]
			\arrow["{\overline{x}}"', from=2-4, to=2-5]
		\end{tikzcd}
	\]
	By assumption, $(S\times_X Y)_{\overline{x}} \cong \prod \Spec(\overline{k})$ and since $f$ is surjective, all of the fibers of $\varphi$ are also \'etale.
	\par $\implies:$ By assumption $S$ is connected all the fibers of $\varphi$ have the same cardinality $n$. We proceed by induction on $n$, with the case $n = 1$ being trivial.
\end{proof}

\begin{theorem}[The Fiber Functor in Topology]\label{theorem:fiber_functor}
	Let $x \in  X$ be a point. The fiber functor at $x$ which associates to each covering space $f: Y \to X$ the set of $p^{-1}(x)$ induces an equivalence between the category of covers of $X$ and the category of left $\pi_1(X,x)$-sets.
\end{theorem}

\begin{proof}
	See \cite{Szamuely}, Theorem 2.3.4.
\end{proof}


\subsection{The Monodromy Action}
The monodromy action over a space $X$ is defined for all covering spaces of $X$.
\begin{construction}[The monodromy action]
	Let $p(y)=x$ and let $\alpha \in \pi_1(X,x)$ be represented by a path $f: [0,1] \to X$. By the previous lemma, there is a unique lift $\tilde{f}$ with $\tilde{f}(0) = y$. Since $p \circ \tilde{f} = f$, we have $(p \circ \tilde{f})(1) = x$, so the fundamental group $\pi_1(X, x)$ acts on $F_x(Y)$. This action is called the \textit{monodromy action}.
\end{construction}

A $Y$-automorphism is defined to be an automorphism of $Y$ in $\Top/X$. Let $\text{Cov}(X)$ be the subcategory of $\Top/X$ consisting of covering spaces. For each point $x \in X$ there is a functor $F_x: \text{Cov}(X) \to \Set$, sending a covering $p: Y \to X$ to the set $\{y \in Y \mid p(y) = x\}$. Now define an automorphism of $F_x$ to be a natural transformation $\psi: F_x \to F_x$ with a two-sided inverse. The set $\text{Aut}(F_x)$ of automorphisms of $F_x$ carries the structure of a group by composition. Note that for a covering $Y$ of $X$ and an automorphism $\phi \in \text{Aut}(F_x)$, there is by definition a morphism $\phi(Y): F_x(Y) \to F_x(Y)$. We define $\pi_1(X,x)$ to be the automorphism group of $F_x$.

\begin{remark}
	Theorem \ref{locallyTrivial} shows that \'etale morphsism are locally trivial in the \'etale topology. This allows us to cover schemes using a finer topology than the Zariski topology. In topology we can trivialize a fiber bundle $E \stackrel{p}{\to} B$ over an open cover $\coprod U_i \to B$ such that each pullback $p^*\ U_i $ becomes isomorphic to a trivial bundle $\Delta(S) \times U_i$, where we view a cover as a disjoint union of open subsets $U_i \subset B$.
\end{remark}

\begin{definition}
	Let $\Et/X$ be the subcategory of schemes over $X$ whose sturcture maps to $X$ are \'etale.
\end{definition}

\begin{definition}
	A connected finite \'etale cover $X \to S$ is said to be \textit{Galois} if its automorphism group $\Aut(X/S)$ over $S$ acts transitively on geometric fibers. Denote by $\Gal/X$ the full subcategory of $\Et/X$ consisting of Galois \'etale covers.
\end{definition}

\begin{proposition}\label{prop:galois_cover_construction}
	For any connected finite \'etale cover $f \colon X \to S$ there exists a morphism $\varphi \colon G \to X$ such that $f \circ \varphi \colon G \to X$ is a finite \'etale Galois cover such that every $X$-morphism from a Galois cover to $X$ factors through $f \circ \varphi$
\end{proposition}

\begin{proof}
	Fix a geometric point $\overline{s} \colon \Spec(\overline{k}) \to S$ and consider the geometric fiber $X_{\overline{s}}$. Because all the points of $X_{\overline{s}}$ have the same image in $S$, we may choose an ordering $(\overline{x}_1, \dots, \overline{x}_n)$ to obtain a geometric point $\overline{x}$ of the $n$-fold fiber product $X^n \coloneqq X \times_S \cdots \times_S X$. Now let $G$ be the connected component of $X^n$ containing $\overline{x}$. We claim that $P$ is a Galois cover of $S$. To show this, we need to prove that $\Aut(P/S)$ acts transitively on the geometric fiber $P_{\overline{s}}$. Toward this end let $F = \{\overline{x}_1, \dots, \overline{x}_n\}$ be the set of $\Spec(\overline{k})$-points of $X_{\overline{s}}$. Now each point in the geometric fiber $X^n_{\overline{s}}$ corresponds to an element $F^n$. If each coordinate of
\end{proof}

\begin{lemma} \label{lemma:rigidity}
	If $Y$ is a connected scheme over $X$ and two $X$-morphisms $f_1, f_2 \colon Y \to Z$ are morphisms to a scheme $Z$ \'etale over $X$ with $f_1 \circ \overline{y} = f_2 \circ \overline{z}$ for some geometric point $\overline{y}: \Spec(\overline{k}) \to Y$, then $f_1 = f_2$.
\end{lemma}

\begin{proof}
	See~\cite{Szamuely}, Corollary 5.3.3.
\end{proof}

\begin{corollary}
	Let $f \colon Y \to X$ be a connected finite \'etale cover. The nontrivial elements of  $\Aut(Y/X)$ acts without fixed points on each geometric fiber. The group $\Aut(X/S$ is finite.
\end{corollary}

\begin{proof}
	Let $\varphi$ be some automorphism of $Y$ over $X$. Then $f = f \circ \varphi$ so we may apply the previous lemma and the first statement follows. The second statement follows from the fact that the permutation representation of $\Aut(X/S)$ on the underlying finite set of geometric fibers if faithful.
\end{proof}

\begin{definition}[The Fiber Functor in Algebraic Geometry]
	Let $x: \Spec(\Omega) \to X$ be a geometric point, where $\Omega$ is an algebraically field. The fiber functor at $x$ associates to each \'etale cover $f: Y \to X$ the underlying set of $\Spec(\Omega) \times_X Y$.
\end{definition}

\begin{definition}[Automorphisms of Functors]
	Let $F: C \to D$ be a functor. An \textit{automorphism of $F$} is a morphism of functors $\alpha \colon F \to F$ that has a two-sided inverse $\alpha^{-1}$. The automorphisms of $F$ form a group $\Aut(F)$ called the \textit{automorphism group of $F$}. Note that for each object $A$ of $C$, an automorphism $\alpha \in \Aut(F)$ induces an automorphism $F(A) \to F(A)$. If the functor $F$ takes values in $\Set$, there is a natural left action
	\[
		\Aut(F) \times F(C) \to F(C)
	\]
\end{definition}

\begin{definition}[The \'etale fundamental group]
	Let $X$ be a scheme and $\overline{x} \colon \Spec(k) \to X$ a geometric point. The \'etale fundamental group $\pi^{\et}_1(X,\overline{x})$ is defined to be the automorphism group of the functor $F_{\overline{x}}$.
\end{definition}



\begin{remark}
	In topology, the fiber functor is representable if $X$ is a connected and locally simply connected space. The representing object $\tilde{X}$ is called the universal cover. In algebraic geometry this functor is usually not representable. It is however pro-representable:
\end{remark}

\begin{definition}[Pro-representability]
	Let $C$ be a category and $F: C \to \Set$ a functor. We say that $F$ is \textit{pro-representable} if there exists an inverse system $(A_\alpha,\varphi_{\alpha \beta})$ in $C$ such that for each $X$ in $C$ we have
	\[
		\varinjlim \Hom(A_\alpha, X) \cong F(X)
	\]
\end{definition}

\begin{theorem}\label{thm:pro_rep}
	The fiber functor $F_{\overline{x}} \colon \Et/X \to \Set$ is pro-representable.
\end{theorem}

\begin{proof}
	We need to construct an inverse system in $\Et/X$ such that the above isomorphism holds. Take as index set $I$ all finite \'etale Galois covers $A_i$ of $X$. This is a poset under the order $A_i \le A_j$ if there is a morphism $A_j \to A_i$. We can apply proposition \ref{prop:galois_cover_construction} to a connected component $P$ of the fiber product $A_i \times_X A_j$ to obtain a Galois cover $A_k$ with maps $A_k \to P \to A_i$ and $A_k \to P \to A_j$. To obtain a well defined inverse system we need to pick a single morphism $\varphi_{ij} \colon A_i \to A_j$ for each $A_j \le A_i$. To do this fix an arbitrary element $p_j \in F_{\overline{x}}(A_j)$. Since $A_j \to S$ is Galois we may apply Corollary \ref{lemma:rigidity} to obtain a unique $S$-automorphism $\gamma$ of $A_i$ such that $(\varphi_{ij} \circ \gamma) (p_j) = p_i$. This yields a well defined inverse system $\Gal/X \coloneqq (A_i, \varphi_{ij})$ of groups such that $F_{\overline{x}}(\varphi_{ij})(p_j) = p_i$ whenever $i \le j$. Now for each $Y$ finite \'etale over $X$ and every $A_i$ Galois over $X$ there is a natural map $\Hom(A_i, Y) \to F_{\overline{x}}$ sending $f \in \Hom(A_i,Y)$ to $F_{\overline{x}}(f)(p_i)$. These maps are compatible with the maps $\varphi_{ij}$ in $\Gal/X$ so we obtain a well defined functorial map $\varinjlim \Hom_(A_i, Y) \to F_{\overline{x}}(Y)$. We construct an inverse to this map: By taking decomposing a scheme $Y$ into a disjoint union, we may assume that $Y$ is connected and by proposition \ref{prop:galois_cover_construction} there is a Galois cover $g: G \to Y$, where $G$ by definition is an object of the inverse system $\Gal/X$. Because $G$ is Galois, for each $\overline{y} \in F_{\overline{x}}(Y)$ there is a unique $X$-automorphism $\gamma$ such that $F_{\overline{x}}(g \circ \gamma)$ maps $p_i$ to $\overline{y}$. The map from $F_{\overline{y}}$ to $\varinjlim \Hom(A_i, Y)$ sending $\overline{y}$ to the class of $g \circ \gamma$ in $\varinjlim\Hom(A_i,Y)$ is the desired inverse.
\end{proof}

\begin{corollary}
	Every automorphism of $F_{\overline{x}}$ comes from a unique automorphism of $(A_i, \varphi_{ij})$.
\end{corollary}

\begin{proof}
	An automorphism of $F_{\overline{x}}$ must map the system of points $(p_i)$ to another system of points $(p'_i)$. Each $A_i$ is Galois, so for each $i$ there is a unique automorphism $\gamma_i \colon P_i \to P_i$ which sends $p_i$ to $p'_i$. Since the $p_i$ are compatible under the morphisms $\varphi_{ij}$, the automorphisms $\gamma_i$ are as well.
\end{proof}

\begin{remark}
	Unraveling definitions one immediately obtains the isomorphism $\pi^{\et}_1(\Spec(k), \overline{s}) \simeq \Gal(k)$
\end{remark}

\begin{definition}[Galois covers]
	A connected finite \'etale cover $Y \to X$ is \textit{Galois} if its group of $X$-automorphism acts transitively on geometric fibers.
\end{definition}

\begin{definition}[Affine maps of schemes]
	Let $\varphi: X \to Y$ be a morphism of schemes over a field $k$. We say $\varphi$ is an affine map if every open affine subscheme $U = \Spec(R) \subseteq Y$ has as preimage an affine open subscheme $f^{-1}(U) = \Spec(S)$ of $X$. If, in addition, the corresponding map $R \to S$ makes $S$ into a finitely generated $R$-module, we say that $\varphi$ is finite.
\end{definition}

\begin{theorem}
	For $X$ a connected scheme and $\overline{x} \colon \Spec(\overline{k}) \to X$ a geometric point, the group $\pi^{\et}_1(X,\overline{x})$ is profinite
\end{theorem}

\begin{proof}
	By Theorem \ref{} there is an inverse system $(A_i, \varphi_{ij})$ of Galois covers which pro-represents $F_{\overline{x}}$. By \ref{}, each group $\Aut(A_i/S)$ is finite so $\pi^{\et}_1(X,\overline{x})$ is profinite.
\end{proof}

\begin{theorem}
	The action of $\pi^{\et}_1(X, \overline{x})$ on $F_{\overline{x}}(Y)$ is continuous and the functor $F_{\overline{x}}$ induces an equivalence between the category of finite continuous left $\pi^{\et}_1(X, \overline{x})$-sets and the category $\text{Cov}_{\et}(X)$ of finite \'etale covers of $X$. Under this correspondence connected covers of $X$ map to sets with transitive $\pi^{\et}_1(X, \overline{x})$-action and Galois covers map to finite quotients of $\pi^{\et}_1(X, \overline{x})$.
\end{theorem}


\begin{proof}
	The proof of the second statement is similar to the case of fields. It is proved in \cite{SGA1}, Expos\'e V. Theorem 4.1. in a more general context and applied to the case of schemes in Expos\'e V. 7.
\end{proof}

%In the literature on algebraic geometry, there are many different kinds of conditions that we may place on morphisms of schemes. Just as we will define the \'etale topology of a scheme, there are similarly the flat topology, the fpqc topology and the fppf topology, each stemming from a different notion of morphism. Because every \'etale morphism is flat, many proofs in the literature just carry out proofs in the flat topology since the corresponding statement for the \'etale topology follows. For example every sheaf for the flat topology restricts to a sheaf for the \'etale and Zariski topology.
%
%\begin{definition}\label{def:flatness}
%	A ring homomorphism $f: A \to B$ is flat if the functor $- \otimes_A B$ is exact. A morphism $f: Y \to X$ of schemes is flat if for all afffine subsets $U \subset Y$ and $V \subset X$ with $f(V) \subset U$, the map $\Gamma(U, \mathcal{O}_X) \to \Gamma(V, \mathcal{O}_Y)$ is flat.
%\end{definition}
%
%\begin{example}
%	\begin{itemize}
%		\item Open immersions are flat and \'etale.
%		\item The composite of flat morphisms is flat.
%		\item Any base extension a flat morphism is flat
%	\end{itemize}
%\end{example}
%
%Let $M$ be a finitely generated $A$-module. The following are equivalent:
%\begin{itemize}
%	\item $M$ is flat.
%	\item $M_m$ is a free $A_m$-module for all maximal ideals $m$ of $A$.
%	\item $\widetilde{M}$ is a locally free sheaf on $\Spec(A)$.
%	\item $M$ is projective as an $A$-module.
%\end{itemize}
%
%\begin{lemma}
%	\begin{enumerate}
%		\item The composition of affine morphisms is affine.
%		\item The composition of finite locally free morphisms is finite locally free.
%		\item The composition of \'etale morphisms is \'etale.
%	\end{enumerate}
%\end{lemma}
%
%\begin{proof}
%	The first statement follows immediately from the fact that a morphism $f: X \to Y$ is affine if and only if for \textit{every} affine open $V \subseteq X$, $f^{-1}(V)$ is affine, see~\cite{Hartshorne}.
%\end{proof}
%
