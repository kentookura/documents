\section{Sheaf semantics}
We will now construct a semantics of intuitionistic propositional logic using a site $(C,J)$ and its topos $\sh(C,J)$. The basic idea is that we can use the structure of a site to speak of local truth. These semantics will provide us with a way to construct subobjects $\{x \mid \varphi(x) \} \rightarrowtail
	F$ by pulling back along the map $\text{true}: \mathbf{1} \rightarrowtail \Omega$:
% https://q.uiver.app/?q=WzAsNSxbMSwwLCJcXHt4ICBcXG1pZCBcXHZhcnBoaSh4KVxcfSJdLFsyLDAsIjEiXSxbMSwxLCJYIl0sWzIsMSwiXFxPbWVnYSJdLFswLDEsIlUiXSxbMCwxXSxbMCwyXSxbMiwzLCJcXHZhcnBoaSh4KSIsMl0sWzEsMywidHJ1ZSIsMCx7InN0eWxlIjp7InRhaWwiOnsibmFtZSI6Im1vbm8ifX19XSxbNCwwLCIiLDAseyJzdHlsZSI6eyJib2R5Ijp7Im5hbWUiOiJkYXNoZWQifX19XSxbNCwyLCJcXGFscGhhIl1d
\[\begin{tikzcd}
		& {\{x  \mid \varphi(x)\}} & 1 \\
		U & F & \Omega
		\arrow[from=1-2, to=1-3]
		\arrow[from=1-2, to=2-2]
		\arrow["{\varphi(x)}"', from=2-2, to=2-3]
		\arrow["\text{true}", tail, from=1-3, to=2-3]
		\arrow[dashed, from=2-1, to=1-2]
		\arrow["\alpha", from=2-1, to=2-2]
	\end{tikzcd}\]
The map $\{ x \mid \varphi(x) \} \to F$ has the property such that any other \'etale open $U \to X$ with $U \vDash \varphi$ factors through $\{x \mid \varphi(x) \}$ where $x$ is a variable of type $F$ and denote this by $x \colon F$. We will now formulate some conditions from scheme theory in this internal language. A good reference is \cite{Blechschmidt}.
%of a sheaf $F$ such that for any \'etale open $U \to X$ with $U \vDash \varphi$

\begin{center}
	\def\arraystretch{1.5}%  1 is the default, change whatever you need

	\begin{tabular}{|c|p{8cm}|}
		\hline
		Logic                               & Geometry                                                                                                             \\
		\hline
		$U \vDash \top$                     & $U = U$.                                                                                                             \\
		$U \vDash \bot$                     & $U = \varnothing$.                                                                                                   \\
		$U \vDash s \colon F$               & $s(U) \in F(U)$.                                                                                                     \\
		$U \vDash s = t \colon F$           & $s(u) = t(u) \in F(U)$.                                                                                              \\
		$U \vDash \varphi \land \psi$       & $U \vDash \varphi$ and $U \vDash \psi$.                                                                              \\
		$U \vDash \bigwedge_{j} \varphi_j$  & $ U \vDash \varphi_j $ for all $j$.                                                                                  \\
		$U \vDash \varphi \vee \psi$        & There is some covering $\coprod_i U_i \to U$ such that for all $i$ we have $U_i \vDash \varphi$ or $U_i \vDash \psi$ \\
		$U \vDash \bigvee_{j} \varphi_j$    & There is some covering $\coprod_i U_i \to U$ such that for all $i$ we have $U_i \vDash \varphi_j$ for some $j \in J$ \\
		$U \vDash \varphi \Rightarrow \psi$ & For all  $V \to U$ we have $V \vDash \varphi$ implies $V \vDash \psi$                                                \\
		\hline
	\end{tabular}
\end{center}

In order to construct a logic internal to $\sh(C,J)$ we need to inductively define terms of a language and subsequently define how to interpret these terms in the category $\sh(C,J)$.

%\subsection{Internal $\Hom$}
%In $\Set$ the product functor $- \times X$ admits a right adjoint denoted by exponential notation $(-)^X$. For a set $Z$, the set $Z^X$ is the set of all functions $f \colon X \to Z$. The bijection
%\[
%	\Hom(Y \times X, Z) \to \Hom(Y,Z^X)
%\]
%is sends a function $f \colon Y \times X \to Z$ to the function $f' \colon Y \to Z^X$. A category which has a terminal object, all products and for each $X$ the functor $Y \to Y \times X$ admits a right adjoint is called \textit{cartesian closed}. There is a connection between cartesian closed categories and the simply typed lambda calculus.
%
%\begin{proposition}
%	For a small category $C$, the category $\Set^{C^{\op}}$ is cartesian closed.
%\end{proposition}
%
%\begin{construction}[The Mitchell-B\'enabou Type Theory]
%	We inductively define terms of a language. Let $\mathcal{E}$ be a topos.
%	\begin{itemize}
%		\item For each object $X$ of $\mathcal{E}$ there is a type also denoted by $X$.
%		\item A variable $x \colon X$ is interpreted as the identity $X \to X$.
%	\end{itemize}
%\end{construction}
