% Mention names: Poincare, Brouwer, Betti, Vietoris, Cech, Grothendieck, Lawvere

%this is a complete mess
%\section*{History of \'Etale Cohomology}
%For varieties over the complex numbers $\textbf{C}$, one can use the topology inherited from $\textbf{C}$ to define cohomology. In more general settings this topology is not available. The natural topology defined on varietes and schemes, the Zariski topology is rather course. In the Zariski topology, every open set is dense. A theorem of grothendieck implies that constant sheaves on an irreducible variety has no cohomology. This problem was fixed by the definition of a Grothendieck topology. Here one replaces the open subsets $U \hookrightarrow X$ by a more general class of morphisms $V \to X$. In the case of \'etale cohomology, we consider the class of \'etale morphisms into $X$. 
%\cite{milneLEC}

