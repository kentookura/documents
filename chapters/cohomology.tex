
\section{De Rham Cohomology}

An important version of cohomology arises from the theory of integration on manifolds. Stoke's theorem, Gauß's theorem and Green's theorem from analysis are all special cases of cohomological phenomena. The idea behind it is that the differential forms that can arise on a manifold are closely related to its topology. 

\section{The Mayer-Vietoris Sequence}

The Mayer-Vietoris sequence is an important tool for computing cohomology. It relates the cohomology of a space to the cohomology of its parts. To be more precise, suppose $U$ and $V$ are two subspaces of $X$ such that their interior covers $X$. Then, roughly speaking, there is an exact sequence

$$
\cdots \to H^n(X) \to H^n(U) \oplus  H^n(V) \to H^n(U\cap V) \to H^{n+1}(X) \to \cdots      
$$

If the cohomology of $U$ and $V$ are known or easier to compute than the cohomology of$X$, this exact sequence provides a lot of information.


Cohomology groups are not just of interest in their own right. In many cases, one can show that the cohomology classes $[\gamma] \in H^n(X, \mathcal{A})$ correspond bijectively to some other kind of geometrical construction involving $X$.

For instance, when $p:Z \to B$ is a map of topological spaces and $\mathcal{G}$  is a sheaf of subgroups of $Aut_B(Z)$, one can define a notion of “twist of $p$ with structure sheaf $\mathcal{G}$”. One obtains the bijection
\begin{align}
            \left\{ \parbox[c]{1.1in}{\centering
                       Isomorphism classes of twists of $p$
                       with structure sheaf $\mathcal{G}$}
            \right\}
            \stackrel{\sim}{\longrightarrow}
            \parbox[c]{0.5in}{\centering
                       $H^1(B, \mathcal{G})$}
\end{align}

There is also a bijection
\begin{align}
            \left\{ \parbox[c]{1.1in}{\centering
              Isomorphism classes of vector bundles of rank $n$ with basis $B$}
            \right\}
            \stackrel{\sim}{\longrightarrow}
            \parbox[c]{0.5in}{\centering
            $H^1(B, GL_{n,B})$}
\end{align}
and in particular
\begin{align}
            \left\{ \parbox[c]{1.1in}{\centering
            Isomorphism classes of line bundles with basis $B$}
            \right\}
            \stackrel{\sim}{\longrightarrow}
            \parbox[c]{0.5in}{\centering
                $H^1(B, \mathcal{O}_B^\times)$}
\end{align}

%TODO construct simplicial cohomology, follow Hatcher

%To get a first taste of cohomology we consider a directed graph $X$. The set of all functions from the set of vertices of $X$to $\mathbb{Z}$ is an abelian group by pointwise addition. We will denote this group by $C^1_*(X)$ and call it the group of $1$-cochains. Likewise there is a group $C^2_*(X)$, consisting of functions from the edges of $X$ to $\mathbb{Z}$. We can interpret the vertices of $X$ to lie on a map, and an element $\varphi \in C^1_*(X)$ to assign an elevation to each vertex. There is then a natural map


\section{Brouwer fixed-point theorem}

\section{Lefschetz fixed-point theorem}

The Lefschetz fixed-point theorem generalizes the Brouwer fixed-point theorem

The definition of derived functor cohomology is very general but for computations it is more feasible to use an approximation tool called \v{C}ech cohomology.