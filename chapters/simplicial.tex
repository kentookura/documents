\section{Simplicial Complexes}

Simplicial complexes are a generalization of directed graphs (among other things). They are widely used in algebraic topology because they provide a combinatorial model for "nice" topological spaces. Every covering $\mathcal{U}$ of a space $X$ gives rise to a simplicial complex $\mathcal{N(U)}$, the nerve of $\mathcal{U}$. The nerve theorem asserts that if a covering is "sufficiently nice", this simplicial complex is a good approximation for $X$, in the sense that $\mathcal{N(U)}$ presents the homotopy type of $X$.\\
Let $\mathcal{U} = \{U_\alpha\} = \{U_\alpha\}_{\alpha \in A}$ be a covering of a topological space or an object of a site $X$. For any finite set $l = \{a_0, \dots a_n\} \subseteq A$ we write $U_l$ or $U_{a_0 \cdots a_n}$ instead of $U_{a_0} \cap \cdots \cap U_{a_n}$ or $U_{a_0} \times_X \cdots \times_X U_{a_n}$.



\begin{theorem}[Nerve Theorem]
  Let $X$ be a paracompact space and $\mathcal{U}$ an open cover of $X$ such that every nonempty intersection of finitely many $U_i \in \mathcal{U}$ is contractible. Then $X$ is homotopy equivalent to $\mathcal{N(U)}$.
\end{theorem}
\begin{proof}
  By a previous lemma. %4.G.2 in Hatcher, will try to a implement

\end{proof}