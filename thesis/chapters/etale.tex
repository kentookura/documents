There are two reasons why \'etale morphisms are important in algebraic geometry. They allow us to formulate covering space theory for schemes as they behave like local homeomorphisms. They also behave like generalised open sets, so we can consider sheaves on a sort of generalised topology on $X$ containing not only open subsets $U \subseteq X$ but more generally morphisms of schemes $Y \to X$. We will define the \textit{\'etale site of $X$}, denoted by $\Et/X$, a category whose objects are schemes $Y$ equipped with an \'etale morphism $\varphi \colon Y \to X$ and whose morphisms are morphisms over $X$. We declare a covering of an \'etale open $U$ in $\Et/X$ to be a jointly surjective family of \'etale maps: $\{f_i \colon U_i \to U\}$ with $\bigcup_i f(U_i) = U$. We will show that under mild hypotheses, an affine surjective morphism $\varphi \colon X \to S$ is \'etale if and only if there is a finite locally free and surjective morhpism $\psi \colon Y \to S$ such that $X \times_S Y \simeq \coprod_i Y$ is a trivial cover of $Y$, so only if it is locally trivial in this more general sense. A category equivalent to the category of sheaves on a site is called a Grothendieck topos. In the next chapter we will define the \'etale topos $\sh_{\et}(X)$ of $X$ which is the category of sheaves on $\Et/X$. The \'etale topos $\sh_{\et}(X)$ of $X$ may be thought of as a generalised space, locally modeled on $X$, with a close relation to the geometric properties of $X$.

\section{\'Etale Algebras}
We will first consider the case of \'etale algebras over a field $k$. Under the $\Spec$ functor, a finite \'etale algebra $k \to A$ gets mapped to to a disjoint union of points lying ``smoothly'' over $\Spec(k)$. This should be expected, as a space locally isomorphic to a point should just be a disjoint union of points.

\subsection{Reminder on field extensions and Galois theory}
\begin{construction}
	Let $k$ be a field and fix once and for all separable and algebraic closures $k_s \subset \overline{k}$ of $k$. Recall that an extension $L/k$ is separable if its minimal polynomial $p(x)$ has no multiple roots (in a splitting field of $p(x)$). It follows from basic algebra that a polynomial $p(x)$ is separable if and only if it is coprime to its derivative $p'(x)$. This means that $p'(x)$ gets mapped to a unit under the canonical map $k[x] \to k[x]/(p(x)) \simeq L$. An algebraic extension $L$ of $k$ is a \textit{Galois extension} if the elements of $L$ that are fixed under the automorphism group $\Aut_k(L)$ are precisely the elements of $k$.A separable extension $L/k$ is Galois if and only if the minimal polynomial of each element $a \in L$ splits into linear factors in $L$. If $L$ is Galois, we call $\Aut_k(L)$ the \textit{Galois group} of $L$ over $k$ and denote it by $\Gal(L/k)$. In particular $k_s$ is a Galois extension. We call its Galois group the \textit{absolute Galois group of $k$} and denote it by $\Gal(k)$.
\end{construction}

\begin{remark}
	Separability of an extension $L/k$ is in effect a smoothness condition, reminiscent of the conditions under which the inverse mapping theorem from analysis holds
\end{remark}

\begin{theorem}[Main theorem of Galois theory]
	Let $K$ be a Galois extension of $k$ and $L$ a subextension. Then $\Gal(K/L)$ is a closed subgroup of $\Gal(K/k)$. Moreover, there is an inclusion reversing bijection between subfields of $K$ and closed subgroups $H \subset G$, given by the maps
	\[
		L \to H \coloneqq \Gal(K/L) \quad \text{ and } \quad H \to L \coloneqq K^H,
	\]
	where $K^H$ denotes the subfield of $K$ fixed by $H$. A subextension $L/k$ is Galois over $k$ if and only if $\Gal(K/L)$ is normal in $\Gal(K/k)$.
\end{theorem}

\begin{proof}
	See \cite{Szamuely}, Theorem 1.3.11.
\end{proof}

\begin{definition}
	A $k$-algebra $A$ is \'etale over $k$ if it is isomorphic to a product of separable extensions of $k$. If $A$ is finite dimensional over $k$ we say that $A$ is finite \'etale over $k$. We define $\Spec(A) \to \Spec(k)$ to be \'etale (respectively finite \'etale) if $A$ is an \'etale (respectively finite \'etale) algebra over $k$.
\end{definition}

\begin{construction}
	Consider the category $\Sepk$ of \'etale algebras over $k$ and fix separable and algebraic closures $k^\sep \subset \overline{k}$. The initial object of this category is $k$ itself. Consider the representable functor
	\[
		\Hom(-, k^\sep) \colon \Sepk \to \Gal(k)\text{-Set}.
	\]
	For a finite separable extension $L$ of $k$, the number of homomorphisms from $L$ into $k^\sep$ is $[L:k] = n$, so the cardinality of $\Hom(L, k^\sep)$ is $n$. This is indeed a $\Gal(k)$-Set, with the action given by $(g, \varphi) \to g \circ \varphi$ for $g \in \Gal(k)$ and $\varphi \in \Hom_k(L, k^\sep)$. More generally, consider an \'etale $k$-algebra $A \simeq \prod L_i$. Then the number of homomorphisms from $A$ to $k^\sep$ are $\prod n_i$
\end{construction}

\subsection{Interlude ondiscrete \texorpdfstring{$G$}{G}-modules}
The ultimate goal of this section is now to construct an equivalence of categories between the category of finite \'etale $k$-algebras and the category of finite sets with continuous let $\Gal(k)$-action. In general, sets with a continuous $G$-action for a fixed group $G$ will be relevant to us, so we construct some examples and make some preliminary definitions.

\begin{definition}[Discrete $G$-modules]
	A discrete $G$-module is an abelian group $M$ endowed with a left action $a: G \times M \to M$ such that $a$ is continuous if $M$ is given the discrete topology.
\end{definition}

\begin{example}
	Let	$L$ be a finite separable extension of $k$. The set $\Hom_k(L, k_s)$ is finite has cardinality $[L:k]$. This set is endowed with a left action of $\Gal(k_s/k)$ given by $\varphi \to g \circ \varphi$ for $g \in \Gal(k)$ and $\varphi \in \Hom_k(L, k_s)$. If $k_s$ is an infinte extension, $\Gal(k)$ is a profinite group and hence carries a totally disconnected topology.
\end{example}

\begin{lemma}\label{lemma:stabilizer}
	Let $G$ be a topological group and $X$ a discrete space with a $G$-action $m \colon G \times X \to X$. Then $m$ is continuous if and only if each stabilizer
	\[
		G_x = \{ g \in G \mid gx = x \}
	\]
	is open in $G$.
\end{lemma}

\begin{proof}
	Consider the preimage $U_x \eqqcolon \{ (g,y) \in G \times X \colon gy = x\}$ of a point $x \in X$ under the map $m: G \times X \to X$. We can write $U_x$ as a disjoint union of sets $\{(g,y) \in G \times \{y\} \colon gy = x\}$. These sets are either empty or homeomorphic to $G_x$ under the map $g \to (gh, y)$, where $h$ is such that $hy = x$, which implies that $ghy = gx = x$, so we have that the openness of $G_x$ implies the continuity of $m$. Conversely, $G_x$ is the preimage of $x$ under the composition $G \xrightarrow{i_x} G \times X \xrightarrow{m} X$, where $i_x(g) = (g,x)$, so $G_x$ is open if $m$ is continuous.
\end{proof}

\subsection{Classification theorems for \'etale algebras}
Here we ``completely'' a number of categories.
\begin{proposition}
	The action of $\Gal(k)$ on $\Hom_k(L, k_s)$ is continuous and transitive. Furthermore, the functor mapping an \'etale algebra to a $\Gal(k)$-set restricts to an equivalence between the category of finite separable extensions of $k$ and the category of finite sets with a continuous and transitive left $\Gal(k)$-action.
\end{proposition}

\begin{corollary}
	If $L$ is Galois over $k$, then $\Hom_k(L,k_s) \cong \Gal(k)/\Gal(L)$.
\end{corollary}

\begin{proof}
	By theorem \ref{lemma:stabilizer}, the stabilizer $U_\varphi$ is open in $\Gal(k)$, so the action is continuous. By the theorem of the primitive elements, $L$ is generated by some $\alpha$ with minimal polynomial $f$ and each $\varphi \in \Hom_k(L,k_s)$ is given by mapping $\alpha$ to a root of $f$ in $k_s$, and $\Gal(k)$ permutes these roots transitively, and the statement follows.
\end{proof}\\

We obtain a map $\Hom_k(L,k_s) \to U \Gal(k)$, which is in fact an isomorphism of $\Gal(k)$-sets. Now if $U$ is normal we obtain the quotient $\Gal(k)/U$, which happens if and only if $L$ is Galois over $k$.


\begin{construction}
	Let $F$ be a presheaf on $\Et/\Spec(k)$. By abuse of notation we write $F(K)$ instad of $F(\Spec(K))$. We define a descrete $G$-module $M_F$ as follows: If $L/k$ is a finite separable extension, then $G = \Gal(k)$ acts on $F(L)$ by functoriality of $F$. Define $M_F = \colim F(L)$ where $L$ runs over all subfields $L$ of $k_s$ that are finite over $k$. Then $M_F$ is a discrete $G$-module. Conversely, for a discrete $G$-module $M$ we define a sheaf $F_M$ by setting $F_M(A) = \Hom_G(F(A),M)$ where $F(A) = \Hom_k(A,k_s)$. By Theorem \ref{thm:sheaf_condition}, presheaf $F$ on $\Et/k$ is a sheaf if and only if \ref{diagram:equalizer} is exact for a single affine cover $\Spec(A) \to \Spec(B)$, but affine \'etale covers in $\Et/k$ are precisely of the form $\displaystyle \coprod \Spec(L_i) \to \displaystyle \coprod \Spec(K_j)$ where each $L_i$ and $K_j$ is separable over $k$, so the sequence \[ F(\Spec(L)) \to \prod_{i} F(\Spec(L_i)) \xbigtoto{} \prod_{i,j} F(\Spec(L_i \otimes_L L_j)) \] needs to be exact.
\end{construction}

\begin{enumerate}
	\item $F(\prod A_i) = \bigoplus F(A_i)$ for every finite family $\{A_i\}$ of \'etale algebra
	\item $F(L) \xrightarrow{\sim} F(K)^{\Gal(K/k)}$ for finite Galois extensions $K/L/k$
\end{enumerate}

For $F$ a sheaf on $\Et/k$ define $M_F = \varinjlim F(K)$, where the colimit runs over all finite Galois extensions over $k$. Then $M_F$ is a discrete $\Gal(k)$-module. Converseley, if $M$ is a discrete $G$-module, define $F_m(A) = \Hom_{\Gal(k)}(F(A),M)$, where $F(A) = \Hom(A, k_s)$. Then $F_M$ is a sheaf on $\Spec(k)$. This defines an equivalence of categories between the \'etale topos $\sh(\Et/k)$ of $k$ and the category of discrete $G$-modules.

\begin{construction}
	Let $G$ be a topological group. Denote by $BG$ the category of \textit{right $G$-sets} whose objects are sets $X$ equipped with a right $G$-action $X \times G \to X$ which is continuous when $X$ is given the discrete topology. Let $\Hom_G(X,Y)$ denote the set of morphisms in $BG$, which consists $G$-equivariant functions.
\end{construction}



\begin{definition}

\end{definition}

\begin{theorem}[Main Theorem of Galois Theory, Grothendieck's version]\label{theorem:galois}
	Let $k$ be a field. The functor mapping a finite \'etale $k$-algebra $A$ to the finite set $\Hom_k(A, k_s)$ induces an equivalence of categories between the cateogry of \'etale $k$-algebras and the category of sets with a continous left $\text{Gal}(k)$-action
\end{theorem}

\begin{proof}
	$A$ is isomorphic to a product of finite separable extension $\prod L_i$.Each element of the $\Gal(k)$-set $\coprod_i \Hom_k(L_i, k_s)$ has an open stabilizer by definition of the profinite topology on $\Gal(k)$. Conversely consider a $\Gal(k)$ set $T$ and write it as a disjoint union of orbits $\coprod T_i$. Pick an element $t_i \in T_i$ and let $G_i \subset \Gal(k)$ be its open stabilizer. Then the scheme
	\[
		\coprod_i \Spec(k_s^{G_i})
	\]
	is \'etale over $k$ because by Galois theory each $k_s^{G_i}$ is a finite separable extension of $k$. This gives an inverse to the functor.
\end{proof}

The following theorem is strongly reminiscent of the following theorem from algebraic topology:

\begin{theorem}
	Let $(X,x)$ be a pointed, connected and locally simply connected topological space. The fiber functor $\text{Fib}_x$ induces an equivalence of categories between the category of covers of $X$ with the category of left $\pi_1(X,x)$-sets.
\end{theorem}

See Chapter (...) for the definitions used in the statement.

Let $L/K$ be a finite field extension of degree $n$. Each element $a$ of $L$ induces a $K$-linear map
\[
	m_a: L \to L,\ b \to ab.
\]
Since $L \cong K^{\oplus n}$ as a $K$-module, we may choose a basis of $L$ over $K$ and represent the map $m_a$ as an $n \times n$ matrix with entries in $K$. We may take the trace and determinant of this matrix. But the trace and determinant are independent of the choice of the choice of basis, so
\[
	\Tr_{L/K}(a) = \Tr(\text{matrix of }m_a)
\]
and
\[
	\Norm_{L/K}(a) = \text{det}(\text{matrix of }m_a).
\]
are well defined.

\begin{definition}
	Let $L/K$ be a finite field extension. The \textit{trace pairing for $L/K$} is the symmetric $K$-bilinear form
	\[
		Q_{L/K} : L \times L \to K,\ (a,b) \to \Tr_{L/K}(ab)
	\]
\end{definition}

\begin{lemma}
	Let $L/K$ be a finite field extension. The following are equivalent:
	\begin{enumerate}
		\item $L/K$ is separable.
		\item $\Tr_{L/K}$ is not identically zero.
		\item $Q_{L/K}$ is nondegenerate.
	\end{enumerate}
\end{lemma}

\begin{proof}
	The equivalence of (2.) and (3.) are clear.
	For the equivalence of (1.) and (2.) see (Keith Conrad, separability)
	%The trace map is $K$-linear with target $K$ so it is either identically 0 or surjective. We only have to consider the case where $K$ has characteristic $p \neq 0$, since finite extensions in characteristic 0 are always separable and $\Tr_{L/K}(1) = [L:K] \neq 0$ in characteristic 0.\par
	%Let $L/K$ be separable. By the primitive element theorem we can write $L = K(\alpha)$, where $\alpha$ is separable over $K$.
\end{proof}

\begin{theorem}
	Let $A$ be a finite dimensional commutative $k$-algebra and denote by $\overline{A}$ the $\overline{k}$-algebra $A \otimes_k \overline{k}$. The following are equivalent:
	\begin{enumerate}
		\item $A$ is \'etale over $k$.\label{etale}
		\item $A \otimes_k \overline{k}$ is isomorphic to a finite product of copies of $\overline{k}$, hence \'etale over $\overline{k}$.\label{product}
		\item $A \otimes_k \overline{k}$ is reduced
		\item The discriminant of any basis of $A$ over $k$ is nonzero.\label{trace}
		      %The trace pairing $A \times A \to k, (x,y) \to \text{Tr}(xy)$ is nondegenerate.
	\end{enumerate}
\end{theorem}

\begin{proof}
	(\Implies{etale}{product}):
	Suppose that $L$ is a finite separable extension of $k$. This means that $L = k[x]/(f)$, where $f$ is a polynomial of degree $n$ which splits into $n$ distinct factors, say $(x-\alpha_i)$ in $\overline{k}$. By the chinese remainder theorem we have
	\[L \otimes_k \overline{k} \cong \overline{k}[x]/(f) = \overline{k}[x]/(x-\alpha_1)\cdots(x-\alpha_n) \cong \prod_{i=1}^n \overline{k}[x] / (x-\alpha_i) \cong \prod_{i=1}^n \overline{k},\] from which the first direction follows.\\
	(\Implies{etale}{trace}):
	If $A = \prod k_i$ where each $k_i$ is a separable field extension of $k$, then $\text{disc}(A) = \prod \text{disc}(k_i)$ which is nonzero by the fact that $\text{disc}(k_i) \neq 0$ if $k_i$ is a separable extension.
	%TODO finish the proof
\end{proof}

We generalise the notion of \'etale algebra to algebras over a ring.

\begin{definition}
	Let $R$ be a ring, $A$ a free $R$-algebra of finite rank. As before, every $a \in A$ defines an $R$-linear map by multiplication and the trace map $A \to R$ is well defined. We say that $A$ is a \textit{separable} or \textit{\'etale} algebra if the $R$-bilinear mapping
	\[
		Q : A \times A \to R,\ (a,b) \to \Tr(ab)
	\]
	is nondegenerate.
\end{definition}

\section{\'Etale morphisms of schemes}
\begin{definition}
	A morphism of schemes $f: X \to Y$ is called \textit{affine} if there is an open affine cover by subsets $U_i$ such that $f^{-1}(U_i)$ is affine for each $i$. A morphism of schemes $f: X \to Y$ is called \textit{finite} if there is an affine cover by subsets $U_i = \Spec(A)$ such that $f^{-1}(U_i) = \Spec(B_i)$ is a finitely generated $A_i$-module. If moreover each $B_i$ is a free separable $A_i$-algebra, it is said to be \'etale.
\end{definition}
It follows that $f: \Spec(B) \to \Spec(A)$ is an \'etale morphism if and only if $B$ is an \'etale $A$-algebra.

There is the following structure theorem:
\begin{theorem}
	Let $A$ be a ring. An $A$-algebra $B$ is \'etale if it is of finite presentation and if the following equivalent conditions are satisfied:
	\begin{enumerate}
		\item For every $A$-algebra $C$ and the square zero ideal $I$ in $C$, the canonical map $\Hom_{A-\text{alg}}(B,C) \to \Hom_{A\text{-alg}}(B,C/I)$ is a bijection.
		\item $B$ is a flat $A$-module and $\Omega_{B/A} = 0$, where $\Omega_{B/A}$ is the module of relative differentials.
		\item Let $B = A[x_1, \dots, x_n]/I$ be a presentation of $B$. Then, for each point $p \in \Spec(A[x_1, \dots, x_n)]$ containing $I$, there exists functions $f_1, \dots, f_n \in I$ such that $I_p$ is generated by the images of $f_1, \dots, f_n$ and $\det(\partial f_i/\partial x_j) \not\in p$.
	\end{enumerate}
\end{theorem}


\begin{definition}\label{def:flatness}
	A ring homomorphism $f: A \to B$ is flat if the functor $- \otimes_A B$ is exact. A morphism $f: Y \to X$ of schemes is flat if for all afffine subsets $U \subset Y$ and $V \subset X$ with $f(V) \subset U$, the map $\Gamma(U, \mathcal{O}_X) \to \Gamma(V, \mathcal{O}_Y)$ is flat.
\end{definition}

In the technical proofs one finds in the literature, the following definitions are frequently used:
\textbf{Examples of flat maps}
\begin{itemize}
	\item Open immersions are flat
	\item The composite of flat morphisms is flat
	\item Any base extension a flat morphism is flat
\end{itemize}

Let $M$ be a finitely generated $A$-module. The following are equivalent:
\begin{itemize}
	\item $M$ is flat.
	\item $M_m$ is a free $A_m$-module for all maximal ideals $m$ of $A$.
	\item $\widetilde{M}$ is a locally free sheaf on $\Spec(A)$.
	\item $M$ is projective as an $A$-module.
\end{itemize}

\begin{lemma}
	\begin{enumerate}
		\item The composition of affine morphisms is affine.
		\item The composition of finite locally free morphisms is finite locally free.
		\item The composition of \'etale morphisms is \'etale.
	\end{enumerate}
\end{lemma}

\begin{proof}
	The first statement follows immediately from the fact that a morphism $f: X \to Y$ is affine if and only if for \textit{every} affine open $V \subseteq X$, $f^{-1}(V)$ is affine, see\cite{Hartshorne}.
\end{proof}

\begin{lemma}
	If $f: Y \to X$ is \'etale, then the base change $U \times_X Y$ along any morphism $U \to X$ is \'etale.
\end{lemma}

\begin{proof}

\end{proof}

\begin{remark}
	\'Etale maps satisfy the 2 out of 3 property. This means that for morphisms of schemes $f \colon X \to Y$ and $g \colon Y \to Z$, if any 2 of $f, g$ and $f \circ g$ are \'etale, then so is the third.
\end{remark}

\begin{definition}
	A surjective finite \'etale morphism is called an \'etale cover. An \'etale cover $\varphi : Y \to X$ is called trivial if $Y$ is isomorphic do a finite disjoint union of copies of $X$ $Y \cong \coprod X$ and $\varphi$ restricts to the identity on each component.
\end{definition}

\begin{theorem}
	\label{locallyTrivial}
	Let $X$ be a connected scheme and $\varphi : Y \to X$ an affine surjective morphism. Then $\varphi$ is finite \'etale if and only if there is a finite, locally free and surjective morphism $f: S \to X$ such that $Y \times_X S$ is a trivial cover of $S$.
\end{theorem}

\begin{remark}
	This structure allows us to cover schemes using a finer topology than the Zariski topology. In topology we can trivialize a fiber bundle
	\[
		F \to E \stackrel{p}{\to} B
	\]
	over an open cover $\coprod U_i \to B$ such that each pullback $p^*\ U_i $ becomes isomorphic to a trivial bundle $\Delta(S) \times U_i$, where we view a cover as a disjoint union of open subsets $U_i \subset B$. In algebraic geometry we need a generalized open set $U \to X$ to do this.
\end{remark}

\begin{proof}
	$(\implies)$
	We first show that $\varphi$ is finite and locally free. Since $f$ is locally free, each point of $X$ has an affine open neighborhood $U = \Spec(R)$ such that $f$ restricts to a morphism $\Spec(A) \to \Spec(R)$ where $A$ is a finitely generated and free $R$-module. Since $\varphi$ is affine, it restricts to $\Spec(B) \to \Spec(R)$ over $U$ for some $R$-algebra $B$. The basechange $S \times_X Y$ restricts to $\Spec(A \otimes_R B)$ over $\Spec(R)$ and we have the following commutative diagrams:
	\begin{figure}
		\centering
		\begin{minipage}{0.4\textwidth}
			\begin{tikzcd}
				% https://q.uiver.app/?q=WzAsNCxbMCwxLCJTIl0sWzEsMCwiWSJdLFsxLDEsIlgiXSxbMCwwLCJTIFxcdGltZXNfWCBZIl0sWzEsMiwiXFx2YXJwaGkiXSxbMCwyLCJmIiwyXSxbMywxXSxbMywwXV0=
				{S \times_X Y} & Y \\
				S & X
				\arrow["\varphi", from=1-2, to=2-2]
				\arrow["f"', from=2-1, to=2-2]
				\arrow[from=1-1, to=1-2]
				\arrow[from=1-1, to=2-1]
			\end{tikzcd}
		\end{minipage}
		\begin{minipage}{0.4\textwidth}
			% https://q.uiver.app/?q=WzAsNCxbMCwwLCJcXFNwZWMoQSBcXG90aW1lc19SQikiXSxbMCwxLCJcXFNwZWMoQSkiXSxbMSwxLCJcXFNwZWMoUikiXSxbMSwwLCJcXFNwZWMoQikiXSxbMCwxXSxbMCwzXSxbMSwyXSxbMywyXV0=
			\begin{tikzcd}
				{\Spec(A \otimes_RB)} & {\Spec(B)} \\
				{\Spec(A)} & {\Spec(R)}
				\arrow[from=1-1, to=2-1]
				\arrow[from=1-1, to=1-2]
				\arrow[from=2-1, to=2-2]
				\arrow[from=1-2, to=2-2]
			\end{tikzcd}
		\end{minipage}
	\end{figure}

	By assumption $A \otimes_R B$ is a finitely generated and free $A$-module, so it is also fnitely generated and free as an $R$-module. It is also isomorphic to a finite direct sum of copies of $A$. This can only happen if $B$ is finitely generated and free over $R$. \par

	Now let $\overline{x} : \Spec(\overline{k}) \to S$ be a geometric point of $S$. By composition with $f$ we get a geometric point of $X$. Now the geometric fibers $Y_{(f \circ \overline{x})}$ and $(S \times_X Y)_{\overline{x}}$  are isomorphic by the universal property of pullbacks.
	\[
		% https://q.uiver.app/?q=WzAsOSxbMiwxLCJYIl0sWzIsMCwiWSJdLFsxLDAsIllfeyhmIFxcY2lyYyBcXG92ZXJsaW5le3h9KX0iXSxbMSwxLCJcXG92ZXJsaW5le2t9Il0sWzMsMCwiKFMgXFx0aW1lc19YIFkpX3tcXG92ZXJsaW5le3h9fSJdLFszLDEsIlxcb3ZlcmxpbmV7a30iXSxbNCwwLCJTIFxcdGltZXNfWCBZIl0sWzQsMSwiUyJdLFswLDJdLFsxLDBdLFsyLDFdLFszLDAsImYgXFxjaXJjIFxcb3ZlcmxpbmV7eH0iLDJdLFsyLDNdLFs0LDVdLFs0LDZdLFs2LDddLFs1LDcsIlxcb3ZlcmxpbmV7eH0iLDJdXQ==
		\begin{tikzcd}
			& {Y_{(f \circ \overline{x})}} & Y & {(S \times_X Y)_{\overline{x}}} & {S \times_X Y} \\
			& {\overline{k}} & X & {\overline{k}} & S \\
			{}
			\arrow[from=1-3, to=2-3]
			\arrow[from=1-2, to=1-3]
			\arrow["{f \circ \overline{x}}"', from=2-2, to=2-3]
			\arrow[from=1-2, to=2-2]
			\arrow[from=1-4, to=2-4]
			\arrow[from=1-4, to=1-5]
			\arrow[from=1-5, to=2-5]
			\arrow["{\overline{x}}"', from=2-4, to=2-5]
		\end{tikzcd}
	\]
	By assumption, $(S\times_X Y)_{\overline{x}} \cong \prod \Spec(\overline{k})$ and since $f$ is surjective, all of the fibers of $\varphi$ are also \'etale.
\end{proof}
Theorem \ref{locallyTrivial} shows that \'etale morphsism are locally trivial in the \'etale topology.

\begin{remark}
	If we have an \'etale cover $f: Y \to X$ and consider a geometric point $\overline{x} : \Spec(\Omega) \to X$ of $X$, then the fiber of $f$ over $\overline{x}$ arises from the pullback
	% https://q.uiver.app/?q=WzAsNCxbMCwwLCJTcGVjKFxcT21lZ2EpXFx0aW1lc19YIFkiXSxbMCwxLCJTcGVjKFxcT21lZ2EpIl0sWzEsMCwiWSJdLFsxLDEsIlgiXSxbMCwxXSxbMCwyXSxbMiwzXSxbMSwzXV0=
	\[\begin{tikzcd}
			{\Spec(\Omega)\times_X Y} & Y \\
			{\Spec(\Omega)} & X
			\arrow[from=1-1, to=2-1]
			\arrow[from=1-1, to=1-2]
			\arrow[from=1-2, to=2-2]
			\arrow[from=2-1, to=2-2]
		\end{tikzcd}\]
	Since $\Spec(\Omega)\times_X Y$ is \'etale over $\Spec(\Omega)$ and $\Omega$ is algebraically closed, it follows that $\Spec(\Omega)\times_X Y \cong \coprod \Spec(\Omega)$.
\end{remark}

\subsection{Galois theory for \'etale covers}
Now we develop some Galois theory for \'etale morphisms of schemes. The nontriviality of the group of automorphisms $\Aut_X(S)$ is a key difference between the Zariski topology and the \'etale topology: As the open sets $U \subset X$ of a scheme $X$ form a poset, there is at most one morphism $U \to X$ when viewing the topological structure as a category. This is no longer the case for the Zariski topology and this fact plays a major role in \'etale cohomology.

\begin{definition}
	A connected finite \'etale cover $X \to S$ is said to be \textit{Galois} if its automorphism group $\Aut(X/S)$ over $S$ acts transitively on geometric fibers.
\end{definition}

%WRONG NOTATION USED IN THE PROPOSITION
\begin{proposition}\label{prop:galois_cover_construction}
	For any connected finite \'etale cover $f \colon Y \to X$ there exists a morphism $\varphi \colon S \to U$ such that $f \circ \varphi \colon S \to X$ is a finite \'etale Galois cover such that every $X$-morphism from a Galois cover to $Y$ factors through $f \circ \varphi$
\end{proposition}

\begin{proof}
	Fix a geometric point $\overline{s} \colon \Spec(\overline{k}) \to S$. The geometric fiber $X_{\overline{s}} = \Spec(\prod \overline{k}) \coprod \Spec(\overline{k})$ induces $n$ geometric points, where $n$ is the cardinality of the fiber. Because all the points of $X_{\overline{s}}$ have the same image in $S$, we may choose an ordering $(\overline{x}_1, \dots, \overline{x}_n)$ to obtain a geometric point $\overline{x}$ of the $n$-fold fiber product $X^n \coloneqq X \times_S \cdots \times_S X$. Now let $P$ be the connected component of $X^n$ containing $\overline{x}$. We claim that $P$ is a Galois cover of $S$. To show this, we need to prove that $\Aut(P/S$ acts transitively on the geometric fiber $P_{\overline{s}}$.
\end{proof}

\begin{definition}[The fiber functor $F_x$]
	Let $x: \Spec(\Omega) \to X$ be a geometric point, where $\Omega$ is an algebraically field. The fiber functor at $x$ associates to each \'etale cover $f: Y \to X$ the underlying set of $\Spec(\Omega) \times_X Y$.
\end{definition}

\begin{remark}
	In topology, the fiber functor is representable if $X$ is a connected and locally simply connected space. The representing object $\tilde{X}$ is called the universal covering. In algebraic geometry this functor is usually not representable. It is however pro-representable:
\end{remark}

\begin{definition}[Pro-representability]
	Let $C$ be a category and $F: C \to \Set$ a functor. We say that $F$ is \textit{pro-representable} if there exists an inverse system $(A_\alpha,\varphi_{\alpha \beta})$ in $C$ such that
	\[
		\varinjlim \Hom(A_\alpha, X) \cong F(X)
	\]
	for each $X$ in $C$.
\end{definition}

\begin{lemma}
	Let $f: Y \to X$ be a connected finite \'etale cover. There is a morphism $\varphi: S \to Y$ such that $f \circ \varphi: S \to X$ is a finite \'etale cover. Moreover every morphism over $X$ from a Galois cover to $Y$ factors through $\varphi$.
\end{lemma}

\begin{theorem}\label{thm:pro_rep}
	The fiber functor $F_{\overline{x}}$ is prorepresentable.
\end{theorem}

\begin{proof}[Proof of theorem \ref{thm:pro_rep}]
	We need to construct an inverse system in $\Et/X$ such that the above isomorphism holds. Take as index set $I$ all finite \'etale Galois covers $A_\alpha$ of $X$. This is a poset under the order $A_\alpha \le A_\beta$ if there is a morphism $A_\beta \to A_\alpha$. We can applyi proposition \ref{prop:galois_cover_construction} to
\end{proof}

\begin{theorem}
	Let $X$ be a scheme and $\overline{x} : \Spec(\Omega) \to X$ a geometric point.
\end{theorem}

\begin{definition}[Galois covers]
	A connected finite \'etale cover $Y \to X$ is \textit{Galois} if its group of $X$-automorphism acts transitively on geometric fibers.
\end{definition}


\begin{definition}[Affine maps of schemes]
	Let $\varphi: X \to Y$ be a morphism of schemes over a field $k$. We say $\varphi$ is an affine map if every open affine subscheme $U = \Spec(R) \subseteq Y$ has as preimage an affine open subscheme $f^{-1}(U) = \Spec(S)$ of $X$. If, in addition, the corresponding map $R \to S$ makes $S$ into a finitely generated $R$-module, we say that $\varphi$ is finite.
\end{definition}

\begin{proposition}
	Let $k$ be a field. Then $X$ is finite \'etale over $\Spec(k)$ if and only f $X$ is isomorphic to a disjoint union $\amalg \Spec(K_i)$, where each $K_i$ is a finite separable extension of $k$
\end{proposition}
\begin{proof}
	Consider first the case that $X$ is connected. It follows that $X = \Spec(A)$ and that $A$ is a vector space of finite dimension over $k$. If $A$ is not a field then it has a
\end{proof}



\begin{definition}[Faithfully flat morphisms]
	An $R$-module $M$ is \textit{flat} if the functor $N \to M \otimes_R M$ is exact. In other words: tensoring with $M$ preserves exact sequences. If the functor is also faithful, we say that $M$ is \textit{faithfully flat} over $R$.
\end{definition}

