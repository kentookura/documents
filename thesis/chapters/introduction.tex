\section{A Glimpse of Topos Theory}
Topos theory arose in the 1960's as a geometric concept in the work of Grothendieck and others in algebraic geometry. They defined a topos as a category equivalent to the category of sheaves $\sh(C,J)$ on some site $(C,J)$. Sites axiomatize and generalize the notion of open covering from topology, where the objects of $C$ are the generalized open sets and $J$ is a Grothendieck topology, which will be defined later. Suffice it to say that for each topological space $X$, the poset of open subsets of $X$ considered as a category gives rise to a site. Grothendieck toposes are a special case of the more general notion of elementary toposes, which were introduced by Lawvere and Tierney. While Grothendieck's school sought to construct a general framework for doing geometry in an effort to construct a suitable cohomology theory for schemes, Lawvere and Tierney emphasize the role of the subobject classifiers for working in the internal logic of a topos. The subobject classifier $\Omega$ for some (elementary) topos $\mathcal{E}$ is a generisation of the following situation:
For a subset $S \subset X$, there is a characteristic function $\char \colon X \to \{0, 1\}$ of $S$ given by
\[ \char_S(x) =
	\begin{cases}
		\ 1,\ x \in S, \\
		\ 0,\ x \not \in S,
	\end{cases}
\]
where $0$ is interpreted as the truth value \textbf{false} and $1$ is interpreted as \textbf{true}. Each subset $S$ of $X$ can be recovered by its characteristic function by pulling back \textbf{true} along $\char_s$. This means that the subobject functor $\text{sub} \colon \Set \to \Set$ which sends a set $S$  to its powerset $\mathcal{P}(S)$ is representable. In $\Set$, the subobject classifier has two elements $\mathbf{true}$ and $\mathbf{false}$, which are the subobjects of the final object $\{*\}$ in the category $\Set$. This corresponds to the fact that the internal logic of $\Set$ is classical logic with two truth values. However, every Grothendieck topos posesses a subobject classifier $\Omega$ which in general is a far larger object. For the topos $\mathcal{E} = \sh(X)$ of sheaves on some  space $X$, the subobject classifier $\Omega$ is a sheaf on $X$which for an open $U \subset X$ returns
\[
	\Omega(U) = \{ V \mid V \text{ is an open subset of }U\}.
\]
This object allows us to construct subobjects $\{x \mid \varphi(x)\} \subset X$ for each object $X$ of $\mathcal{E}$ in much the same way as in the category of sets by using a language $\mathcal{L}$ constructed from $\mathcal{E}$. This language has a semantic interpretation in the case of Grothendieck toposes $\sh(C,J)$ which makes precise the notion of local truth. There is the \textit{forcing relation} $\Vdash$ which is defined for $\varphi(x)$ a term in a suitable type theory with a free variable of type $X$ and an object $U \to X$ over $X$ in the over-category $C/X$ . We define $U \Vdash \varphi(x)$ to hold if $U \to X$ factors through a map $U \to \{x \mid \varphi(x)\}$ as indicated in the diagram below:

\[
	% https://q.uiver.app/?q=WzAsNSxbMCwxLCJVIl0sWzEsMSwiWCJdLFsxLDAsIlxce3ggXFxtaWQgXFx2YXJwaGkoeClcXH0iXSxbMiwwLCJcXG1hdGhiZnsxfSJdLFsyLDEsIlxcT21lZ2EiXSxbMCwxXSxbMCwyLCIiLDIseyJzdHlsZSI6eyJib2R5Ijp7Im5hbWUiOiJkYXNoZWQifX19XSxbMiwzXSxbMyw0LCJcXHRleHR7dHJ1ZX0iLDJdLFsyLDFdLFsxLDRdXQ==
	\begin{tikzcd}
		& {\{x \mid \varphi(x)\}} & {\mathbf{1}} \\
		U & X & \Omega
		\arrow[from=2-1, to=2-2]
		\arrow[dashed, from=2-1, to=1-2]
		\arrow[from=1-2, to=1-3]
		\arrow["{\text{true}}"', from=1-3, to=2-3]
		\arrow[from=1-2, to=2-2]
		\arrow[from=2-2, to=2-3]
	\end{tikzcd}
\]
Here $\{ x \mid \varphi(x)\}$ is defined as the subobject arising from the map $\varphi(x) \colon X \to \Omega$. In chapter TODO we will see how this map arises when $\varphi(x)$ is a term of the internal language $\mathcal{L}$. If the dotted map exists, interpret this as saying that $\varphi(x)$ holds locally on $U$. This leads naturally to the notion of cohomology, which measures the obstruction from constructing global solutions from local solutions.\par

We will take an in depth look at a particular example: The \'etale topos of a scheme. We first develop the theory of \'etale maps and their Galois theory. This theory generalises classical Galois theory from fields to schemes and will allow us to define the fundamental group $\pi_1^{\et}(X)$ of a connected scheme $X$. Then we introduce the notion of site and define sheaves on sites, directly generalizing sheaves on topological spaces. We then take a closer look at sheaves on the \'etale site of a scheme.\par

There are multiple good reasons to consider \'etale morphisms. One is that \'etale maps behave much like local homeomorphisms in topology and yield a good covering theory, which allows us to define fundamental groups of schemes. Another reason is that in a certain sense the \'etale topology is much finer that the Zariski topology, allowing us to trivialize bundles and coverings over \'etale open covers $\coprod U_i \to U$.

\section{Prerequisites}
We assume aqcuaintance with the basics of category theory and algebraic geometry including limits adjoint functors, representables, flat morphisms and locally free sheaves. This text contains a brief reminder of the Yoneda lemma. We will make free use of results from commutative algebra but will provide definitions and references wherever needed.

\section{Reminder on Schemes and the Zariski Topology}
The basic building blocks of algebraic geometry are affine schemes. An affine scheme $\Spec(A)$ is a geometric object constructed out of the prime spectrum
\[
	\Spec(A) = \{p \subseteq A \mid p \text{ prime ideal}\}
\]
of a commutative ring $A$. We would like to interpret the ring $A$ as the ring of functions on $\Spec(A)$, analogous to the ring
$\Sh{O}(M) := \{f: M \to \R \mid f \text{ continuous}\}$
for a manifold $M$. As a notational trick we write $f(p)$ for the image of $f$ under the quotient map $A \to A/p$. Note that the domain of a function $f$ depends on where it is evalued! The function $f \in A$ sends a point $p \in \Spec(A)$ to $f(p) = \overline{p} \in A/p$.  We will often consider schemes over a field, which are schemes equipped with a morphism $X \to \Spec(k)$, $k$ a field. This field will then play the role of $\R$ as in the example of manifolds. We equip $\Spec(A)$ with the Zariski topology, which has a basis given by sets of the form
\begin{align*}
	D_f & = \{ p \in \Spec(A) \mid (f) \not \subseteq p \} \\
	    & = \{ p \in \Spec(A) \mid f(p) \neq 0\}.
\end{align*}
\begin{definition}
	The \textit{Zariski topology} on $\Spec(A)$ consists of open sets of the form $D_I = \{ p \in \Spec(A) \mid I \not \subset (p)\}$ where $I \subseteq A$ is an ideal. The closed sets are of the form
	\begin{align*}
		V_I & = \{ p \in \Spec(A) \mid I \subseteq p \}              \\
		    & = \{ p \in \Spec(A) \mid f(p) = 0\ \forall f \in I \},
	\end{align*}
	the \textit{vanishing set} or \textit{zero locus} of $f$.
\end{definition}

Since $f(x) \neq 0$ for all $x \in D_f$, we view the ring $A[1/f]$ as the ring of rational functions defined on $D_f$. It consists of elements of the form
\[
	\Biggl\{ \frac{g}{f^k} \mid g \in A,\ k \in \N \Biggr\}.
\]
The assingment $D_f \to A[1/f]$ extends to a sheaf $\Sh{O}_{\Spec(A)}: \Spec(A) \to \mathsf{Ring}$, called the \textit{structure sheaf of $\Spec(A)$}. This sheaf has the property that the ring of global sections, denoted by $\Gamma(\Spec(A), \mathcal{O}_{\Spec(A)})$ is isomorphic to $A$. We obtain a pair of functors
\begin{gather*}
	\Spec \colon  \text{CRing}^{op}  \to \text{LRS} \\
	A \to \Spec(A)
\end{gather*}
from the category of commutative rings to the category of locally ringed spaces. An affine schemes is a locally ringed space isomorphic to $\Spec(R)$ for some ring $R$. A scheme is a locally ringed space with a covering by affine schemes. The functor seding an affine scheme $\Spec(A)$ to the ring of global sections of the structure sheaf $\Sh{O}_{\Spec(A)}$ is fully faithful and induces a bijection

\[
	\Hom_{\text{CRing}}(R, \Gamma(Y,\Sh{O}_Y)) \cong \Hom_{\text{Sch}}(Y, \Spec(R))
\]

Basic examples of affine schemes arise from polynomial rings: scheme-theoretically, the zero locus of a polynomial $f \in k[x_1, \dots, x_n]$ is given by the affine scheme $\Spec(k[x_1, \dots, x_n]/(f))$, for instance the parabola over $k$ is $\Spec(k[x,y]/(x^2-y))$.

\begin{definition}
	A scheme is a locally ringed space $X$ such that there is a covering $\{U_i\}_{i \in I}$ of $X$ such that each $U_i$ is isomorphic to an affine scheme $\Spec(A_i)$ for some ring $A_i$.
\end{definition}

%Now the $\Spec$ functor maps rings to geometric objects, and there is a related functor which sends a module $M$ over a ring $R$ to a sheaf $\widetilde{M}$ on $\Spec(R)$. A quasicoherent sheaf $F$ is defined to be one for which there is a covering $\{U_i\}$ of $R$ by affine schemes $U_i = \Spec(R_i)$ such that the restriction of $F$ to each $U_i$ is isomorphic to $\widetilde{M_i}$ for some $R_i$-module $M_i$
%
%Suppose $X = \Spec(A)$ is an affine scheme. If $\mathfrak{p} \subset \mathfrak{q}$ are two prime ideals of $A$, one contained in the other, then the neighborhood filter of $\mathfrak{p}$ is contained in the neighborhood filter of $\mathfrak{q}$. This induces a binary relation and in fact a partial order on the points of $X$. We say that $x$ \textit{is a specialization of} $y$ if $x \in \overline{y}$.
%
%\begin{example}
%	Let $A$ be a valuation ring of a field $K$, which means that for all $x \in K$ one has $x \in A$ of $x^{-1} \in A$. Then the ideals of $A$ are totally ordered by inclusion: We have a unique maximal ideal which consists of those $x \in K$ for which $v(x) \ge 0$, where , which corresponds to a uniqe closed point in $\Spec(A)$
%
%	\begin{lemma}
%		Let $R$ be a valuation ring of with fraction field $K$. The ideals of $R$ are totally ordered by inclusion.
%	\end{lemma}
%
%	\begin{proof}
%		Let $I$ and $J$ be two ideals of $A$ with $I$ not contained in $J$, pick $a \in I \setminus J$. If $b \in J$, we must show that $b \in I$. If $b = 0$ we are done. Now let $b \neq 0$. We have $b/a \in R$, because otherwise $a/b$ in $R$ which would imply $a = (a/b)b \in J$, which is a contradiction. Therefore $b = (b/a)a \in I$.
%	\end{proof}
%\end{example}
%
%We view elements of a ring $R$ as functions defined on $\Spec(R)$. If we have two subsets $A \subset B$ of $R$, then $V(B) \subset V(A)$, so $V(B)$ is a \textit{special subset of $V(A)$}, satisfying more conditions than $V(B)$. The specialisation order is trivial for Hausdorff spaces.
