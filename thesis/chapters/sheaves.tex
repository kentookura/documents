In geometry we are often faced with the situation that we can solve a certain problem locally and would like to construct a global solution. Consider the following lifting problem (say, in the category of topological spaces):
\[
	\begin{tikzcd}
		& Y\\
		X & Z
		\arrow["{?}", dashed, from=2-1, to=1-2]
		\arrow[from=1-2, to=2-2]
		\arrow[from=2-1, to=2-2]
	\end{tikzcd}
\]
We are looking for a morphism $f \colon X \to Y$ such that the diagram commutes. Now it might be the case that we can lift an open set $U$ such that the diagram

\[
	\begin{tikzcd}
		& Y\\
		U & Z
		\arrow["f",from=2-1, to=1-2]
		\arrow[from=1-2, to=2-2]
		\arrow[from=2-1, to=2-2]
	\end{tikzcd}
\]

commutes. Furthermore, as we are considering continuous functions, if we can find lifts $\varphi_1  \colon U_1 \to Y$ and $\varphi_2 \colon U_2 \to Y$ such that $\varphi_1(U_1 \cap U_2) = \varphi_2(U_1 \cap U_2)$, there is a unique lift $\varphi \colon U_1 \cup U_2 \to Y$. However, given an open cover $\{U_i\}$ of $X$ it might be impossible to find morphisms $\varphi_i$ which are compatible on all intersections, so there might be no global solutions at all! Sheaf cohomology was designed to analyse the global sections of a sheaf using techniques from homological algebra. More precisely, it measures the obstruction to finding global sections of sheaves. The search for a suitable cohomology theory for (sheaves on) schemes led mathematicians in the 1960's to the notion of the \'etale site, a generalised topology on schemes. At the same time, Lawvere was interested in formulating mathematical foundations in purely categorical language and noticed that the category of sheaves on a space (or more generally a site) admits the basic operations of set theory. For example, set comprehension $\{x \in X \mid \varphi(x)\}$ and formation of powersets $P(X)$ have natural analogues in categories of sheaves. However, the internal logic of a sheaf topos $\sh(C,J)$ is no longer classical, two-valued logic. Instead, there is a whole sheaf of truth values $\Omega$ called the subobject classifier. The basic idea is that instead of a proposition $\varphi(x)$ simply being true or false, we think of truth as a local notion. For example, consider the open set $\text{Antarctica} \subset \text{Earth}$. Then the proposition $\text{``It~is~cold''}$ might be true on $\text{Antarctica}$, so
\[
	\text{Antarctica} \Vdash \text{It is cold}
\]
but
\[
	\text{Earth} \Vdash \text{It is cold}
\]
is probably not the case. The systematic use of the internal language of a sheaf topos allows us to treat classical objects such as sheaves of $\mathcal{O}_X$-modules simply as (internal) modules. The structure sheaf of a scheme becomes an internal local ring. For a lengthy and very readable exposition on these methods see Ingo Blechschmidt's PhD thesis \cite{Blechschmidt}.

A common operation in geometry is \textit{trivialisation}. Fiber bundles are a common example of this. For example, the M\"obius strip $M$ viewed as a space over the circle $S^1$ locally has the structure of a product space, but globally is ``twisted''. In categorical approaches to geometry, we often have a collection of ``affine pieces'' or ``local models''which we glue together to obtain a class of geometric objects. An intuitive example is the case of manifolds. A connected manifold of dimension $n$ may be constructed by gluing open balls $B_i \subset \R^n$ together. For example, the circle can be constructed by gluing two open intervals $(0,1)$ together with a small overlap on the ends. The approach to algebraic geometry using locally ringed spaces takes as affine pieces the spaces of the form $\Spec(A)$, where $A$ is a commutative ring. Furthermore, there is a functor mapping $A$-modules to sheaves on $\Spec(A)$, which induces an equivalence of categories between the category of $A$-modules and the category of quasi-coherent $\mathcal{O}_X$-modules. We have the following theorem:

\begin{theorem}
	Let $X$ be an affine scheme. For every quasi-coherent $\mathcal{O}_X$-module $F$ we have $H^r(X,F)=0$ for all $r \ge 0$.
\end{theorem}

\begin{proof}
	See \cite{EGA3} 1.3.1.
\end{proof}

In fact, for a noetherian scheme $X$, being affine and having vanishing cohomology $H^i(X,F)$ for all quasi-coherent $F$ and $i \ge 0$ are equivalent. Now a general scheme $X$ admits a covering by open affine subschemes $U_i = \Spec(A_i)$, and the category of quasicoherent sheaves on $X$ is also ``glued'' from quasicoherent sheaves in a suitable sense, but these sheaves then no longer necessarily have vanishing cohomology. Much of algebraic geometry is formulated in terms of quasicoherent sheaves and their cohomology. The Zariski topology is suited to study coherent cohomology almost by definition. The theory of Grothendieck toposes and sites was originally developed in search of a cohomology theory for varieties over finite fields from which the Weil conjectures could be deduced. In this aspect \'etale cohomology has been an undeniable success. The $\ell$-adic cohomology is a Weil cohomology theory for varieties over $\F_q$, where $\ell \neq \text{char}(\F_q)$, which is constructed out of the \'etale cohomology of the constant sheaves $\Z/\ell^k \Z$. To obtain a Weil cohomology theory, one defines

\[
	H^i(X,\Q_\ell) \coloneqq \varprojlim H^i(X, \Z/\ell^k \Z) \otimes_{\Z_\ell} \Q_\ell,
\]
where $\Q_\ell$ is the quotient field of $\Z_\ell$. We list some of the properties that are fulfilled by this cohomology theory:

\begin{itemize}
	\item $H^*(-, \Q_\ell) \colon \{ \text{ smooth projective variteis over } k \} \to \{\text{ $\Q_\ell$-vector spaces}\}$
	      is a contraviarant functor.
	\item $H^i(X, \Q_\ell) = 0$ for $i<0$ and $i>2n$.
	\item There is a linear trace map $\Tr \colon H^{2n}(X, \Q_\ell) \to \Q_\ell$
	\item For all $i,j$ there is a cup product
	      \[
		      H^i(X, \Q_\ell) \times H^j(X, \Q_\ell) \to H^{i+j}(X, \Q_\ell)
	      \]
	\item If $X$ is smooth and proper over $\F_q$ and of dimension $n$, then $H^{2n}(X, \Q_\ell) \simeq \Q_\ell$ and the cup product
	      \[ H^i(X, \Q_\ell) \times H^{2n-1}(X, \Q_\ell) \to H^{2n}(X, \Q_\ell)
	      \]
	      is a perfect pairing for $0 \le i \le 2n$.
\end{itemize}

Singular cohomology is a Weil cohomology theory for Varieties over $\C$ if we consider them using the analytic topology. This cohomology theory provides a rigorous way for counting ``holes'' in a topological space. For instance, the circle $S^1$ has 1 one-dimensional hole while the torus $T^2$ has 2. The sphere $S^2$ has 1 two-dimensional hole but no one-dimensional holes, in symbols\footnote{$\Z$ appears here because it is the free group on one generator.}

\begin{align*}
	H^1_{sing}(S^1) \simeq \Z,\quad & H^1_{sing}(T^2) \simeq \Z \oplus \Z, \\
	H^1_{sing}(S^2) \simeq  0,\quad & H^2_{sing}(S^2) \simeq \Z.
\end{align*}

\begin{proposition}
	$H^i(X, \Sh{F})$ is 0 for any constant sheaf on an irreducible topological space.
\end{proposition}

\begin{proof}
	Because every open susbset $U$ of an irreducible topological space is connected, the constant sheaf $F$ defined by a group $G$ takes talue $F(U) = G$ for each nonempty $U$, so $F$ is flasque, hence $H^i(X,F) = 0$ for $r>0$.
\end{proof}

This is a strong indication that the traditional tools of cohomology are inadequate to analyse the geometry of schemes.

\section{Sites}
We have seen strong analogies between \'etale maps and open sets. Specifically, \'etale maps allow us to trivialize certain geometric constructions as if we were working in a finer topology than the Zariski topology. Furthermore, the theory of surjective \'etale maps should be interpreted as the theory of covering spaces for schemes. We will now develop some more topological analogies, equipping the category $\Et/X$ with a Grothendieck topology. The definition of Grothendieck topology is based on the observation that the definition sheaves makes no reference to points, only the (category of) open sets of $X$. Even so, one does not need the structure of the topology, just the notion of open covering. Because there is no natural object in $\Op(X)$ describing a cover $\{U_i\}$, one is lead to consider the free cocompletion $\yo \colon C \to \widehat{C}$ because it contains the coproduct $\coprod \yo(U_i)$.

Let $S \subset \yo(A)$ be a subfunctor and $f \colon B \to A$ a morphism. We can construct a subfunctor $f^*S \subset \yo(B)$ by defining
\[
	f^*S = S \times_{\yo(A)} \yo(B),
\]
which, when applied to an object $X$ gives
\[
	(f^*S)(X) = \{g \in \Hom(X,B) \mid f \circ g \in S(X)\}.
\]
This is a subsheaf of $\yo(B)$ because pullbacks of monics are monics. We have
\[
	fg \in S(C) \iff g \in (f^*S)(C)
\]
for all composable morphisms $f \colon  B \to A$ and $g \colon C \to B$. Setting $g = id_B$ we have
\[
	f \in S(B) \iff f^*S(B) = \Hom(B,A).
\]
Define a relation ``$S$ covers $f$'' whenever $S \subset \yo(U)$ is a subfunctor and $f \colon V \to U$ is a morphism with $f^*S \in J(V)$. Thus $S$ covers $f \circ g \iff f^*S$ covers $g$.

\begin{definition}[Grothendieck Topologies]
	A Grothendieck topology on a category $C$ is a function $J$ which assings to each object $U$ of $C$ a collection of subfunctors $J(A)$ of $\yo(A)$ subject to the following conditions:
	\begin{itemize}
		\item For every object we have $\yo(A) \in J(A)$.
		\item
		      Given a subfunctor $S \subset \yo(A)$ and a morphism $f \colon B \to A$ such that the diagram
		      \[
			      \begin{tikzcd}
				      f^*S       & S \\
				      {\yo(B)} & {\yo(A)}
				      \arrow["f_S"    ,       from=1-1, to=1-2]
				      \arrow["r"      , tail, from=1-2, to=2-2]
				      \arrow["r_f"    , tail, from=1-1, to=2-1]
				      \arrow["\yo(f)"',       from=2-1, to=2-2]
			      \end{tikzcd},
		      \]
		      is a pullback, we have $R_f \in J(B)$.
		\item Given an arbitrary subfunctor $R \subset \yo(A)$ and a subfunctor $S \in J(A)$ of $\yo(A)$ such that $f^*R \in J(B)$ for all $f \colon B \to A$, then $R \in J(A)$.
	\end{itemize}

	There is an ``elementwise'' formulation of these conditions which resembles the topological picture more:

	\begin{definition}[Grothendieck Pretopologies]
		Let $C$ be a category. A Grothendieck pretopology or a basis for a topology on $C$ is given by a function $\Cov$ which assigns to each object $U$ of $C$ a collection of families of morphisms $\Cov(U)$ with codomain $U$ with the following properties:
		\begin{enumerate}
			\item If $V \to U$ is an isomorphism, then $\{V \to U\} \in C$
			\item If $\{U_i \to U\} \in \Cov(C)$, and $V \to U$ is a morphism, then $U_i \times_U V$ exists and $\{U_i \times_U V \to V\}$ is in $\Cov(C)$.
			\item If $\{U_i \to U\} \in \Cov(C)$, and for each $U_i$ we have $\{V_{ij} \to U_i\}$, the composition $\{V_{ij} \to U\}$ is in $\Cov(C)$.
		\end{enumerate}
	\end{definition}
\end{definition}


Let $K$ be a Grothendieck pretopology on $C$. Then $K$ generates a Grothendieck topology by defining for each $A$ a set $J(A)$ consisting of those subfunctors of $\yo(A)$ which contain a covering family $\{U_i \to U\}_{i \in I}$. Note that for a topology $J$ on $C$ there is a maximal basis which generates $J$, see~\cite{SIGL}, III.2.2. While Grothendieck topologies are more ``functorial'', the pretopology viewpoint is geometrically intuitive. We will make use of both, depending on the situation.

\begin{example}
	A topological space $X$ may be equipped with the structure of a pretopology by declaring the coverings of open sets $U \subseteq X$ to be those families of open sets $\{U_i\}$ contained in $U$ such that $\displaystyle\bigcup U_i = U$. The pullback $U_i \times_U U_j$ is then simply the intersection $U_i \cap U_j$. Recall that a subfunctor $S \subset \Hom(-,A)$ may be described by a set $\{V_i\}$ of subsets of $A$ such that $S(V) \simeq \{*\}$. Thus $S$  is then a cover if $\bigcup V_i = A$.
	%Let $\mathcal{O}(X)$ be the corresponding locale. A family of morphisms $\{U_i \to U\}$ is a cover if $U = \bigvee U_i$
\end{example}

\begin{example}[The Atomic Topology]
	Let $C$ be a category and define $\text{At}(A)$ to consist of all nonempty subfunctors of $\yo(A)$ for all objects $A$ of $C$. In order for this to define a topology on $C$, it must be possible to complete any two morphisms $f \colon  B \to A$ and $g \colon B' \to A$ to a commutative square
	\[
		\begin{tikzcd}
			X & B \\
			B' & A
			\arrow[from=1-1, to=2-1]
			\arrow[from=1-1, to=1-2]
			\arrow[from=1-2, to=2-2]
			\arrow[from=2-1, to=2-2]
		\end{tikzcd}
	\]
	for some object $X$. In particular, the atomic topology is well defined for any category with pullbacks.
\end{example}

\begin{example}[The \'Etale Site of a Scheme]
	The main site we will consider is the \'etale site of a scheme $X$. The underlying category is the subcategory of schemes over $X$ which are finite \'etale over $X$. From \ref{lemma:etale_site} it follows that the set of families jointly surjective \'etale morphisms form a site. The \'etale topology topology is finer than the Zariski topology in the sense that an open immersions $U \to X$ are \'etale, so each open set $U \subset X$ may be thought of as an \'etale open. From now on we write $\Et/X$ for this site, omitting the topology from the notation.
\end{example}

\begin{example}[The \'Etale Site of $\Spec(k)$]
	Let $A$ be an finite \'etale $k$-algebra. Recall that $\Spec(A)$ is isomorphic to a disjoint union of points $\coprod \Spec(L_i)$ where the $L_i$ are finite separable extensions of $k$. Then an \'etale cover of $A$ is given by choosing a finite \'etale algebra $A_i$ for each $L_i$ and taking spectra:
	\[
		\coprod \Spec(A_i) \to \coprod \Spec(L_i) \simeq \Spec(A).
	\]
	Let $L$ be a finite separable extension of $k$ A sieve on $\Spec(L)$ is a covering if it is nonempty. More generally, a sieve on $\Spec(A)$ is a covering if it is nonempty, hence the \'etale topology on $\Et/\Spec(k)$ coincides with the atomic topology.
\end{example}

\subsection{Presheaves}
Recall that a presheaf $P$ of sets on a category $C$ is simply a functor $P \colon C^{\op} \to \Set$. For an object $A$ of $C$, we call the elements $s \in P(A)$ the sections of $P$ over $A$. The functor category $\Set^{C^{\op}} = \widehat{C}$ has as objects presheaves $P \colon C^{\op} \to \Set$ and as morphisms natrual transformations $P \to P'$. For each object $X$ of $C$ there is a presheaf $\yo(X) = \Hom_C(-,X)$, which is defined on an object by $\yo(X)(Y) = \Hom_C(Y,X)$ and for morphisms $f \colon Y' \to Y$ and $u \in \Hom_C(Y,X)$ by
\begin{gather*}
	\yo(X)(f) \colon \Hom_C(Y,X) \to \Hom_C(Y', X)\\
	\yo(X)(f)(u) = u \circ f
\end{gather*}
A presheaf isomorphic to one of the form $\yo(A)$ is called representable, with $A$ being the representing o bject. The Yoneda lemma implies that this object is unique up to isomprhism. Recall that the Yoneda lemma says that there is a bijectivetion between natural transformations $\yo(A) \to P$ and elements of $P(A)$, where $P$ is an arbitrary presheaf. If $f \colon A \to B$ is a morphism in $C$, then there is a natural transformation $\yo(A) \to \yo(B)$ given by composing with $f$. We obtain a fully faithful functor
\[\yo \colon C \to \Set^{C^{\op}},\quad A \to \Hom_C(-,A)\]
called the yoneda embedding.

\begin{construction}[Subobjects]
	Recall that a morphism $f \colon A \to B$ in some category $C$ is called a monomorphism if for any object $C$ and parallel arrows $g, h \colon C \xbigtoto{} A$ with $fg = fh$ implies $g = h$. We consider two monomorphisms $f \colon A \to B$ and $f' \colon A' \to B$ to be equivalent if there is an isomophism $A \simeq A'$ with $f'h = f$. A subobject of an object $A$ of $C$ is an equivalence class of monomorphisms with codomain $A$. The collection of subobjects of $A$, denoted by $\mathbf{Sub}_C(A)$ is partially ordered, where $[f] \le [g]$ if and only if there is a morphism $k \colon S \to T$ such that $f = gk$, where $[f]$ and $[g]$ are the classes of $f$ and $g$.
\end{construction}

\begin{example}[Subfunctors]
	Let $P$ be a presheaf on $C$. A subfunctor $S$ of $P$ is defined to be a presheaf such that for each object $A$, the set $S(A)$ is a subset of $P(A)$ and each map $S(f): S(B) \to S(A)$ is a restriction of $P(f)$ for all morphisms $f \colon A \to B$.
\end{example}

\begin{definition}
	In a category $C$ with finite limits a \textit{subobject classifier} is a monic $\true \colon 1 \to \Omega$ such that for every monic $S \to X$ there is a unique morphism $\phi$ such that diagram below is a pullback square.
	\[
		\begin{tikzcd}
			S & 1 \\
			A & \Omega
			\arrow[tail, from=1-1, to=2-1]
			\arrow[from=1-1, to=1-2]
			\arrow["{\true}", from=1-2, to=2-2]
			\arrow["\phi"', from=2-1, to=2-2]
		\end{tikzcd}
	\]
\end{definition}

\begin{proposition}
	In a category with a subobject classifier, the object $\Omega$ is a representing object for the functor $A \to \mathbf{sub}_C(A)$ and we have an isomoprhism
	\[ \mathbf{sub}_C(A) \simeq \Hom_C(A,\Omega).\]
	Furthermore, the object $\mathbf{1}$ is the final object in $C$.
\end{proposition}

\begin{proof}
	See \cite{SIGL}, I.3.1.
\end{proof}

\subsection{Subobjects in Presheaf Categories}\label{section:presheaf_subobjects}

\begin{construction}
	Let $\widehat{C}$ be a presheaf category. Suppose $\widehat{C}$ has a subobject classifier $\Omega$. The subobjects of a representable $\yo(A)$ must be given by
	\[
		\sub_{\widehat{C}}(\yo(A)) \simeq \Hom_{\widehat{C}}(\yo(A), \Omega) \simeq \Omega(A)
	\]
	Thus the subobject classifier of $\widehat{C}$ is given by the presheaf $A \to \Omega(A)$ where we have the chain of equalities:
	\[
		\Omega(A)  = \sub_{\widehat{C}}(\Hom_C(-,A))  = \{S \mid S \text{ is a subfunctor of $\yo(A)$}\}.
	\]
\end{construction}

%There is an alternative description of subfunctors which we will make use of in sheaf theory.

%\begin{definition}[Sieves]
%	Let $C$ be a category and $A$ an object of $C$. A \textit{sieve} on $A$ is a set $S$ of morphisms with codomain $A$ such that $fh \in S$ whenever $f \in S$ and $fh$ is defined.
%\end{definition}
%
%\begin{remark}
%	If $S$ is a subfunctor $\yo(U)$ and $f \colon V \to U$ is a morphism, then the set
%	\[
%		f^*(S) = \{ g \mid g \text{ has codomain } V \text{ and the composite }fg \text{ is in } S\}
%	\]
%	is a defines a subfunctor of $\yo(V)$.
%\end{remark}
%
\begin{example}
	Let $\Op(X)$ be the poset of opens of a topological space $X$. Consider the representable functor $\yo(U) = \Hom(-,U)$. If $V \subset U$, then $\Hom(V,U)$ is the one element set, as $\Op(X)$ is a poset. If $V \not\subset U$, then $\Hom(V,U) = \varnothing$. Now let $S$ be a subfunctor of $\Hom(-,U)$. Then $S$ may be equivalently described as the set of all those $V \subset U$ with $S(V) = 1$ or as the subset $T \subset \Op(U)$ of those open sets such that $W \subset V \in T$ implies $W \in S$.
\end{example}

Every sieve $S$ on $A$ determines a subfunctor of $\yo(A)$ and conversely, every subfunctor of $\yo(A)$ determins a sieve on $A$. Given a subfunctor $T \subset \yo(A)$, the set
\[
	S = \{f \mid f \in S(B) \text{ for some object } B\}
\]
is a sieve on $A$. Conversely a sieve $S$ on $A$ determines a subfunctor $T$  of $\yo(A)$ by defining
\[
	T(B) = \{ f \mid f \in S \text{ and } f \text{ has domain } B\} \subset \yo(A)(B).
\]
Since these associations are inverses to each other, we can equivalently define the subobject classifier of a presheaf category $\Set^{C^{\op}}$ as the presheaf which to an object $A$ the set of sieves on $A$.

\begin{remark}
	Returning to the example of the category of opens $\Op(X)$, we can define a covering sieve for $U$ to be a sieve $S$ on $U$ such that $U$ is the union of all the open sets $V \in S$.
\end{remark}

\begin{definition}[Sheaves on Topological Spaces]\label{def:sheaves}
	Let $X$ be a topological space. A presheaf $F$ on $X$ is a \textit{sheaf}, if for any open covering $\{U_i\}$ of $U$ and any family of section $x_i \in F(U_i)$ such that $x_i|_{U_i \cap U_j} = x_j|_{U_i \cap U_j}$ there is a unique $x \in U = \bigcup U_i$ such that $x|_{U_i} = x_i$. A subsheaf of $F$ is simply a subfunctor of $F$ which is also a sheaf.
\end{definition}

\begin{proposition}\label{prop:classifier}
	The subobject classifier $\Omega$ for the category of sheaves $\sh(X)$ on a space $X$ is given on objects $U \subset X$ by
	\[
		\Omega(U) = \{ V \mid V \text{ is an open subset of } U \}
	\]
	and on morphisms $V \subset U$ by
	\[
		\Omega(U) \to \Omega(V),\quad W \to W \cap V.
	\]
\end{proposition}

\begin{proof}
	We first show that $\Omega$ is indeed a sheaf. Let $\{U_i\}$ be an open cover of $U$ and let $V_i \in \Omega(U_i)$ be sections with $V_i \cap U_j = V_j \cap U_i \subset U_i \cap U_j$. Then the union of the $V_i$ is a section of $\Omega$ over $U$ with the property that $V \cap U_i = V_i$. To show that $\Omega$ is a subobject classifier for $\sh(X)$, consider a subobject $S \subset F$. By the definition of subobject classifier, this subobject should arise from a characteristic natural transformation $\phi \colon F \to \Omega$. Define this function elementwise by the function
	\begin{gather*}
		\phi \colon F(U) \to \Omega(U)\\
		\quad x \to \bigcup W_i
	\end{gather*}
	where the $W_i$ are the open subsets of $U$ such that the restriction of $x$ to $W_i$ lies inside of $S(W_i)$. This map is natural in $U$ and because $S$ is a subsheaf, $x|_W$ lies in $S(W)$. Now for each $U$  we can pull back the map $\phi_U$ along the map $\true \colon 1 \to \Omega(U)$ which picks out $U \in \Omega(U)$:
	\[
		\begin{tikzcd}
			P(U) & 1 \\
			F(U) & \Omega(U)
			\arrow[from=1-1, to=2-1]
			\arrow[from=1-1, to=1-2]
			\arrow["\true", from=1-2, to=2-2]
			\arrow["\phi_U"', from=2-1, to=2-2].
		\end{tikzcd}
	\]
	The constructed pullback $P(U)$ is equal to $\{x \in F(U) \mid \phi_U x = U\} \subset F(U)$, and since limits and in particular pullbacks are constructed pointwise in presheaves the ``total'' pullback
	\[
		\begin{tikzcd}
			P & 1 \\
			F & \Omega
			\arrow[from=1-1, to=2-1]
			\arrow[from=1-1, to=1-2]
			\arrow["{\true}", from=1-2, to=2-2]
			\arrow["\phi"', from=2-1, to=2-2]
		\end{tikzcd}
	\]
	is indeed the subsheaf $S$. The map $\phi$ is also unique, so the map $\true: 1 \to \Omega$ is a subobject classifier.
\end{proof}

\begin{definition}[Sheaves on Sites]
	Let $(C,J)$ be a site. A presheaf $F$ on $C$ is a \textit{sheaf}, if for any covering $\{U_i\}$ of $U$, the following diagram is an equalizer:
	\begin{equation} \label{diagram:equalizer}
		F(U) \to \prod_{i} F(U_i) \xbigtoto{} \prod_{i,j} F(U_i \times_U U_j)
	\end{equation}
	If the map $F(U) \to \displaystyle\prod F(U_i)$ is injective, then we say that $F$ is a \textit{separated presheaf}. If the reference to the site $J$ is clear from context, we will also say that $F$ is a sheaf on $C$.
\end{definition}

A map of sheaves $\varphi \colon F \to G$ is simply a natural transformation $F \to G$.
We obtain the category $\sh(C,J)$ of sheaves on $C$ with respect to the topology $J$. This is in fact a full subcategory of presheaves $\Set^{C^{op}}$. We call categories equivalent to categories of sheaves on a site Grothendieck toposes. The category of sheaves $\sh(X, \et_X)$ on the \'etale toplogy of a scheme $X$ is called the \textit{\'etale topos of $X$}.

\section{Sheafification}
In this section we will see that Grothendieck toposes are left exact reflective subcategories $\mathcal{E} \to \widehat{C}$ of presheaf categories. More precisely, given a small category $C$ there is a bijection between
\begin{itemize}
	\item Equivalence classes of left exact reflective subcategories $\mathcal{E} \to \widehat{C}$
	\item Grothendieck topologies $J$ on $C$ such that $\mathcal{E} \simeq \sh(C,J)$.
\end{itemize}

Recall that a full subcategory $i \colon D \to C$ reflective if the inclusion functor $i$ admits a left adjoint $a \colon C \to D$. We already know that $\sh(C,J)$ is a full subcategory of $\widehat{C}$.

\begin{proposition}\label{thm:associated_sheaf}
	The forgetful functor $\mathsf{Sh}(C,J) \to \Set^{C^{op}}$ admits a limit-preserving left adjoint
	\[a : \Set^{C^{op}} \to \mathsf{Sh}(C,J).\]
\end{proposition}

\begin{proof}
	See~\cite{SIGL}.
\end{proof}

\begin{corollary}
	The cateogry of sheaves on $X$ is a full reflective subcategory of the category of presheaves on $X$.
\end{corollary}

The functor $a$ is the \textit{associated sheaf functor}, also called \textit{sheafification}. As the name suggests, it provides universal way to turn a presheaf into a sheaf.
Note that there are two ways a presheaf can fail to be a sheaf:
\begin{itemize}
	\item
	      Local sections may fail to patch to a global section. An example of a presheaf whose local sections fail to patch is the presheaf of bounded continuous functions on $\R$. If we cover $\R$ by bounded intervals $\{U_i\}$, the identity function is bounded on each $U_i$ but is obviously not globally bounded.
	\item
	      Sections that agree locally may not agree globally. We construct an example. $S$ be the a discrete topological space with two points $0$ and $1$. We define a presheaf $\Sh{F}$ by setting $\Sh{F}(\varnothing) = \{*\}$ and $\Sh{F}(U) = \R^U$. We define the restriction maps to send a section $s \in \Sh{F}(U)$ to the constant function $s|_V \equiv 0$ when $V \subseteq U$ is nonempty. Now let $s,t \in \Sh{F}(S)$ be defined by $s \equiv 1$ and $t \equiv -1$. Then $\{\{0\}, \{1\}\}$ is an open cover of $S$ and $s|_{0} = t|_{0}$ and $s|_{1} = t|_{1}$, but $s \neq t$.
	      This cannot happen if the map
	      \[
		      i: \Sh{F}(U) \to \mathsf{eq}(\prod_{i} \Sh{F}(U_i) \xbigtoto{} \prod_{i,j} \Sh{F}(U_i \cap U_j))
	      \]
	      is injective. A presheaf for which $i$ is injective is called separated.
\end{itemize}

Thus, sheafification proceeds in two steps. The first step removes sections that agree locally but not globally, making a presheaf into a separated presheaf. The second step adds the sections of matching local sections.

\begin{theorem}
	A category is a Grothendieck topos if and only if it is a localization of $\widehat{C}$ for some small category $C$.
\end{theorem}

\begin{proof}
	See~\cite{Borceux3}, Proposition 3.5.4 and Corollary 3.5.5.
\end{proof}

\section{Sheaves on the \'Etale Site}
In this section we will now prove the important fact that all representable presheaves are sheaves in the \'etale topology. This gives us a rich supply of sheaves on $\Et/X$ to study.

\begin{lemma}\label{sheaf:disjoint}
	Let $F$ be a sheaf on the Zariski topology on a scheme $X = \coprod U_i$ which is the disjoint union of subschemes $U_i$. Then $F(X) = \prod F(U_i)$.
\end{lemma}

\begin{proof}
	The pullback $U_i \times_X U_j$ is empty for $i \neq j$. Evaluating a sheaf on the empty set yields the terminal object $\{*\}$, so the sheaf condition becomes
	\[
		F(X) \to \prod_i F(U_i) \xbigtoto{} \prod_{i,j} \{*\} \simeq \{*\},
	\]
	which yields the required isomorphism.
\end{proof}

\begin{theorem}\label{thm:sheaf_condition}
	A presheaf $F$ on $\Et/X$ is a sheaf if and only if $F$ satisfies the sheaf condition for Zariski open coverings and for \'etale coverings consisting of a single map $V \to U$, where $V$ and $U$ are affine.
\end{theorem}

The idea of the proof is to rewrite coverings $\{ U_i \to U\}$ as a single morphism $\coprod U_i \to U$ and to use the fact that schemes are locally affine.

\begin{proof}
	Let $F$ be a presheaf satisfying the condition of the theorem. Because of the previous lemma and because of the equality
	\[
		\Bigl(\coprod U_i \Bigr) \times_U \Bigl( \coprod U_j \Bigr) = \coprod_{i,j} U_i \times_U U_j,
	\] the sheaf condition for the covering $\{U_i \to U\}$ is equivalent to the sheaf condition for the covering $\coprod_i U_i \to U$. If the indexing set $I$ is affine each $U_i$ is affine, $\coprod U_i$ is again affine and the sheaf condition holds.

	Now let $\{U_i \to U\}$ be an arbitrary cover and let $f: \mathcal{U} \to U$ the corresponding morphism from the coproduct $\mathcal{U} = \coprod U_i$. Let $\{V_i \to U\}$ be a covering of $U$ by affine open sets. Then $f^{-1}(V_i)$ is a union of open affines, say
	\[
		f^{-1}(V_i) = \bigcup_j V_{ij}
	\]
	Each $f(V_{ij})$ is open in $V_i$ and $U_i$ is quasi-compact. Therefore there is a finite set $K_i$ of indices such that $\{V_{ik} \to V_i\}_{k \in K_i}$ is a covering. We obtain a finite affine cover $\{V_{ik} \to U_i\}_{k \in K_i}$ for each $V_i$. Consider the diagram:

	\[
		% https://q.uiver.app/?q=WzAsOCxbMCwwLCJGKFUpIl0sWzEsMCwiRihVJykiXSxbMiwwLCJGKFUnIFxcdGltZXNfVSBVJykiXSxbMSwxLCJcXHByb2RfaSBcXHByb2RfaiBGKFZfe2lqfSkiXSxbMCwxLCJcXHByb2QgRihVX2kpIl0sWzAsMiwiXFxwcm9kX3tpLGp9IEYoVV9pICBcXGNhcCBVX2opIl0sWzEsMiwiXFxwcm9kX3tpLGp9XFxwcm9kX3trLGx9IEYoVl97aWt9IFxcY2FwIFZfe2psfSkiXSxbMiwxLCJcXHByb2RfaSBcXHByb2Rfe2osbH0gRihWX3tpan0gXFx0aW1lc19VIFZfe2lsfSkiXSxbMCwxXSxbMSwyLCIiLDAseyJvZmZzZXQiOjF9XSxbMSwyLCIiLDAseyJvZmZzZXQiOi0xfV0sWzEsM10sWzAsNF0sWzQsM10sWzQsNSwiIiwyLHsib2Zmc2V0IjoxfV0sWzQsNSwiIiwyLHsib2Zmc2V0IjotMX1dLFs1LDZdLFszLDYsIiIsMix7Im9mZnNldCI6LTF9XSxbMyw2LCIiLDIseyJvZmZzZXQiOjF9XSxbMyw3LCIiLDIseyJvZmZzZXQiOjF9XSxbMyw3LCIiLDIseyJvZmZzZXQiOi0xfV0sWzIsN11d
		\begin{tikzcd}
			{F(U)} & {F(\mathcal{U})} & {F(\mathcal{U} \times_U \mathcal{U})} \\
			{\displaystyle \prod F(V_i)} & {\displaystyle \prod_i \displaystyle \prod_k F(V_{ik})} & {\displaystyle \prod_i \displaystyle \prod_{k,l} F(V_{ik} \times_U V_{il})} \\
			{\displaystyle \prod_{i,j} F(V_i  \cap V_j)} & {\displaystyle \prod_{i,j} \displaystyle \prod_{k,l} F(V_{ik} \cap V_{jl})}
			\arrow[from=1-1, to=1-2]
			\arrow[shift right=1, from=1-2, to=1-3]
			\arrow[shift left=1, from=1-2, to=1-3]
			\arrow[from=1-2, to=2-2]
			\arrow[from=1-1, to=2-1]
			\arrow[from=2-1, to=2-2]
			\arrow[shift right=1, from=2-1, to=3-1]
			\arrow[shift left=1, from=2-1, to=3-1]
			\arrow[from=3-1, to=3-2]
			\arrow[shift left=1, from=2-2, to=3-2]
			\arrow[shift right=1, from=2-2, to=3-2]
			\arrow[shift right=1, from=2-2, to=2-3]
			\arrow[shift left=1, from=2-2, to=2-3]
			\arrow[from=1-3, to=2-3]
		\end{tikzcd}
	\]

	We need to verify that the top row is exact. The columns are because the $V_i$ are open subsets of $X$ and $F$ is a sheaf for the Zariski topology by assumption. The middle row is a product of exact sequences and hence exact. It follows that the map $F(U) \to F(\mathcal{U})$ is injective and that $F$ is a separated presheaf. Thus the bottom arrow is injective. It follows by a diagram chase that the top row is exact, so $F$ is a sheaf.
\end{proof}

\begin{theorem}
	Every presheaf $\widehat{Z}$ represented by an $X$-scheme $Z$ given by $U \to \Hom_X(U,Z)$ is a sheaf. We will often omit the notation and simply write $Z(U)$ instead of $\widehat{Z}(U)$.
\end{theorem}

\begin{proof}
	It is obviously a sheaf for the Zariski topology. By the previous theorem, it is sufficient to show exactness of the sequence
	\[
		\widehat{Z}(\Spec(A)) \to \widehat{Z}(\Spec(B)) \xbigtoto{} \widehat{Z}(\Spec(B \otimes_A B)).
	\]
	But if $\Spec(A) \to \Spec(B)$ is surjective and \'etale, then $B \to A$ is faithfully flat and of finite type. This implies that $B \to A$ is a strict epimorphism, which means that
	\[
		\Hom(A,Z) \to \Hom(B,Z) \xbigtoto[p_2^*]{p_1^*}\Hom(B \otimes_A B, Z)
	\]
	is exact for all $Z$ (for a proof of this fact, see \cite{milneLEC}, Theorem 2.17.) and the statement follows.
\end{proof}

\begin{corollary}
	The yoneda embedding $\yo\colon \Et/X \to \widehat{\Et/X}$ factors through the inclusion $\sh(\Et/X) \to \Set$.
\end{corollary}

%\begin{construction}[Sheaves of abelian groups on $\Spec(k)$]
%	Let $F$ be a presheaf on $\Et/\Spec(k)$. By abuse of notation we write $F(A)$ instad of $F(\Spec(A))$ for $A$ an \'etale $k$-algebra. We define a discrete $\Gal(k)$-module $M_F$ as follows: If $L/k$ is a finite separable extension, then $G = \Gal(k)$ acts on $F(L)$ by functoriality of $F$. Define $M_F = \colim F(L)$ where $L$ runs over all subfields $L$ of $k_s$ that are finite over $k$. Then $M_F$ is a discrete $G$-module. Conversely, for a discrete $G$-module $M$ we define a sheaf $F_M$ by setting $F_M(A) = \Hom_G(F(A),M)$ where $F(A) = \Hom_k(A,k_s)$. By Theorem \ref{thm:sheaf_condition}, presheaf $F$ on $\Et/k$ is a sheaf if and only if \ref{diagram:equalizer} is exact for a single affine cover $\Spec(A) \to \Spec(B)$, but affine \'etale covers in $\Et/k$ are precisely of the form $\displaystyle \coprod \Spec(L_i) \to \displaystyle \coprod \Spec(K_j)$ where each $L_i$ and $K_j$ is separable over $k$, so the sequence \[ F(\Spec(L)) \to \prod_{i} F(\Spec(L_i)) \xbigtoto{} \prod_{i,j} F(\Spec(L_i \otimes_L L_j)) \] needs to be exact.
%\end{construction}
%
%\begin{enumerate}
%	\item $F(\prod A_i) = \bigoplus F(A_i)$ for every finite family $\{A_i\}$ of \'etale algebra
%	\item $F(L) \xrightarrow{\sim} F(K)^{\Gal(K/k)}$ for finite Galois extensions $K/L/k$
%\end{enumerate}
%
%For $F$ a sheaf on $\Et/k$ define $M_F = \varinjlim F(K)$, where the colimit runs over all finite Galois extensions over $k$. Then $M_F$ is a discrete $\Gal(k)$-module. Converseley, if $M$ is a discrete $G$-module, define $F_m(A) = \Hom_{\Gal(k)}(F(A),M)$, where $F(A) = \Hom(A, k_s)$. Then $F_M$ is a sheaf on $\Spec(k)$. This defines an equivalence of categories between abelian sheaves $\Ab_{\et}(\Spec(k))$ of $k$ and the category of discrete $G$-modules.
%

% \begin{definition}
% 	The \textit{category of abelian sheaves on $\Et/X$} is defined to be the category of abelian group objects in $\sh_{\Et}(X)$
% \end{definition}
% 
% \begin{corollary}
% 	A group object internal to $\Et/X$ is a scheme $S$ over $X$ with maps $\circ \colon S \times_X S \to S$, $e \colon X \to S$ and $\iota \colon S \to S$ subject to the conditions of the previous definition.
% \end{corollary}

%\subsection{Abelian Sheaves and Group Schemes}
%
%\section{Local Rings and Stalks}
%%ROUGH
%Local rings are rings with a unique maximal ideal. A ring is local if and only if a sum of any two non-units is again a non-unit. To motivate the name, consider the following example:
%
%Let $C[-1,1]$ be the ring of continuous real valued functions on the interval $[-1,1]$ and consider the equivalence relation $\sim$ which identifies two function $f$ and $g$ if there is an open neighborhood $U$ of $0 \in [-1,1]$ such that $f_U = g_U$. The equivalence classes defined by this relation are called the germs of $C[-1,1]$ at $0$ and in fact form a ring, since we may add and multiply two germs. This ring is called the \textit{stalk} of $C[-1,1]$ at $0$. It is clear that a germ is invertible if and only if $f(0) \neq 0$ and it follows that the sum of two non-units is again a non-unit, so $C[-1,1]$ is local.
%
%More generally we can define the stalk for presheaves on an arbitrary topological space: If $F$ is a presheaf on a space $X$ and $x \in X$ is a point, we define the stalk $F_x$ of $F$ at $x$ as the colimit $\colim_{U \ni x}F(U)$ over all neighborhoods of $x$. If $X = \Spec(A)$ is an affine scheme,
%%ROUGH%
%
%We can also form stalks of sheaves on the \'etale toplogy. Let $X$ be a scheme and $x$ a point of $X$. An \'etale neighborhood of $x$ is defined to be a pair $(U, y)$ together with an \'etale morphism $\varphi\colon U \to X$ such that $\varphi(y) = x$. A morphism of \'etale neighborhoods $f\colon (V,z) \to (U,y)$ is a morphism of $X$-schemes $f\colon V \to U$ such that $f(v) = u$. The stalks of sheaves are defined as a colimit over all \'etale neighborhoods.
%
%
%We will show that for each $x \in X$, the functor $F \to F_x$ is exact. This provides us with a way to check exactness locally: An exact sequence of abelian sheaves
%\[
%	0 \to F \to G \to H \to 0
%\]
%is exact if and only if the seqence
%\[
%	0 \to F_x \to G_x \to H_x \to 0
%\]
%is exact for each $x \in X$.
%Recall that a category $\mathcal{C}$ is \textit{cofiltered} if the following holds:
%\begin{enumerate}
%	\item For any pair of objects $a_1$ and $a_2$, there is an object $b$ with maps $b \to a_1$ and $b \to a_2$.
%	\item Every pair of morphisms $f,g\colon a \to b$, there is an equalizer $h\colon c \to a$. This means that $f \circ h = g \circ h$.
%\end{enumerate}
%
%\begin{example}
%	The subcategory of $\Op(X)$ consisting of all open sets containing $x$ is cofiltered because $\Op(X)$ has all finite pullbacks. The second condition does not need to be verified, since there is at most one morphism between any two open sets.
%\end{example}
%
%\begin{example}
%	Let $X$ be a scheme and consider the structure sheaf $\mathcal{O}_X$ in the Zariski topology. The stalks of this sheaf, written $\mathcal{O}_{X,x}$ for $x \in X$ are local rings: By definition each point $x \in X$ is contained in an affine neighorhood $\Spec(A) \subset X$. We define a map from the stalk $\Sh{O}_p$ to the ring $A_p$, where $p$ is the localisation of $A$ at the the multiplicative subset $A\setminus (p)$ by sending a section of .
%	In the \'Etale topology, the limits $\varprojlim_{U \ni x} \mathcal{O}(U)$ turn out to be the strict henselizations of the local rings $\mathcal{O}_{X,x}$.
%\end{example}
%
%\section{Some categories and functors related to sheaves}
%
%Rough draft \par
%Sheaves are closely related to bundles. Because bundles are not locally trivial, we define vector bundles on schemes via sheaves. We prove that there is the sheaf-bundle adjunction and hopefully apply it.
%
%\subsection{Bundles and sheaves}
%
%A map $f: Y \to X$ of topological spaces is also called a bundle over $X$. Every bundle gives rise to the sheaf $Gamma_f$, defined by
%\[
%	\Gamma_f(U)	= \{ s : U \to f^{-1}(U) \mid f \circ s = id \}.
%\]
%This sheaf is called the \textit{sheaf of sections of $f$}. Each map $f \to g$ of bundles over $X$ induces a map of sheaves.
%\begin{remark}
%	It is not the case that the \'espace \'etale of an \'etale sheaf is again a scheme. I have heard this is a reason to consider algebraic spaces
%\end{remark}
%
%\begin{proposition}
%	Let $f: X \to Y$ be a morphism of schemes. The presheaf $\Gamma_f$ is a sheaf in the \'etale topology.
%\end{proposition}
%\begin{proof}
%
%\end{proof}
%
%\begin{definition}[Fiber bundles]
%	Let $X$ be a topological space. A \textit{fiber bundle over $X$ with fiber $F$} is a space $p: B \to X$ over $X$ such that for each point $x \in X$ there is an open neighborhood $U$ of $x$ such that $p^{-1}(U) \cong U \times F$. A cover $\{U_i\}$ is called \textit{a trivialization of $B$} if $p^{-1}(U_i) \cong U_i \times F$ for each $i$. We will write $U_{ij}$ instead of $U_i \cap U_j$ and similarly for $U_i \cap U_j \cap U_k$.
%\end{definition}
%\begin{remark}
%	Suppose we are given a cover $\{U_i\}$ of $X$. Under which conditions can we construct a fiber bundle $p: B \to X$ with fiber $F$ such that $\{U_i\}$ is a trivialization of $B$? We know that we want $B_i \coloneqq p^{-1}(U_i)\cong U_i \times F $, so $B$ should be glued together from the $B_i$. In order to do this we need to specify isomorphisms $\phi_{ij} : U_i \to U_j$ subject to the \textit{cocycle condition}:
%	\[(...)\]
%	.
%\end{remark}
%
%\begin{construction}
%	Each fiber bundle $B$ over $X$ with projection $p$ gives rise to a sheaf. For each open set $U$ of $X$ we can consider the set of sections
%	\[
%		\{s \in \mathcal{M}(U) \mid p \circ s = \text{id}\}.
%	\]
%	This forms a sheaf on $X$. We have seen (...) that fibre bundles are not locally trivial in the Zariski topology. For this reason the analogs of fibre bundles in algebraic geometry are defined using sheaves with a local triviality condition.
%\end{construction}
%\begin{example}
%	Let $p : M \to S^1$ be the M\"obius strip with fiber $[-1,1]$ from the introduction. Let $\mathcal{M}$ be the sheaf of sections of this bundle. There is a subsheaf $\mathcal{N}$ which consists of those sections which vanish nowhere:
%	\[
%		\mathcal{N}(U) = \{s : U \to U \times [-1,1] \mid \pi_1(s) \neq 0 \text{ for all } s \in U\}.
%	\]
%	This subsheaf has no global sections. This is because of the non-orientability of the M\"obius strip. We may choose a trivialisation $\{U_1, U_2, U_3\}$ of $M$ such that each $U_i$ is connected. Then the sign of sections $s_i: U_i \to U_i \times [-1,1]$ must be constant.
%\end{example}


\section{The global sections functor}
Let $X$ be a topological space and $F$ a sheaf on $X$. The elements of $F(X)$ are called \textit{global sections of $F$}. It is convenient to view $F$ as a variable and to write $\Gamma(X,-)$ as the functor which sends a sheaf $F$ to the set of its global section $\Gamma(X, F)$. Note that a sheaf may not have global sections. For example, the sheaf of sections of the double cover of a circle has no global sections, but locally  there are 3. Sheaf cohomology is all about applying the techniques from homological algebra to study the global sections of sheaves. Note that $X$ is the terminal object in both $\Op(X)$ and in $\Et/X$, so a global section for an \'etale sheaf is defined in the same way.

Consider the terminal object $1$ in the category $\Set^{C^{op}}$. It is the presheaf defined by $1(X) = \{*\}$ for all $X \in C$. A morphism of presheaves $\gamma: 1 \to P$ picks an element $\gamma_U$ for each object $U$ of $C$ such that the following diagram commutes for each $f: U \to U'$:
\[
	% https://q.uiver.app/?q=WzAsMyxbMSwwLCJcXHsqXFx9Il0sWzAsMSwiXFxTaHtGfShVJykiXSxbMiwxLCJcXFNoe0Z9KFUpIl0sWzAsMV0sWzEsMiwiZl8qIl0sWzAsMl1d
	\begin{tikzcd}
		& {\{*\}} \\
		{\Sh{F}(U')} && {\Sh{F}(U)}
		\arrow[from=1-2, to=2-1]
		\arrow["{f_*}", from=2-1, to=2-3]
		\arrow[from=1-2, to=2-3]
	\end{tikzcd}
\]
We obtain a functor $\Gamma: \Set^{C^{op}} \to \Set, \Gamma(P) = \Hom(1, P)$. Conversely, we can assign to each set $S$ the constant presheaf $\Delta C$ by setting $\Delta C(U) = S$ and letting all restrictions be identities. There are natural isomorphisms
\[
	\Hom_{\widehat{C}}(\Delta C, P) \cong \Hom(S, \Gamma P),
\]
so the global sections functor is left adjoint to the constant presheaf functor.
Since the inclusion of sheaves into presheaves has a left adjoint it preserves limits. In particular, terminal object $1$ is also a sheaf. Since adjoint functors compose, it follows that the functor $\Gamma : \mathsf{Sh}(C,J) \to \Set$ is right adjoint to the composition $a \circ \Delta: \Set \to \mathsf{Sh}(C,J)$.


For example, when $X$ is a topological space, $\Hom(1,F)(X)$ consists of all global sections of the sheaf $\Sh{F}$. Here $\Hom(1,F)$ is in the category $\mathsf{Sh}(C,J)$

\begin{theorem}
	There is an adjoint pair of functors
	% https://q.uiver.app/?q=WzAsMixbMCwwLCJcXG1hdGhzZntTaH0oQyxKKSJdLFsxLDAsIlxcU2V0Il0sWzEsMCwiXFxEZWx0YSIsMCx7Im9mZnNldCI6LTF9XSxbMCwxLCJcXEdhbW1hIiwwLHsib2Zmc2V0IjotMX1dXQ==
	\[
		\begin{tikzcd}
			{\mathsf{Sh}(C,J)} & \Set
			\arrow["\Delta", shift left=1, from=1-2, to=1-1]
			\arrow["\Gamma", shift left=1, from=1-1, to=1-2]
		\end{tikzcd}
	\]
	The global sections functor $\Gamma: \mathsf{Sh}(C,J) \to \Set,\ \Gamma(F) = \Hom(1,F)$ is right adjoint to the constant sheaf functor $\Delta: \Set \to \Sh(C,J)$.
\end{theorem}

\subsection{Subobject Classifiers for \'Etale Toposes}
In proposition~\ref{prop:classifier} we showed that for a topological space $X$, the category $\sh(X)$ of sheaves on $X$ has a subobject classifier. In this section we construct a subobject classifier for categories of sheaves on an arbitrary site.

\begin{definition}
	Let $(C,J)$ be a site. We define a subfunctor $S \subset \yo(A)$ to be \textit{closed for $J$} if for all morphisms $f \colon B \to A$
	\[
		\{h \colon D \to B \mid fh \in S(D) \} \in J(D) \implies f \in S(B).
	\]
\end{definition}

The property of being closed in stable under pullback. Indeed, let $S$ be a subfunctor of $\yo(A)$ and $g \colon B \to A$ an arbitrary morphism. Suppose that $h^*(S)$ covers $f$ for a morphism $f \colon C \to B$. By definition $S$ covers $hf$. Since $S$ is closed, we have $hf \in S(C)$. Since $S$ is closed, $hf \in S(C)$ or $f \in h^*S(B)$. Thus $h^*S$ is closed and the definition \[\Omega(A) = \{S \subset \yo(A) \mid S \text{ is closed }\}\] defines a presheaf on $C$, with the restriction along a morphism $h \colon B \to C$ given by $S \to h^*S$.

\begin{proposition}
	The presheaf $\Omega \colon C^{\op} \to \Set$ is a sheaf and in fact a subobject classifier for $\sh(C,J)$ when taken together with the map
	\[\true \colon 1 \to \Omega \] where $\true$ is the natural transformation given on objects by $A \to \yo(A)$.
\end{proposition}

\begin{proof}
	See \cite{SIGL}, III.7. Lemma 1.
\end{proof}

