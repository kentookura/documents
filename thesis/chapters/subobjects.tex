\subsection{Presheaves}
Recall that a presheaf $P$ of sets on a category $C$ is simply a functor $P \colon C^{\op} \to \Set$. For an object $A$ of $C$, we call the elements $s \in P(A)$ the sections of $P$ over $A$. The functor category $\Set^{C^{\op}} = \widehat{C}$ has as objects presheaves $P \colon C^{\op} \to \Set$ and as morphisms natrual transformations $P \to P'$. For each object $X$ of $C$ there is a presheaf $\yo(X) = \Hom_C(-,X)$, which is defined on an object by $\yo(X)(Y) = \Hom_C(Y,X)$ and for morphisms $f \colon Y' \to Y$ and $u \in \Hom_C(Y,X)$ by
\begin{gather*}
	\yo(X)(f) \colon \Hom_C(Y,X) \to \Hom_C(Y', X)\\
	\yo(X)(f)(u) = u \circ f
\end{gather*}
A presheaf isomorphic to one of the form $\yo(A)$ is called representable, with $A$ being the representing o bject. The Yoneda lemma implies that this object is unique up to isomprhism. Recall that the Yoneda lemma says that there is a bijectivetion between natural transformations $\yo(A) \to P$ and elements of $P(A)$, where $P$ is an arbitrary presheaf. If $f \colon A \to B$ is a morphism in $C$, then there is a natural transformation $\yo(A) \to \yo(B)$ given by composing with $f$. We obtain a fully faithful functor
\[\yo \colon C \to \Set^{C^{\op}},\quad A \to \Hom_C(-,A)\]
called the yoneda embedding.

\begin{construction}[Subobjects]
	Recall that a morphism $f \colon A \to B$ in some category $C$ is called a monomorphism if for any object $C$ and parallel arrows $g, h \colon C \xbigtoto{} A$ with $fg = fh$ implies $g = h$. We consider two monomorphisms $f \colon A \to B$ and $f' \colon A' \to B$ to be equivalent if there is an isomophism $A \simeq A'$ with $f'h = f$. A subobject of an object $A$ of $C$ is an equivalence class of monomorphisms with codomain $A$. The collection of subobjects of $A$, denoted by $\mathbf{Sub}_C(A)$ is partially ordered, where $[f] \le [g]$ if and only if there is a morphism $k \colon S \to T$ such that $f = gk$, where $[f]$ and $[g]$ are the classes of $f$ and $g$.
\end{construction}

\begin{example}[Subfunctors]
	Let $P$ be a presheaf on $C$. A subfunctor $S$ of $P$ is defined to be a presheaf such that for each object $A$, the set $S(A)$ is a subset of $P(A)$ and each map $S(f): S(B) \to S(A)$ is a restriction of $P(f)$ for all morphisms $f \colon A \to B$.
\end{example}

\begin{definition}
	In a category $C$ with finite limits a \textit{subobject classifier} is a monic $\true \colon 1 \to \Omega$ such that for every monic $S \to X$ there is a unique morphism $\phi$ such that diagram below is a pullback square.
	\[
		\begin{tikzcd}
			S & 1 \\
			A & \Omega
			\arrow[tail, from=1-1, to=2-1]
			\arrow[from=1-1, to=1-2]
			\arrow["{\true}", from=1-2, to=2-2]
			\arrow["\phi"', from=2-1, to=2-2]
		\end{tikzcd}
	\]
\end{definition}

\begin{proposition}
	In a category with a subobject classifier, the object $\Omega$ is a representing object for the functor $A \to \mathbf{sub}_C(A)$ and we have an isomoprhism
	\[ \mathbf{sub}_C(A) \simeq \Hom_C(A,\Omega).\]
	Furthermore, the object $\mathbf{1}$ is the final object in $C$.
\end{proposition}

\begin{proof}
	See \cite{SIGL}, I.3.1.
\end{proof}

\subsection{Subobjects in Presheaf Categories}\label{section:presheaf_subobjects}

\begin{construction}
	Let $\widehat{C}$ be a presheaf category. Suppose $\widehat{C}$ has a subobject classifier $\Omega$. The subobjects of a representable $\yo(A)$ must be given by
	\[
		\sub_{\widehat{C}}(\yo(A)) \simeq \Hom_{\widehat{C}}(\yo(A), \Omega) \simeq \Omega(A)
	\]
	Thus the subobject classifier of $\widehat{C}$ is given by the presheaf $A \to \Omega(A)$ where we have the chain of equalities:
	\[
		\Omega(A)  = \sub_{\widehat{C}}(\Hom_C(-,A))  = \{S \mid S \text{ is a subfunctor of $\yo(A)$}\}.
	\]
\end{construction}

%There is an alternative description of subfunctors which we will make use of in sheaf theory.

%\begin{definition}[Sieves]
%	Let $C$ be a category and $A$ an object of $C$. A \textit{sieve} on $A$ is a set $S$ of morphisms with codomain $A$ such that $fh \in S$ whenever $f \in S$ and $fh$ is defined.
%\end{definition}
%
%\begin{remark}
%	If $S$ is a subfunctor $\yo(U)$ and $f \colon V \to U$ is a morphism, then the set
%	\[
%		f^*(S) = \{ g \mid g \text{ has codomain } V \text{ and the composite }fg \text{ is in } S\}
%	\]
%	is a defines a subfunctor of $\yo(V)$.
%\end{remark}
%
\begin{example}
	Let $\Op(X)$ be the poset of opens of a topological space $X$. Consider the representable functor $\yo(U) = \Hom(-,U)$. If $V \subset U$, then $\Hom(V,U)$ is the one element set, as $\Op(X)$ is a poset. If $V \not\subset U$, then $\Hom(V,U) = \varnothing$. Now let $S$ be a subfunctor of $\Hom(-,U)$. Then $S$ may be equivalently described as the set of all those $V \subset U$ with $S(V) = 1$ or as the subset $T \subset \Op(U)$ of those open sets such that $W \subset V \in T$ implies $W \in S$.
\end{example}

Every sieve $S$ on $A$ determines a subfunctor of $\yo(A)$ and conversely, every subfunctor of $\yo(A)$ determins a sieve on $A$. Given a subfunctor $T \subset \yo(A)$, the set
\[
	S = \{f \mid f \in S(B) \text{ for some object } B\}
\]
is a sieve on $A$. Conversely a sieve $S$ on $A$ determines a subfunctor $T$  of $\yo(A)$ by defining
\[
	T(B) = \{ f \mid f \in S \text{ and } f \text{ has domain } B\} \subset \yo(A)(B).
\]
Since these associations are inverses to each other, we can equivalently define the subobject classifier of a presheaf category $\Set^{C^{\op}}$ as the presheaf which to an object $A$ the set of sieves on $A$.

\begin{remark}
	Returning to the example of the category of opens $\Op(X)$, we can define a covering sieve for $U$ to be a sieve $S$ on $U$ such that $U$ is the union of all the open sets $V \in S$.
\end{remark}

\begin{definition}[Sheaves on Topological Spaces]\label{def:sheaves}
	Let $X$ be a topological space. A presheaf $F$ on $X$ is a \textit{sheaf}, if for any open covering $\{U_i\}$ of $U$ and any family of section $x_i \in F(U_i)$ such that $x_i|_{U_i \cap U_j} = x_j|_{U_i \cap U_j}$ there is a unique $x \in U = \bigcup U_i$ such that $x|_{U_i} = x_i$. A subsheaf of $F$ is simply a subfunctor of $F$ which is also a sheaf.
\end{definition}

\begin{proposition}\label{prop:classifier}
	The subobject classifier $\Omega$ for the category of sheaves $\sh(X)$ on a space $X$ is given on objects $U \subset X$ by
	\[
		\Omega(U) = \{ V \mid V \text{ is an open subset of } U \}
	\]
	and on morphisms $V \subset U$ by
	\[
		\Omega(U) \to \Omega(V),\quad W \to W \cap V.
	\]
\end{proposition}

\begin{proof}
	We first show that $\Omega$ is indeed a sheaf. Let $\{U_i\}$ be an open cover of $U$ and let $V_i \in \Omega(U_i)$ be sections with $V_i \cap U_j = V_j \cap U_i \subset U_i \cap U_j$. Then the union of the $V_i$ is a section of $\Omega$ over $U$ with the property that $V \cap U_i = V_i$. To show that $\Omega$ is a subobject classifier for $\sh(X)$, consider a subobject $S \subset F$. By the definition of subobject classifier, this subobject should arise from a characteristic natural transformation $\phi \colon F \to \Omega$. Define this function elementwise by the function
	\begin{gather*}
		\phi \colon F(U) \to \Omega(U)\\
		\quad x \to \bigcup W_i
	\end{gather*}
	where the $W_i$ are the open subsets of $U$ such that the restriction of $x$ to $W_i$ lies inside of $S(W_i)$. This map is natural in $U$ and because $S$ is a subsheaf, $x|_W$ lies in $S(W)$. Now for each $U$  we can pull back the map $\phi_U$ along the map $\true \colon 1 \to \Omega(U)$ which picks out $U \in \Omega(U)$:
	\[
		\begin{tikzcd}
			P(U) & 1 \\
			F(U) & \Omega(U)
			\arrow[from=1-1, to=2-1]
			\arrow[from=1-1, to=1-2]
			\arrow["\true", from=1-2, to=2-2]
			\arrow["\phi_U"', from=2-1, to=2-2].
		\end{tikzcd}
	\]
	The constructed pullback $P(U)$ is equal to $\{x \in F(U) \mid \phi_U x = U\} \subset F(U)$, and since limits and in particular pullbacks are constructed pointwise in presheaves the ``total'' pullback
	\[
		\begin{tikzcd}
			P & 1 \\
			F & \Omega
			\arrow[from=1-1, to=2-1]
			\arrow[from=1-1, to=1-2]
			\arrow["{\true}", from=1-2, to=2-2]
			\arrow["\phi"', from=2-1, to=2-2]
		\end{tikzcd}
	\]
	is indeed the subsheaf $S$. The map $\phi$ is also unique, so the map $\true: 1 \to \Omega$ is a subobject classifier.
\end{proof}
